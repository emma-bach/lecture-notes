\documentclass{report}
\usepackage[a4paper,margin=1in]{geometry}
\usepackage{fancyhdr}
\usepackage[titles]{tocloft}
\usepackage{tikz}
\usepackage{xcolor}

\usepackage{multicol}
\usepackage{amsmath}
\usepackage{amssymb}
\usepackage{amsthm}
\usepackage{pdfpages}
\usepackage{bm}
%\usepackage{bbm}
%\usepackage{biblatex}
%\addbibresource{bib.bib}

%hyperref should be last apparently
\usepackage{hyperref}

\renewcommand\cftsecdotsep{\cftdot}
\renewcommand\cftsubsecdotsep{\cftdot}
\renewcommand\epsilon{\varepsilon}

% Starts a new paragraph without indentation
% and with an empty line between paragraphs
\newcommand*{\newpar}{\par\vspace{\baselineskip}\noindent}
\newcommand{\trans}{\twoheadrightarrow}

\pagestyle{fancy} %allows headers

\lhead{Emma Bach}
\rhead{\today}


\begin{document}
\newtheorem{lemma}{Lemma}[section]
\newtheorem{theorem}[lemma]{Theorem}

\theoremstyle{definition}
\newtheorem{definition}[lemma]{Definition}
\newtheorem{example}[lemma]{Example}
%
%
%
\begin{titlepage}
	\centering
	{\Large \textsc{Mitschrieb}\par}
	\vspace{0.5cm}
	{\huge\bfseries Numerik\par}
    \vspace{0.5cm}
	{\Large\itshape Emma Bach\par}
	\vfill
	
	Basierend auf:\par 
	\vspace{1cm}
	Vorlesungen Numerik I + II von\par
	Prof. Dr. Patrick  \textsc{Dondl}\par
	
	\vfill

% Bottom of the page
	{\large \today\par}
\end{titlepage}

\tableofcontents
\thispagestyle{fancy}
%
%
%
\chapter*{Preface}
\addcontentsline{toc}{chapter}{Preface}
These notes are not based on any particular lecture I attended. Instead, they were created during self-study of theoretical topics related to functional programming and the $\lambda$-calculus.
\newpar
My primary source is the ``Handbook of Logic in Computer Science'', edited by Samson Abramsky, Dov  Gabbay and Thomas Maibaum, in particular volume 2, which deals with computational structures \cite{handbookvol2}.  Additional sources I found helpful include the video ``What is plus times plus?'' by YouTube channel 2Swap, which is a wonderful introduction to the $\lambda$-calculus \cite{2swap}.
%
%
%
\chapter{Term Rewriting Systems}
\section{Abstract Reduction Systems}
\begin{definition}
\hphantom{}
\begin{enumerate}
 \item An \textbf{abstract reduction system} is a structure $\mathcal{A} = \langle A, \{\rightarrow_{\alpha}\}_{\alpha \in I}\rangle$, consisting of a set of terms $A$ and a set of binary relations $\rightarrow_{\alpha} \in A^2$, called the \textbf{reduction relations} or the \textbf{rewrite relations}. As shorthand, the relation $\rightarrow_{\alpha}$ is occasionally written simply as $\alpha$. An abstract reduction system with only one relation $\rightarrow$ is also known as a \textbf{replacement system} or a \textbf{transformation system}.
 \item If $a \rightarrow_\alpha b$, then $b$ is known as a \textbf{one step ($\alpha$-) reduct} of $a$, meaning that $b$ can be created by applying the reduction $\rightarrow_\alpha$ to $a$ once.
 \end{enumerate}
\end{definition}
%
Note that we let $\rightarrow_\alpha$ be any arbitrary relation, without additional demands of transitivity, reflexivity, or anything else of the sort.
\begin{example}
As an example, we could consider a term rewriting system that reduces sums of natural numbers. To write this system as an abstract reduction system, we can define $A$ and $\rightarrow$ as follows:
\begin{itemize}
 \item Let $A$ be the smallest set of terms such that:
 \begin{itemize}
 \item Every natural number $n \in \mathbb{N}$ is a term.
 \item If $t_1$ and $t_2$ are terms, then $t_1 + t_2$ is a term.
 \end{itemize}
 \item We define the following reduction relations $\to_0$ and $\to_1$:
 \begin{itemize}
  \item If $n$ and $m$ are natural numbers and $k = n+m$, then $n + m \rightarrow_0 k$
  \item If $t_1 \to_0 k$, then $t_1 + t_2 \to_{1} k + t_2$
 \end{itemize}
\end{itemize}
We could now apply these reduction rules to reduce a sum:
\begin{equation*}
 10+5+2+7 \to_1 15 + 2 + 7 \to_1 17 + 7 \to_0 24
\end{equation*}
The term $15+2+7$ is a one step $1$-reduct of the term $10+5+2+7$, while the term $17+7$ is a one-step $1$-reduct of $15+2+7$ and therefore a two step $1$-reduct of $10+5+2+7$.
\end{example}
%
\begin{definition}
Given a reduction relation $\to_{\alpha}$, we are often interested in investigating new relations created by extending $\to_{\alpha}$. Some relations that are of interest to us include:
\begin{itemize}
 \item The \textbf{transitive reflexive closure} $\trans_\alpha$, defined as the smallest transitive relexive relation containing $\to_\alpha$. To give a more explicit definition, we have $a \twoheadrightarrow_\alpha b$ iff:
  \begin{enumerate}
   \item $a \to_\alpha b$, or
   \item $a = b$, or
   \item $\exists n : \exists c_1, \hdots, c_n : a \to_\alpha c_1 \to_\alpha \hdots \to_\alpha c_n \to_\alpha b$
  \end{enumerate}
  In most of the literature, this closure would be denoted as $\to_\alpha^*$. However, the book uses $\trans_\alpha$ because it makes transitive diagrams involving the relation more legible.\
  \item The \textbf{convertibility relation} $=_\alpha$ generated by $\to_\alpha$, defined as the smallest equivalence relation containing $\to_\alpha$. Once again, to give a more explicit definition:
  \begin{center}
   We have $a =_\alpha b$ iff. $\exists u_1, \hdots, u_n \in A$ such that $a = u_1$, $b = u_n$,\\ and for each $u_i$, we have either $u_i \to_\alpha u_{i+1}$ or $u_{i+1} \to_\alpha u_i$.
  \end{center}
  Note that even though $\trans_\alpha$ is the smallest transitive reflexive relation that contains $\to_\alpha$ and $=_\alpha$ is the smallest equivalence (transitive, reflexive, and symmetric) relation that contains $\to_\alpha$, $=_\alpha$ is \textbf{not} equivalent to the symmetric closure of $\trans_\alpha$, as can be seen in Example \ref{ex:symmetricclosure}.
\end{itemize}
%
\begin{example}
\label{ex:symmetricclosure}
Let $\to_\alpha = \{(a,c), (b,c), (c,d)\}$. Then:
\begin{enumerate}
 \item $\trans_\alpha = \{(a,a), (b,b), (c,c), (d,d), (a,c), (a,d), (b,c), (b,d), (c,d)\}$
 \item The symmetric closure of $\trans_\alpha$ would be\\ $\trans_\alpha^s = \{(a,a), (b,b), (c,c), (d,d), (a,c), (c,a), (a,d), (d,a), (b,c), (c,b), (b,d), (d,b), (c,d), (d,c)\}$. Note that this relation is no longer transitive, since it includes $(a,c)$ and $(c,b)$, but not $(a,b)$.
 \item $=_\alpha$ would include every single pair of these elements.
\end{enumerate}
In short, if $a \to_\alpha c \leftarrow_\alpha b$, then we have $a =_\alpha b$, but not necessarily $a \trans_\alpha^s b$.
\end{example}
%
\end{definition}
%
%
%
%
\bibliography{refs}{}
\bibliographystyle{plain}
\end{document}
