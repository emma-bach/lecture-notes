\documentclass{report}

% custom margins
\usepackage[a4paper,margin=1.5in]{geometry}
\renewcommand{\baselinestretch}{1.2}

% emma's long list of custom macros and universally used packages
%AMS packages - Symbols, "Theorem" and "Proof" Environments
\usepackage{amsmath}
\usepackage{amssymb}
\usepackage{amsthm}
\usepackage{physics}
\renewcommand{\div}{\tn{div}~}

% Nicer Headers and Footers
\usepackage{fancyhdr}
\usepackage[yyyymmdd]{datetime}
\renewcommand{\dateseparator}{-}

% Table of Contents
\usepackage[titles]{tocloft}
\usepackage[titletoc]{appendix}

% Tikz and Graphics
\usepackage{tikz}
\usepackage{tikz-cd}
\usepackage{xcolor}
\usepackage[many]{tcolorbox}

% Nicer Underlining
\usepackage{contour}
\usepackage{ulem}

% Multiple Text Columns
\usepackage{multicol}

% FloatBarrier
\usepackage{placeins} 

% hyperref should be last apparently
\usepackage{hyperref}

%--- RENEWED COMMANDS ---

% Variant Greek Letters
\renewcommand\epsilon{\varepsilon}
\renewcommand\phi{\varphi}

% Nicer Table of Contents
\renewcommand\cftsecdotsep{\cftdot}
\renewcommand\cftsubsecdotsep{\cftdot}

%--- TEXT FORMATING ---

% nice underlining
\renewcommand{\ULdepth}{1.6pt}
\contourlength{0.8pt}
\newcommand{\ul}[1]{%
	\uline{\phantom{#1}}%
	\llap{\contour{white}{#1}}%
}

% shorthand
\newcommand{\tn}[1]{\textnormal{#1}}
\newcommand{\tbf}[1]{\textbf{#1}}
\newcommand{\tit}[1]{\textit{#1}}
\newcommand{\ttt}[1]{\texttt{#1}}

\newcommand{\mc}[1]{\mathcal{#1}}

\newcommand{\ol}[1]{\overline{{#1}}}

% Underlined, bold, non-cursive Theorem Name
\newcommand{\theoremname}[1]{\ul{\textnormal{\tbf{#1}}}}

% Starts a new paragraph without indentation
% and with an empty line between paragraphs
\newcommand*{\newpar}{\par\vspace{\baselineskip}\noindent}

% Make inf height match sup height
\renewcommand{\inf}{\mathop{\mathrm{inf}\vphantom{\mathrm{sup}}}}

% Make tildes more readable
\renewcommand{\tilde}{\widetilde}

%--- RELATION SYMBOLS AND OPERATORS ---
\newcommand{\surj}{\twoheadrightarrow}
\newcommand{\inj}{\hookrightarrow}
\newcommand{\iso}{\overset{\sim}{\rightarrow}}
\newcommand{\symdiff}{\vartriangle}
\newcommand{\trans}{\twoheadrightarrow}
\newcommand{\tensor}{\otimes}
\newcommand{\bigmid}{~\middle|~}

%--- STUFF IN FANCY BRACKETS ---
\newcommand{\scalar}[2]{\left\langle #1, #2 \right\rangle}
\newcommand{\angles}[1]{\left\langle #1 \right\rangle}
\newcommand{\lr}{\qty}


%--- LETTERS ---
\newcommand{\dmu}{\ d\mu}
\newcommand{\dist}{\textnormal{dist}}
\newcommand{\spt}{\textnormal{spt}}

% mathbb
\newcommand{\bC}{\mathbb{C}}
\newcommand{\bF}{\mathbb{F}}
\newcommand{\bN}{\mathbb{N}}
\newcommand{\bQ}{\mathbb{Q}}
\newcommand{\bR}{\mathbb{R}}
\newcommand{\bZ}{\mathbb{Z}}

% mathcal
\newcommand{\cA}{\mathcal{A}}
\newcommand{\cB}{\mathcal{B}}
\newcommand{\cC}{\mathcal{C}}
\newcommand{\cD}{\mathcal{D}}
\newcommand{\cE}{\mathcal{E}}
\newcommand{\cF}{\mathcal{F}}
\newcommand{\cH}{\mathcal{H}}
\newcommand{\cI}{\mathcal{I}}
\newcommand{\cL}{\mathcal{L}}
\newcommand{\cM}{\mathcal{M}}
\newcommand{\cN}{\mathcal{N}}
\newcommand{\cO}{\mathcal{O}}
\newcommand{\cP}{\mathcal{P}}
\newcommand{\cQ}{\mathcal{Q}}
\newcommand{\cR}{\mathcal{R}}
\newcommand{\cS}{\mathcal{S}}
\newcommand{\cT}{\mathcal{T}}
\newcommand{\cW}{\mathcal{W}}
\newcommand{\cX}{\mathcal{X}}
\newcommand{\cZ}{\mathcal{Z}}

% mathfrac
\newcommand{\fa}{\mathfrak{a}}

% real numbers in fancy costumes
\newcommand{\barR}{\ol{\mathbb{R}}}
\newcommand{\bRnn}{\mathbb{R}^{n \times n}}
\newcommand{\bRmn}{\mathbb{R}^{m \times n}}
\newcommand{\bRnm}{\mathbb{R}^{n \times m}}

% vectors
\renewcommand{\va}{\vec{a}}
\renewcommand{\vb}{\vec{b}}
\newcommand{\vc}{\vec{c}}
\newcommand{\ve}{\vec{e}}
\newcommand{\vF}{\vec{F}}
\newcommand{\vh}{\vec{h}}
\newcommand{\vp}{\vec{p}}
\newcommand{\vr}{\vec{r}}
\newcommand{\vs}{\vec{s}}
\newcommand{\vT}{\vec{T}}
\renewcommand{\vu}{\vec{u}}
\newcommand{\vv}{\vec{v}}
\newcommand{\vw}{\vec{w}}
\newcommand{\vW}{\vec{W}}
\newcommand{\vx}{\vec{x}}
\newcommand{\vy}{\vec{y}}
\newcommand{\vz}{\vec{z}}
\newcommand{\vzero}{\vec{0}}

\newcommand{\veta}{\vec{\eta}}

\renewcommand{\grad}{\vec{\nabla}}

% bold letters
\newcommand{\tbA}{\mathbf{A}}
\newcommand{\tbB}{\mathbf{B}}
\newcommand{\tbC}{\mathbf{C}}
\newcommand{\tbD}{\mathbf{D}}
\newcommand{\tbE}{\mathbf{E}}
\newcommand{\tbY}{\mathbf{Y}}
\newcommand{\tbZ}{\mathbf{Z}}

% sequences
\newcommand{\an}{(a_n)_{n \in \bN}}
\newcommand{\bn}{(b_n)_{n \in \bN}}
\newcommand{\sn}{(s_n)_{n \in \bN}}
\newcommand{\iinI}{_{i \in I}}
\newcommand{\iinN}{_{i \in \bN}}

% special functions
\newcommand{\at}{\textnormal{at}}
\newcommand{\ggT}{\textnormal{ggT}}
\newcommand{\kgV}{\textnormal{kgV}}
\newcommand{\id}{\textnormal{id}}
\newcommand{\Id}{\textnormal{Id}}
\newcommand{\im}{\textnormal{im}}
\newcommand{\inv}{\textnormal{inv}}
\newcommand{\ord}{\textnormal{ord}\ }
\newcommand{\rang}{\textnormal{rang}\ }
\renewcommand{\tr}{\textnormal{tr}\ }
\newcommand{\vol}{\textnormal{vol}}
\newcommand{\cond}{\textnormal{cond}}
\newcommand{\sgn}{\textnormal{sgn}}
\renewcommand{\char}{\textnormal{char}}
\newcommand{\Frac}{\textnormal{Frac}}
\newcommand{\Irr}{\textnormal{Irr}}

% groups and sets
\newcommand{\GL}{\text{GL}}
\newcommand{\SO}{\text{SO}}
\newcommand{\Hess}[1]{\text{Hess}(#1)}

\newcommand{\Grp}{\textnormal{Grp}}
\newcommand{\Mag}{\textnormal{Mag}}
\newcommand{\Mon}{\textnormal{Mon}}
\newcommand{\Ens}{\textnormal{Ens}}
\newcommand{\Hom}{\textnormal{Hom}}
\newcommand{\Aut}{\textnormal{Aut}}

\newcommand{\Mat}{\textnormal{Mat}}
\newcommand{\Matnn}{\textnormal{Mat}(n \times n)}
\newcommand{\MatBB}[3]{\textnormal{Mat}^{#1}_{#2}\left(#3\right)}



% --- THEOREM AND PROOF TYPES ---
% \newtheorem{codename}{printedname}[countedwith]
\newtheorem{lemma}{Lemma}[section]
\newtheorem{theorem}[lemma]{Satz}
\newtheorem{proposition}[lemma]{Proposition}
\newtheorem{corollary}[lemma]{Korollar}

\theoremstyle{definition}
\newtheorem{definition}[lemma]{Definition}
\newtheorem{summary}[lemma]{Zusammenfassung}
\newtheorem{example}[lemma]{Beispiel}
\newtheorem{counterexample}[lemma]{Gegenbeispiel}
\newtheorem{beobachtung}[lemma]{Beobachtung}
\newtheorem{anmerkung}[lemma]{Anmerkung}
\newtheorem{question}[lemma]{Frage}
\newtheorem{application}[lemma]{Anwendung}
\newtheorem{reminder}[lemma]{Erinnerung}
\newtheorem{konsequenz}[lemma]{Konsequenz}

% --- CUSTOM ENVIRONMENTS ---

% sketchy proofs marked with untrustworthy QED symbol
\newenvironment{proofsketch}{\begin{proof}[Beweisskizze]\renewcommand*{\qedsymbol}{\("\square"\)}}{\end{proof}}

% align with exactly one number for labeling
\NewDocumentEnvironment{nalign}{}{\equation\aligned}{\endaligned\endequation}

\makeatletter
\let\amsmath@bigm\bigm

% fancy box behind proofs
\tcolorboxenvironment{proof}{
	colback=black!5!white,
	boxrule=0pt,
	sharp corners,
	breakable,
	enhanced,
}

\renewcommand*\contentsname{Inhalt}
\renewcommand*\proofname{Beweis\newline}

\pagestyle{fancy} %allows headers

\lhead{Emma Bach}
\rhead{\today}

\begin{document}
	\begin{titlepage}
	\centering
	{\Large \textsc{Mitschrieb}\par}
	\vspace{0.5cm}
	{\huge\bfseries Numerik\par}
    \vspace{0.5cm}
	{\Large\itshape Emma Bach\par}
	\vfill
	
	Basierend auf:\par 
	\vspace{1cm}
	Vorlesungen Numerik I + II von\par
	Prof. Dr. Patrick  \textsc{Dondl}\par
	
	\vfill

% Bottom of the page
	{\large \today\par}
\end{titlepage}

	\tableofcontents
	\thispagestyle{fancy}
	\chapter{Mechanische Grundlagen}
	\section{Grundlegende Größen der Physik}
	Annahme: Vektorraum $\bR^3$
	\begin{itemize}
		\item \tbf{Skalare Größen}: $\bR^3 \to \bR$
		\item \tbf{Vektorwertige Größen}: $\bR^3 \to \bR^3$
		\item Notation für Ableitungen: 
		\begin{align*}
			v(t_a) = \frac{ds}{dt}\mid_{t_a}
		\end{align*}
		\item In dieser Vorlesung werden Ableitungen meist durch Differenzenquotienten approximiert. Wir differenziere solche Approximationen von den tatsächlichen Ableitung durch Verwendung des Symbols $\delta$ statt $d$.
		\item Gekrümmte Bewegung vereinfacht als \tbf{Weg-Zeit-Diagramm} $s(t)$. Steile Tangente $\to$ hohe Geschwindigkeit.
		\item Keine Bewegung: $s(t) = c$
		\item Gleichförmige Bewegung: $s(t) = vt = v(t - t_0) + s_0$,\newpar
			  Spezialfall Kreisbewegung: 
			  $
			  \begin{pmatrix}
			  		x(t)\\
			  		y(t)
		  	   \end{pmatrix} = r \begin{pmatrix}
		  	   		\cos(\omega t)\\
		  	   		\sin(\omega t)
	  	   	   \end{pmatrix}$
		\item Gleichmäßige Beschleuningung: $s(t) = \frac{1}{2}at^2$
	\end{itemize}
	\begin{figure}[h!]
		\centering
		\includegraphics[scale=0.3]{images/zentripetalkraft.png}
		\caption{Herleitung der Zentripetalbeschleunigung durch ähnliche Dreiecke}
		\label{zentripetalbeschleunigung}
	\end{figure}
	In Abbildung \ref{zentripetalbeschleunigung} ist dargestellt, wie man bei einer gleichmäßigen Kreisbewegung die Zentripetalbeschleunigung durch eine Konstruktion ähnlicher Dreiecke bestimmt werden kann: Die Geschwindigkeit ist konstant, und gleichzeitig immer orthogonal zum Radiusvektor. Es folgt also:
	\begin{align*}
		\frac{\Delta v}{v} = \frac{\Delta s}{r}
	\end{align*}
	Es folgt:
	\begin{align*}
		\frac{\Delta v}{\Delta t \cdot v} = \frac{\Delta s}{\Delta t \cdot r} = \frac{v}{r}
	\end{align*}
	\FloatBarrier
	\subsection{Newtonsche Gesetze}
	\begin{enumerate}
		\item[\tbf{1.}] \tbf{Newtonsches Axiom: Trägheitsprinzip} - Körper, auf die keine Kraft wirkt, bewegen sich geradlinig mit konstanter Geschwindigkeit (möglicherweise Geschwindigkeit $0$).
		\item[\tbf{2.}] \tbf{Newtonsches Axiom: Aktionsprinzip} - Die Änderung der Bewegung unter Einwirkung einer bewegenden Kraft ist proportional zu der Kraft und geschieht in Richtung dieser:
		\begin{align*}
			\vF = m \cdot \va
		\end{align*}
		\item[\tbf{3.}] \tbf{Newtonsches Axiom: Reaktionsprinzip} -  Übt ein Körper $A$ auf einen anderen Körper $B$ eine Kraft $\vF_{A \to B}$ aus (\textit{actio}), so wirkt eine gleich große, entgegengesetzte Kraft $\vF_{B \to A}$ von $B$ auf $A$. (\textit{reactio}):
		\begin{align*}
			\vF_{A \to B} = - \vF{A \to B}
		\end{align*}
	\end{enumerate}
	Die Kraft, welche auf einen Körper wirkt, ist die Summe aller Teilkräfte, die von allen Quellen aus auf den Körper wirken. Wirken auf einen Körper also exakt entgegengesetzte Kräfte, so befindet sich der Körper im \tbf{Kräftegleichgewicht} und bewegt sich nicht.
	\subsection{Druck}
	Der auf eine Fläche einwirkende Druck ist das Ergebnis einer Kraft $\vF$, die auf eine Fläche $A$ wirkt:
	\begin{align*}
		\vp = \frac{\vF}{A}
	\end{align*}
	Die Einheit des Drucks ist $\text{Pa} = \frac{\text{N}}{\text{m}^2} = \frac{\text{kg}}{\text{m} \cdot \text{s}^2}$
	\subsection{Arbeit}
	\tbf{Arbeit} misst den Transfer an Energie, der innerhalb eines Systems stattfindet, wenn eine Kraft $\vF(t)$ einen Körper (in der Mechanik eine Masse) über einen Weg $\vs(t)$ bewegt.
	\newpar
	Ist die Kraft konstant und der Weg geradlinig, wird die Arbeit einfach durch das Skalarprodukt gemessen:
	\begin{align*}
		W = \vF \cdot \vs
	\end{align*}
	Im Allgemeinen muss man einen Kurvenintegral der Kraft entlang des Weges berechnen:
	\begin{align*}
		W = \int_{t_0}^{t_1} \vF d\vs = \int_{t_0}^{t_1} \vF(t) \cdot \vv(t) dt
	\end{align*}
	\subsection{Energie}
		\begin{align*}
		W = \Delta  E
	\end{align*}
	z.B:
	\begin{itemize}
		\item Potentielle Energie: $W = m \cdot g \cdot \Delta h$ $\to$ $E_{pot} = m \cdot g \cdot h$
		\item Kinetische Energie: $E_{kin} = \frac{1}{2}mv^2$
		\item Thermische Energie: $Q = m \cdot c \cdot \Delta T$, wobei $c$ die Wärmekapazität gibt.
	\end{itemize}
	Energie und Arbeit werden beide in Joule angegeben. Einer der wichtigsten Sätze der Physik ist der Energieerhaltungssatz - die Menge an Energie in einem System ist immer konstant.
	\subsection{Leistung}
	Die Leistung ist die Änderung der Energie pro Zeit:
	\begin{align*}
		P(t) = \frac{dE}{dt}
	\end{align*}
	Somit hängt Leistung auch mit Arbeit zusammen:
	\begin{align*}
		P(t) = \frac{dW}{dt} = \frac{d}{dt} \int_{t_0}^{t_1} \vF(t) \cdot \vv(t) dt = \vF(t) \cdot \vv(t)
	\end{align*}
	\subsection{Impuls}
	Der \tbf{Impuls} eines Körpers ist das Produkt aus Masse und Geschwindigkeit eines Körpers:
	\begin{align*}
		\vp = m \cdot \vv
	\end{align*}
	Die Änderrungsrate des Impulses eines Körpers ist durch die Gesamtkraft gegeben, die auf den Körper wirkt:
	\begin{align*}
		\vF = m \cdot \va = m \cdot \frac{d\vv}{dt} = \frac{d\vp}{dt}
	\end{align*}
	Der Gesamtimpuls innerhalb eines Systems ist ebenfalls konstant. Die Änderungsrate des Gesamtimpulses ist also $\frac{d\vp_{ges}}{dt} = 0$.
	Durch die Impulserhaltung und das dritte Newtonsche Axiom folgt auch, dass die Summe der Kräfte innerhalb eines Systems $0$ sein muss:
	\begin{align*}
		\vF_{ges} = \sum_{i = 1}^n F_i = \sum_{i = 1}^n \frac{d\vp}{dt} = \frac{d\vp_{ges}}{dt} = 0
	\end{align*}
	\subsection{Drehmoment}
	Ein \tbf{Drehmoment} (\tbf{Torque}) ist eine für Hebelsysteme relevante Kräfte, definiert als das Kreuzprodukt eines "Hebelvektors" $\vr$ mit einer Kraft $\vF$:
	\begin{align*}
		\vT = \vF \times \vr
	\end{align*}
	Die Einheit des Drehmoments ist somit $N \cdot m$.
	\section{Umweltphysikalische Beispiele}
	\subsection{Auftriebskraft}
	Der Betrags der Auftriebskraft eines Körpers $K$ in einer Flüssigkeit ist gegeben durch:
	\begin{align*}
		F_{A} = (\rho_{Fluid} - \rho_K) \cdot g \cdot V
	\end{align*}
	Wobei $V$ der Teil des Volumens ist, welcher unter Wasser liegt.
	Im tatsächlichen Kraftvektor wird die Richtung der Kraft dann durch die Richtung des Vektors zwischen dem Schwertkraftmittelpunkt des Systems und dem Schwerpunkt des Körpers gegeben. Bei schwimmenden Körpern herrscht am Schwerpunkt ein Kräftegleichgewicht zwischen der Auftriebskraft und der Schwerkraft.
	\subsection{Staudruck und Luftwiderstand}
	Der \tbf{Staudruck} eines Körpers ist die Erhöhung des Drucks am Staupunkt eines umströmten Körpers gegenüber dem statischen Druck eines Fluids, also der Druck, mit dem das Fluid gegen den Körper druckt.
	\begin{align*}
		p = \frac{1}{2} \cdot \rho \cdot \norm{\vv_\bot}^2
	\end{align*}
	Die dazugehörige \tbf{Staukraft} ist dann gegeben durch:
	\begin{align*}
		F_{Stau} = p \cdot A = \frac{1}{2} \cdot \rho \cdot A \cdot \norm{\vv_\bot}^2
	\end{align*}
	Zur Berechnung des Luftwiderstands an einem Körper wird dann ein Faktor $c_w$ hinzugefügt, welcher beschreibt, wie stark die Luft den Körper umfließen kann.
	\begin{align*}
		F_{Luft} = c_w F_{Stau} = c_w \cdot \frac{1}{2} \cdot \rho \cdot A \cdot \norm{\vv_\bot}^2
	\end{align*}
	Die Relativgeschwindigkeit $v$ eines Flugzeugs gegenüber der Luft wird durch die Bordinstrumente durch Umstellung dieser Formel nach $\norm{\vv_\bot}$ berechnet!
	\begin{align*}
		\frac{1}{2} \cdot \rho_{Luft} \cdot v^2 &= \rho_{Mess} \cdot g \cdot \Delta h\\
		\implies v &= \sqrt{\frac{2 \cdot \rho_{Mess} \cdot g \cdot \Delta h}{\rho_{Luft}}}
	\end{align*}
	Dabei wird in einem Messrohr eine Messflüssigkeit mit Dichte $\rho_{Mess}$ durch den Staudruck um eine Höhe $\Delta h$ bewegt.
	\subsection{Das Kontinuitätsgesetz}
	Flüssigkeiten sind Näherungsweise inkompressibel, dementsprechend ist Volumen mal Flüssigkeit näherungsweise Konstant. Fließt also eine Flüssigkeit durch einen enger werdenden Behälter, wird sie schneller.
	\begin{align*}
		A(t) \cdot v(t) = const
	\end{align*}
	Luft ist deutlich weiter davon entfernt, inkompressibel zu sein, da aber trotzdem ein Widerstand gegen Kompression existiert kann man trotzdem qualitativ den selben Effekt beobachten.
	\subsection{Die Bernoulli-Gleichung}
	In fließendem Wasser gilt folgendes Energiegleichgewicht:
	\begin{align*}
		E_{kin} + E_{pot} = const + W
	\end{align*}
	$W$ ist hier die sogenannte "Volumenarbeit", die ein Teil des Fluids leistet, um vor ihm liegende Teile des Fluids auf höheren Druck zu komprimieren:
	\begin{align*}
		W = - F \cdot \Delta s = (p_2 - p_1) \cdot A \cdot \Delta s = (p_2 - p_1) \cdot \Delta V
	\end{align*}
	Wir erhalten letzendlich:
	\begin{align*}
		\frac{1}{2} \cdot \rho \cdot v^2 + \rho \cdot g \cdot h + p = const 
	\end{align*}
	Dies führt zum sogenannten "Hydrodynamischen Paradoxon" - an Orten mit höherer Flussgeschwindigkeit ist der Druck geringer. So funktionieren zum Beispiel sogenannte Wasserstrahlpumpen - wenn man durch ein Glasrohr, welches immer wieder dünner wird, Wasser laufen lässt, kann man durch ein seitliches Rohr einen Sog erzeugen. Dies ist außerdem ein Faktor dabei, warum starker Wind Dachziegel anheben kann.
	\subsection{Hangabtriebskraft}
	Wir betrachten die Bewegung eines Körpers mit Masse $m$ um einen Weg der Länge $\Delta s$ mit Höhenänderung $\Delta h$ entlang eines Höhengradienten $\nabla H$.
	Für die Änderung der potentiellen Energie gilt $\Delta E_{pot} = m \cdot g \cdot \Delta h$. Diese Energieänderung ist genau die Arbeit $W$, welche durch die Hangabtriebskraft am Körper verrichtet wird:
	\begin{align*}
		W = \vF_H \cdot \Delta \vs = - \Delta E_{pot} = - m \cdot g \cdot \Delta h
	\end{align*}
	Wir erhalten:
	\begin{align*}
		\vF_H = -m \cdot g \cdot \frac{\Delta h}{\Delta \vs} \to -m\cdot g \cdot \frac{dh}{d\vs} = -m \cdot g \cdot \grad H
	\end{align*}
	\subsection{Gradientenkraft}
	Durch den Druckgradienten der Atmosphäre entstehen Druckunterschied über beliebige Strecken $\Delta x$. Somit wirkt auf einen zum Gradienten orthogonalen Querschnitt $A$ eine Kraft von:
	\begin{align*}
		\Delta p &= \grad p \cdot \Delta \vx\\
		\Delta \vF &= -\Delta p \cdot A = - \grad p \cdot \vx \cdot A
	\end{align*}
	Aus dieser Druckgradientenkraft folgt eine Beschleunigung der Luft im betrachteten Volumen $\Delta x \cdot A$. Für die entstehende Beschleunigung gilt dann:
	\begin{align*}
		\va = \frac{\Delta F}{m} = \frac{\Delta F}{\rho \cdot V} = \frac{\grad p \cdot \Delta x \cdot A}{\rho \cdot \Delta x \cdot A} = -\frac{1}{\rho} \grad p
	\end{align*}
	\section{Schwerkraft}
	Die Schwerkraft zwischen zwei Massen $m_1$ und $m_2$ mit Abstand $r$ ist im Allgemeinen gegeben durch:
	\begin{align*}
		F_G = G \cdot \frac{m_1 \cdot m_2}{r^2}
	\end{align*}
	Ist $M$ die Erdmasse und $r$ der Erdradius, kann man so die Fallbeschleunigung $g$ auf einer Beliebigen Distanz $r$ vom Erdmittelpunkt berechnet werden:
	\begin{align*}
		m \cdot g &= G\frac{mM}{r^2}\\
		\implies  g &= \frac{GM}{r^2}
	\end{align*}
	Wir können die Schwerkraft als Zentripetalkraft betrachten und so die Zeit berechnen, welche ein Satellit braucht, um einen Zentralkörper zu umrunden, und die Geschwindigkeit, die dabei erhalten werden muss. 
	\newpar
	Aus dem Gleichgewicht der Zentripetalkraft mit der Schwerkraft erhalten wir die Gleichgewichtsgeschwindigkeit:
	\begin{align*}
		F_z = m \cdot \frac{v^2}{r} = F_G
		\implies v = \sqrt{\frac{GM}{r}}
	\end{align*}
	Wir können die Umlaufzeit nun bestimmen als:
	\begin{align*}
		v &= \sqrt{\frac{GM}{r}} = \omega r = \frac{2\pi}{T} r\\
		\implies T &= 2\pi \sqrt{\frac{r^3}{GM}}
	\end{align*}
	\subsection{Arbeit gegen die Schwerkraft}
	Angenommen, wir heben einen Körper mit Masse $m$ vom Abstand $R_0$ zur Erde auf den Abstand $R_1 = R_0 + h$ an. So gilt:
	\begin{align*}
		W &= \int_{R_0}^{R_1} F \cdot ds\\
		  &= \int_{R_0}^{R_1} G \frac{mM}{r^2} \cdot dR\\
		  &= GmM \left(\frac{1}{R_0} - \frac{1}{R_0 + h}\right)\\
		  &= GmM \left(\frac{1}{R_0} - \frac{1}{R_1}\right)
	\end{align*}
	\subsection{Barometrische Höhenformel}
	Wir wissen, dass auf der Erde genug Gas vorhanden ist, und dass die Schwerkraft ausreichend ist, um eine Atmosphäre zu bilden. Die Kenngröße der Schwerkraft auf der Erde ist die Fallbeschleunigung $g$, die Kenngrößen der Luft sind die Dichte $\rho$, die Molare Masse $M$, und Temperatur $T$.
	\newpar
	Wie ändert sich der Atmosphärische Luftdruck mit steigender Höhe?
	\newpar
	Ist $m$ die Masse an Luft, $n$ die Anzahl der Luftmoleküle und $M$ die durchschnittliche molare Masse der Moleküle, so ist der Druck gegeben durch:
	\begin{align*}
		\rho = \frac{m}{V} = \frac{M \cdot n}{V}
	\end{align*}	
	Für Luft gilt näherungsweise die Zustandsgleichung für ideale Gase $pV = nRT$, also:
	\begin{align*}
		\rho = \frac{1}{V} Mn = \frac{p}{nRT} Mn = \frac{M}{RT} p
	\end{align*}
	$R$ ist hier die sogenannte Gaskonstante $R \approx 8.314 \frac{\text{J}}{\text{K}\cdot \text{mol}}$.
	\newpar
	Für den Druckgradienten folgt:
	\begin{align*}
		dp &= - \frac{dF}{A} = -\frac{dm \cdot g}{A} = \frac{\rho \cdot g \cdot dV}{A} = -\rho \cdot g \cdot dh\\
		\implies \frac{dp}{dh} &= -\rho \cdot g = -\frac{Mg}{RT} \cdot p
	\end{align*}
	Unter der Annahme, dass die Atmosphäre isotherm ist, also in etwa konstante Temperatur hat, folgt
	\begin{align*}
		\frac{dp}{dh} &= -cp\\
		\implies p(h) &= p_0 \cdot e^{-ch} = p_0 \cdot e^{{\textstyle-\frac{Mg}{RT}}h}
	\end{align*}
	Der Druck nimmt also mit zunehmender Höhe exponentiell ab.
	\section{Rotierende Systeme}
	\subsection{Corioliskraft}
	Jede Bewegung auf der Oberfläche einer rotierenden Kugeln kann in drei Komponenten zerlegt werden:
	\begin{itemize}
		\item Axialkomponente: Parallel zur Drehachse.
		\item Tangentialkomponente: Bewegung um die Drehachse herum.
		\item Radialkomponente: Bewegung von der Drehachse weg.
	\end{itemize}
	Bei einer Bewegung in Radial- oder Tangentialrichtung ändert sich der Umfang des Kreises, auf dem ein Körpers um die Drehachse rotiert. Somit entspricht die vorherige Geschwindigkeit des Körpers nicht mehr der Umfangsgeschwindigkeit seines Umfelds, und es entsteht eine Scheinkraft, die sogenannte \tbf{Corioliskraft}. Radialbewegungen werden entgegen der Rotationsrichtung abgelenkt, es gilt:
	\begin{align*}
		\frac{1}{2}a_{Cor} (\Delta t)^2 &= \Delta s = \Delta r \cdot \Delta \phi = v \Delta t \cdot \omega \Delta t = v \cdot \omega \cdot (\Delta t)^2\\
		\implies a_{Cor} &= 2 \cdot \omega \cdot v\\
		\implies F_{Cor} &= 2 \cdot m \cdot \omega \cdot v
	\end{align*}
	Für Tangentialbewegung lässt sich die Bewegung zerlegen in die Umlaufgeschwindigkeit und eine zusätzliche Geschwindigkeit $v'$. Für die gesamte Zentripeltalbeschleunigung gilt:
	\begin{align*}
		a &= \frac{v^2}{R}\\
		  &= \frac{(v_{\omega} + v')^2}{R}\\
		  &= \frac{v_{\omega}^2}{R} + \frac{v'^2}{R} + 2\frac{v_\omega v'}{R}\\
		  &= \frac{v_{\omega}^2}{R} + \frac{v'^2}{R} + 2 \omega v'
	\end{align*}
	Der erste Term entspricht der Zentripetalbeschleunigung der rotierenden Kugel, der zweite entspricht der Zentripetalbeschleunigung der zusätzlichen Tangentialbewegung, und der dritte Term entspricht der Tangentialen Coriolisbeschleunigung. Ähnlich wie bei der Radialgeschwindigkeit gilt also:
	\begin{align*}
		a_{Cor} &= 2 \cdot \omega \cdot v'\\
		F_{Cor} &= 2 \cdot m \cdot \omega \cdot v'
	\end{align*}
	Die Corioliskraft wirkt immer senkrecht zur Bewegungsrichtung, bei tangentialer Bewegung Richtung Erdachse und bei radialer Bewegung in Tangentialrichtung. 
	Auf der Nordhalbkugel führt:
	\begin{itemize}
		\item Eine Bewegung nach Norden zu einer Beschleunigung nach Osten,
		\item eine Bewegung nach Süden zu einer Beschleunigung nach Westen,
		\item eine Bewegung nach Osten zu einer Beschleunigung nach Süden,
		\item und eine Bewegung nach Westen zu einer Beschleunigung nach Norden.
	\end{itemize}
	Eine wichtige Folge ist, dass Richtung Äquator blasende Winde entgegen der Erdrotation in Richtung Westen abgelenkt werden. So entstehen zum Beispiel die Passatwinde.
	\section{Statistische Physik und Reaktionskinetik}
	\subsection{Druck aus Molekülbewegung}
	\subsection{Freiheitsgrade}
	Ein Körper mit Volumen nicht null hat im Allgemeinen sechs Freiheitsgerade - drei in der Bewegung, und drei in der Rotation. Bei einzelnen Atomen gehen wir von einer Ausdehnung von $0$ aus, also von nur drei Freiheitsgraden. Ein Molekül mit $n$ Molekülen hat bei sehr hohen Temperaturen bis zu $3n$ Freiheitsgrade. Diese bestehen für lineare Moleküle aus $3$ Translationsfreiheitsgraden, $2$ Rotationsfreiheitsgraden, und $3n-5$ Vibrationsfreiheitsgraden. Bei nichtlinearen Molekülen ist ein zusätzlicher Rotationsfreiheitsgrade vorhanden, also ein Vibrationsfreiheitsgrad weniger.
	\begin{description}
		\item[Wichtige lineare Moleküle:] $O_2$, $N_2$, $CO_2$
		\item[Wichtige nichtlineare Moleküle:] $H_2O$, $CH_4$, etc.
	\end{description}
	Die Anzahl der aktiven Freiheitsgrade $f_U$ bestimmt die Gesamtenergie der Teilchen und damit des gesamten Systems. Sie ist somit auch für die Wärmekapazität relevant.
	\newpar
	Bei der Ermittlung der Gesamtenergie zählen wir Vibrationsfreiheitsgrade doppelt, da sie sowohl kinetische als auch potentielle Energie besitzen:
	\begin{align*}
		f_U = f_{trans} + f_{rot} + 2 f_{vib}
	\end{align*}
	\subsection{Thermodynamische Temperatur}
	\subsection{Zustandsgleichung des Atomgases}
	\begin{align*}
		p = \frac{1}{3} \cdot \frac{N}{V} \cdot m \cdot v^2
	\end{align*}
	\begin{align*}
		E_{kin} = f \cdot \lr(\frac{1}{2} \cdot k \cdot t) = \frac{3}{2} \cdot k \cdot t
	\end{align*}
	\begin{align*}
		\implies p &= \frac{N}{V} \cdot k \cdot T\\
		\implies p \cdot V &= N \cdot k \cdot T\\
						   &= \frac{N}{N_A} \cdot N_A \cdot k \cdot t\\ 
						   &= n \cdot R \cdot T
	\end{align*}
	Es folgt außerdem, dass die Gesamtenergie $U$ eines Atomgases mit $f$ effektiven Freiheitsgraden und $N$ Teilchen gegeben ist durch:
	\begin{align*}
		U = \frac{f}{2}NkT = \frac{f}{2} nRT
	\end{align*}
	Wir fassen einige Terme zu zwei neuen Größen zusammen - der \tbf{molaren Wärmekapazität}
	\begin{align*}
		c_V = \frac{f}{2}R,
	\end{align*}
	und der \tbf{spezifischen Wärmekapazität}
	\begin{align*}
		C_v = \frac{c_w}{M} = \frac{f}{2}\frac{R}{M}
	\end{align*}
	wobei $M$ die gesamte Masse des untersuchten Stoffes ist.
	\section{Gasgemische}
	Der Druck in einem Gasgemisch entspricht Näherungsweise der Summe der Drucke der einzelnen Gase:
	\begin{align*}
		p \approx \sum_{i \in I} p_i
	\end{align*}
	\subsection{Sättigungsdruck}
	Für jede Temperatur, gibt es einen maximalen \tit{Dampfdruck} $f$, den sogenannten \tit{Sättigungsdruck}, bevor ein Dampf Kondensiert. Dieser steigt nichtlinear mit der Temperatur stark an. Es gilt:
	\begin{align*}
		f &= \frac{m_{H_2 O, max}}{V}\\
		p_{H_2 O} &= \frac{n_{H_2 O} RT}{V} = \frac{m_{H_2 O}RT}{M_{H_2 O}V} = f\frac{R}{M_{H_2 O}}T
	\end{align*}
	Die Bestimmung der Konstante $m_{H_2 O}$-max ist nicht Teil des Vorlesungsstoffes.
	\subsection{Mittlere freie Weglänge}
	Die \tit{mittlere freie Weglänge $\lambda$} eines Gasgesmischs ist die mittlere Distanz, die sich ein Teilchen bewegen kann, bevor es mit einem andere Teilchen zusammenstößt. Dies lsst sich geometrisch herleiten als:
	\begin{align*}
		\lambda = \frac{1}{\displaystyle \sqrt{2}\pi \frac{N}{V}d^2}
	\end{align*}
	\subsection{Boltzmann-Verteilung}
	Die Wahrscheinlichkeit, ein Teilchen mit Energie $E$ anzutreffen, ist proportional zu:
	\begin{align*}
		p(E) \sim e^{-\frac{E}{kt}}
	\end{align*}
	\subsection{Maxwell-Verteilung}
	Die Wahrscheinlichkeitsverteilung der Geschwindigkeit $v$ ist:
	\begin{align*}
		p(v) = \lr(\frac{m}{2kt})^\frac{3}{2} \cdot 4\pi v^2 \cdot e^{\displaystyle -\frac{mv^2}{2kT}}
	\end{align*}
	Die wahrscheinlichste Geschwindigkeit ist dann:
	\begin{align*}
		v_p = \sqrt{\frac{2kT}{m}} = \sqrt{\frac{2RT}{M}},
	\end{align*}
	und daraus folgt für die wahrscheinlichste kinetische Energie:
	\begin{align*}
		E_{kin_p} = \frac{m}{2}v_p^2 = kT
	\end{align*}
	Die Maxwellverteilung ist asymmetrisch, also ist die mittlere Geschwindigkeit größer als die wahrscheinlichste Geschwindigkeit! Für Berechnung muss man die mittlere Geschwindigkeit
	\begin{align*}
		\bar E = \frac{3}{2} kT
	\end{align*}
	benutzen.
	\newpar
	Die Wahrscheinlichkeit, dass Teilchen mit mit einer Geschwindkeit $v \geq v_0$ unterwegs sind, ist identisch zu:
	\begin{align*}
		p(v \geq v_0) = p\lr(E \geq \frac{1}{2}mv_0) = \frac{2}{\sqrt{\pi}} \cdot \sqrt{\frac{E_0}{kT}} \cdot e^{\displaystyle -\frac{E_0}{kT}}
	\end{align*}
	\section{Arrhenius-Gleichung, Reaktionsgeschwindigkeit und RGT-Regel}
	Jede Chemische Reaktion hat eine Aktivierungsenergie $E_A$, die ein Teilchen haben muss, um zu reagieren. Die Reaktion läuft also schneller ab, je mehr Teilchen diese Aktivierungsenergie haben. Die Reaktionsgeschwindigkeit $c$ wird in $\frac{\textnormal{mol}}{\textnormal{s}}$ gemessen. Im einfachsten Fall gilt die \tit{Arrhenius-Gleichung}:
	\begin{align*}
		c = A \cdot e^{\displaystyle -\frac{E_A}{RT}},
	\end{align*}
	wobei $A$ eine "Materialkonstante" der Reaktion ist. In der Praxis ist $A$ jedoch leicht Temperaturabhängig und somit nur näherungsweise konstant.
	\newpar
	In der Praxis ist $E_A >> RT$. Die wichtigste Konsequenz ist, dass in solchen Fällen $c$ Näherungsweise exponentiell mit $T$ wächst - bei einer Temperaturerhöhung von $10$° ist die Reaktionsgeschwindigkeit etwas mehr als doppelt so hoch. Dies ist die sogenannte \tit{RGT-Regel}, welche zum Beispiel ein wichtiger Faktor dafür ist, dass Essen bei höheren Temperaturen schneller schlecht wird.
	
	\chapter{Thermodynamik}
	In der Physik unterscheiden wir zwischen \tit{Wärme} $Q$ (der "thermischen Energie" die während einem "Thermodynamischen Prozess" übertragen wird, gemessen in Joule) und \tit{Temperatur} $T$ (der gewohnten Temperatur, gemessen in Grad Kelvin). Wärme steigt mit Stoffmenge, Temperatur nicht.
	\newpar
	Die Temperaturänderung ist in der Regel proportional zur übertragenen Wärme:
	\begin{align*}
		F \cdot \Delta s = W = Q \sim m \cdot \Delta T
	\end{align*}
	\chapter{Physik der Erde}
	\section{Strahlung}
	
\end{document}