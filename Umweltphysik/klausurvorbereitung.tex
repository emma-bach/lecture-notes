\documentclass[8pt]{extarticle}
\usepackage{extsizes}

% custom margins
\usepackage[a4paper,margin=1cm]{geometry}
\usepackage{lipsum}

% emma's long list of custom macros and universally used packages
%AMS packages - Symbols, "Theorem" and "Proof" Environments
\usepackage{amsmath}
\usepackage{amssymb}
\usepackage{amsthm}
\usepackage{physics}
\renewcommand{\div}{\tn{div}~}

% Nicer Headers and Footers
\usepackage{fancyhdr}
\usepackage[yyyymmdd]{datetime}
\renewcommand{\dateseparator}{-}

% Table of Contents
\usepackage[titles]{tocloft}
\usepackage[titletoc]{appendix}

% Tikz and Graphics
\usepackage{tikz}
\usepackage{tikz-cd}
\usepackage{xcolor}
\usepackage[many]{tcolorbox}

% Nicer Underlining
\usepackage{contour}
\usepackage{ulem}

% Multiple Text Columns
\usepackage{multicol}

% FloatBarrier
\usepackage{placeins} 

% hyperref should be last apparently
\usepackage{hyperref}

%--- RENEWED COMMANDS ---

% Variant Greek Letters
\renewcommand\epsilon{\varepsilon}
\renewcommand\phi{\varphi}

% Nicer Table of Contents
\renewcommand\cftsecdotsep{\cftdot}
\renewcommand\cftsubsecdotsep{\cftdot}

%--- TEXT FORMATING ---

% nice underlining
\renewcommand{\ULdepth}{1.6pt}
\contourlength{0.8pt}
\newcommand{\ul}[1]{%
	\uline{\phantom{#1}}%
	\llap{\contour{white}{#1}}%
}

% shorthand
\newcommand{\tn}[1]{\textnormal{#1}}
\newcommand{\tbf}[1]{\textbf{#1}}
\newcommand{\tit}[1]{\textit{#1}}
\newcommand{\ttt}[1]{\texttt{#1}}

\newcommand{\mc}[1]{\mathcal{#1}}

\newcommand{\ol}[1]{\overline{{#1}}}

% Underlined, bold, non-cursive Theorem Name
\newcommand{\theoremname}[1]{\ul{\textnormal{\tbf{#1}}}}

% Starts a new paragraph without indentation
% and with an empty line between paragraphs
\newcommand*{\newpar}{\par\vspace{\baselineskip}\noindent}

% Make inf height match sup height
\renewcommand{\inf}{\mathop{\mathrm{inf}\vphantom{\mathrm{sup}}}}

% Make tildes more readable
\renewcommand{\tilde}{\widetilde}

%--- RELATION SYMBOLS AND OPERATORS ---
\newcommand{\surj}{\twoheadrightarrow}
\newcommand{\inj}{\hookrightarrow}
\newcommand{\iso}{\overset{\sim}{\rightarrow}}
\newcommand{\symdiff}{\vartriangle}
\newcommand{\trans}{\twoheadrightarrow}
\newcommand{\tensor}{\otimes}
\newcommand{\bigmid}{~\middle|~}

%--- STUFF IN FANCY BRACKETS ---
\newcommand{\scalar}[2]{\left\langle #1, #2 \right\rangle}
\newcommand{\angles}[1]{\left\langle #1 \right\rangle}
\newcommand{\lr}{\qty}


%--- LETTERS ---
\newcommand{\dmu}{\ d\mu}
\newcommand{\dist}{\textnormal{dist}}
\newcommand{\spt}{\textnormal{spt}}

% mathbb
\newcommand{\bC}{\mathbb{C}}
\newcommand{\bF}{\mathbb{F}}
\newcommand{\bN}{\mathbb{N}}
\newcommand{\bQ}{\mathbb{Q}}
\newcommand{\bR}{\mathbb{R}}
\newcommand{\bZ}{\mathbb{Z}}

% mathcal
\newcommand{\cA}{\mathcal{A}}
\newcommand{\cB}{\mathcal{B}}
\newcommand{\cC}{\mathcal{C}}
\newcommand{\cD}{\mathcal{D}}
\newcommand{\cE}{\mathcal{E}}
\newcommand{\cF}{\mathcal{F}}
\newcommand{\cH}{\mathcal{H}}
\newcommand{\cI}{\mathcal{I}}
\newcommand{\cL}{\mathcal{L}}
\newcommand{\cM}{\mathcal{M}}
\newcommand{\cN}{\mathcal{N}}
\newcommand{\cO}{\mathcal{O}}
\newcommand{\cP}{\mathcal{P}}
\newcommand{\cQ}{\mathcal{Q}}
\newcommand{\cR}{\mathcal{R}}
\newcommand{\cS}{\mathcal{S}}
\newcommand{\cT}{\mathcal{T}}
\newcommand{\cW}{\mathcal{W}}
\newcommand{\cX}{\mathcal{X}}
\newcommand{\cZ}{\mathcal{Z}}

% mathfrac
\newcommand{\fa}{\mathfrak{a}}

% real numbers in fancy costumes
\newcommand{\barR}{\ol{\mathbb{R}}}
\newcommand{\bRnn}{\mathbb{R}^{n \times n}}
\newcommand{\bRmn}{\mathbb{R}^{m \times n}}
\newcommand{\bRnm}{\mathbb{R}^{n \times m}}

% vectors
\renewcommand{\va}{\vec{a}}
\renewcommand{\vb}{\vec{b}}
\newcommand{\vc}{\vec{c}}
\newcommand{\ve}{\vec{e}}
\newcommand{\vF}{\vec{F}}
\newcommand{\vh}{\vec{h}}
\newcommand{\vp}{\vec{p}}
\newcommand{\vr}{\vec{r}}
\newcommand{\vs}{\vec{s}}
\newcommand{\vT}{\vec{T}}
\renewcommand{\vu}{\vec{u}}
\newcommand{\vv}{\vec{v}}
\newcommand{\vw}{\vec{w}}
\newcommand{\vW}{\vec{W}}
\newcommand{\vx}{\vec{x}}
\newcommand{\vy}{\vec{y}}
\newcommand{\vz}{\vec{z}}
\newcommand{\vzero}{\vec{0}}

\newcommand{\veta}{\vec{\eta}}

\renewcommand{\grad}{\vec{\nabla}}

% bold letters
\newcommand{\tbA}{\mathbf{A}}
\newcommand{\tbB}{\mathbf{B}}
\newcommand{\tbC}{\mathbf{C}}
\newcommand{\tbD}{\mathbf{D}}
\newcommand{\tbE}{\mathbf{E}}
\newcommand{\tbY}{\mathbf{Y}}
\newcommand{\tbZ}{\mathbf{Z}}

% sequences
\newcommand{\an}{(a_n)_{n \in \bN}}
\newcommand{\bn}{(b_n)_{n \in \bN}}
\newcommand{\sn}{(s_n)_{n \in \bN}}
\newcommand{\iinI}{_{i \in I}}
\newcommand{\iinN}{_{i \in \bN}}

% special functions
\newcommand{\at}{\textnormal{at}}
\newcommand{\ggT}{\textnormal{ggT}}
\newcommand{\kgV}{\textnormal{kgV}}
\newcommand{\id}{\textnormal{id}}
\newcommand{\Id}{\textnormal{Id}}
\newcommand{\im}{\textnormal{im}}
\newcommand{\inv}{\textnormal{inv}}
\newcommand{\ord}{\textnormal{ord}\ }
\newcommand{\rang}{\textnormal{rang}\ }
\renewcommand{\tr}{\textnormal{tr}\ }
\newcommand{\vol}{\textnormal{vol}}
\newcommand{\cond}{\textnormal{cond}}
\newcommand{\sgn}{\textnormal{sgn}}
\renewcommand{\char}{\textnormal{char}}
\newcommand{\Frac}{\textnormal{Frac}}
\newcommand{\Irr}{\textnormal{Irr}}

% groups and sets
\newcommand{\GL}{\text{GL}}
\newcommand{\SO}{\text{SO}}
\newcommand{\Hess}[1]{\text{Hess}(#1)}

\newcommand{\Grp}{\textnormal{Grp}}
\newcommand{\Mag}{\textnormal{Mag}}
\newcommand{\Mon}{\textnormal{Mon}}
\newcommand{\Ens}{\textnormal{Ens}}
\newcommand{\Hom}{\textnormal{Hom}}
\newcommand{\Aut}{\textnormal{Aut}}

\newcommand{\Mat}{\textnormal{Mat}}
\newcommand{\Matnn}{\textnormal{Mat}(n \times n)}
\newcommand{\MatBB}[3]{\textnormal{Mat}^{#1}_{#2}\left(#3\right)}



% --- THEOREM AND PROOF TYPES ---
% \newtheorem{codename}{printedname}[countedwith]
\newtheorem{lemma}{Lemma}[section]
\newtheorem{theorem}[lemma]{Satz}
\newtheorem{proposition}[lemma]{Proposition}
\newtheorem{corollary}[lemma]{Korollar}

\theoremstyle{definition}
\newtheorem{definition}[lemma]{Definition}
\newtheorem{summary}[lemma]{Zusammenfassung}
\newtheorem{example}[lemma]{Beispiel}
\newtheorem{counterexample}[lemma]{Gegenbeispiel}
\newtheorem{beobachtung}[lemma]{Beobachtung}
\newtheorem{anmerkung}[lemma]{Anmerkung}
\newtheorem{question}[lemma]{Frage}
\newtheorem{application}[lemma]{Anwendung}
\newtheorem{reminder}[lemma]{Erinnerung}
\newtheorem{konsequenz}[lemma]{Konsequenz}

% --- CUSTOM ENVIRONMENTS ---

% sketchy proofs marked with untrustworthy QED symbol
\newenvironment{proofsketch}{\begin{proof}[Beweisskizze]\renewcommand*{\qedsymbol}{\("\square"\)}}{\end{proof}}

% align with exactly one number for labeling
\NewDocumentEnvironment{nalign}{}{\equation\aligned}{\endaligned\endequation}

\makeatletter
\let\amsmath@bigm\bigm

\usepackage{titlesec}
\titleformat{\section}[display]
{\centering \normalsize \Large  \bfseries \color{black}}{}{}{}
\titleformat{\subsection}[display]
{\centering \normalsize \large  \bfseries \color{black}}{}{}{}

\newcommand{\N}{\tn{N}}
\newcommand{\Pa}{\tn{Pa}}
\newcommand{\kg}{\tn{kg}}
\newcommand{\m}{\tn{m}}
\newcommand{\s}{\tn{s}}
\newcommand{\J}{\tn{J}}
\newcommand{\pot}{\tn{pot}}
\newcommand{\kin}{\tn{kin}}
\newcommand{\ges}{\tn{ges}}
\usepackage{mhchem}
\newcommand{\wawer}{\ce{H2O}}
\usepackage{enumitem}
\setlist[itemize]{noitemsep} 

\begin{document}
\twocolumn[\section*{Klausurblatt Umweltphysik, Emma Marie Bach :3}]
\subsection*{Einheiten}
\vspace{-10pt}
\begin{multicols}{2}
\noindent
\tbf{Kraft:}\\ $[F] = 1\tn{N} = 1 \frac{\tn{kg} \cdot \tn{m}}{s^2}$\\
\tbf{Druck:}\\ $[p] = 1\tn{Pa} = 1 \frac{\N}{\m^2} = 1 \frac{\kg}{\m \cdot \s^2}$\\
$1 \tn{bar} = 100.000 \Pa = 1000\tn{h}\Pa$\\
\tbf{Arbeit, Energie:}\\ $[W] = [E] = 1 \J = 1 \N \cdot \m$\\
\tbf{Leistung:}\\ $[P] = W = 1 \frac{\J}{\s} = 1 \frac{\N \cdot \m}{\s}$\\
\tbf{Wärmekapazität:}\\
$[c] = 1 \frac{J}{K} = 1 \frac{\N \cdot \m}{K}$\\
\tbf{Teilchenzahl $N$},
\tbf{Stoffmenge} $n = \frac{N}{N_A}$ in Mol!\\
\tbf{Masse} $m$, \tbf{Molare Masse} $M = \frac{m}{n}$\\
\end{multicols}
\subsection*{Mathe}
\tbf{Taylorreihe:} $f(x) \approx \sum_{k = 1}^n f^{(k)} \cdot (x-a)^k$
\subsection*{Rotierende Systeme}
\vspace{-10pt}
\begin{multicols}{2}
	\centering
	\tbf{Zentripetalbeschleunigung:}\\ 
	$\frac{\Delta v}{v} = \frac{\Delta s}{r} \implies a_Z = \frac{v^2}{r}$\\
	\tbf{Zentripetalkraft:}\\
	$F_Z = m \cdot a_Z = m \cdot \frac{v^2}{r}$\\
	\tbf{Umlaufgeschwindigkeit:}\\
	$F_Z = F_G \implies v = \sqrt{\frac{\gamma M}{r}}$\\
	\tbf{Umlaufszeit:}\\
	$v = \omega \cdot r \implies t = 2\pi r \cdot \sqrt{\frac{r}{\gamma M}}$\\
\end{multicols}
\iffalse
\fi
\subsection*{Kräfte und Wege}
\vspace{-10pt}
\begin{multicols}{2}
\centering
\tbf{Grundlagen:}\\
$\vec F = m \cdot \vec a$\\
$\vec F_{a \to b} = - \vec F_{b \to A}$\\
\tbf{Kreisbahn:}\\ 
$s = \phi \cdot R = \omega \cdot R \cdot t$\\
\tbf{Schwerkraft auf der Erde:}\\
$F_G = m \cdot g$\\
\tbf{Schwerkraft allgemein:}\\
$F_G = \gamma \cdot \frac{m \cdot M}{r^2}$\\
\tbf{Auftriebskraft ($V$ = Volumen unter Wasser):}\\
$F_{A} = F_{G_\tn{Fluid}} - F_{G_K} = (\rho_{\tn{Fluid}} - \rho_K) \cdot g \cdot V$
\tbf{Hangabtriebskraft:}\\
$F_H = F_G \cdot \sin(\alpha) = F_G \cdot \norm{\grad H}_2$\\
\tbf{Normalenkraft:}\\
$F_N = F_G \cdot \cos(\alpha)$\\
\tbf{Reibungskraft:}\\
$F_R = \mu \cdot F_N$\\
\tbf{Druck:}\\
$p = \frac{F}{A} = \frac{1}{2} \cdot \rho \cdot v^2$\\
\tbf{Impuls:}\\
$\vec p = m \cdot \vec v \implies \vec F = \dot{\vec p}$\\
$\dot{\vec p}_\ges = \vec F_\ges = 0$\\
\tbf{Drehmoment (Torque) mit Hebelarm $r$:}\\
$T = F \cdot r$\\
$T_1 = F_1 \cdot r_1 = F_2 \cdot r_2 = T_2$
\end{multicols}

\subsection*{Energie, Arbeit, Leistung}
\vspace{-10pt}
\begin{multicols}{2}
\centering
\tbf{Arbeit:}\\
Für $F$ und $\Delta s$ parallel:\\
$W = F \cdot \Delta s$\\
\tbf{Arbeit gegen die Schwerkraft:}\\
$W = F_G \cdot \Delta s = m \cdot g \cdot \Delta h$\\
$= \gamma \cdot m \cdot M \cdot (\frac{1}{R_0} - \frac{1}{R_0 + h})$
\tbf{Potentielle Energie:}\\
$E_{\pot} = m \cdot g \cdot h$\\
\tbf{Arbeit gegen die Reibungskraft:}\\
$W = F_R \cdot \Delta s = \mu \cdot m \cdot g \cdot \Delta s$\\
\tbf{Arbeit gegen die Zentripetalkraft:}\\ 
$W = 0$, da $F \perp \Delta s$\\
\tbf{Kinetische Energie:}
$E_{\kin} = W = F \cdot \Delta s = m \cdot a \cdot \Delta s = m \cdot a \cdot \frac{1}{2} \cdot a \cdot (\Delta t)^2 = \frac{1}{2} \cdot m \cdot v^2$
\tbf{Gespannte Feder:}\\
$F = k \cdot \Delta x$\\
$E = \frac{1}{2} \cdot k \cdot s^2$\\
\tbf{Mechanische Energieerhaltung:}\\
$E_\ges = E_\pot + E_\kin + Q = \tn{const}$
inkl. Reibungswärme $Q$\\
\tbf{Leistung:}\\
$P = \frac{W}{\Delta t} = \frac{\Delta E}{\Delta t} \to \frac{dW}{dt}$\\
\tbf{Leistung bei Kraftauswirkung:}\\
$P = \frac{W}{\Delta t} = \frac{F \cdot \Delta s}{\Delta t} = F \cdot v$\\
\end{multicols}
\subsection*{Strömungsdynamik}
\vspace{-10pt}
\begin{multicols}{2}
	\centering
	\tbf{Staudruck:}\\ 
	$p =\frac{1}{2} \cdot \rho \cdot v^2$\\
	\tbf{Staukraft (ohne Umströmung):}\\
	$F = p \cdot A = \frac{1}{2} \cdot \rho \cdot A \cdot v^2$
	\tbf{Luftwiderstandskraft:}\\
	$F_L = \frac{1}{2} \cdot c_w \cdot \rho \cdot A \cdot v^2$\\
	\tbf{Kontinuitätsgesetz für Inkompressible Fluide:}\\
	$A \cdot v = \tn{const}$\\
	\tbf{Volumenarbeit:}\\
	$W_V = - \int F ~ds
	= - \int p \cdot A ~ds
	= - p \cdot \Delta V$\\
	Vorzeichen zu Gewählt, das für die Kompression positive Arbeit nötig ist.\\
	\tbf{Hydrodynamische Energieerhaltung:}\\
	$E_\ges = E_\pot + E_\kin + W_V = \tn{const}$\\
	\tbf{$\implies$ Bernoulli-Gleichung durch Division der Volumen:}\\
	$\frac{1}{2} \cdot \rho \cdot v^2 + \rho \cdot g \cdot h + p = \tn{const}$\\
	\tbf{Konsequenz}: An Orten mit hoher Strömungsgeschwindigkeit ist der Druck geringer (Hydrodynamisches Paradoxon)
	\tbf{Gradientenkraft:}\\
	$\Delta p = \grad p \cdot \Delta x$\\ 
	$\implies \Delta F = -\Delta p \cdot A = - \grad p \cdot \Delta x \cdot A$\\
	$\implies a = \frac{\Delta F}{m} = - \frac{\grad p}{\rho}$\\
\centering
\end{multicols}
\subsection*{Barometrische Höhenformel}
%\vspace{-10pt}
%\begin{multicols}{2}
\tbf{Molare Masse $M$:} $M \cdot n = m$\\
Aus der idealen Gasgleichung folgt:
$\frac{dp}{dh} = -\rho \cdot g = -\frac{Mg}{RT} \cdot p$\\
\tbf{Barometrische Höhenformel für isotherme Atmosphäre:}\\
$p(h) = p_0 \cdot e^{-\frac{Mg}{RT} \cdot h}$
%\centering
%\end{multicols}
\subsection*{Partialdruck, Dampfdruck}
\vspace{-10pt}
\begin{multicols}{2}
	\centering
	\tbf{Partialdruck $p$ / Dampfdruck $p_{\tn H_2 \tn O}$}: Experimentell bestimmt. Maximaler Dampfdruck (Sättigungsdampfdruck) hängt von der Temperatur ab.
	\tbf{Absolute Luftfeuchtigkeit:}\\
	$f = \frac{m_{\ce{H2O}}}{V}$ $f = \frac{m_{\tn{H}_2\tn O, \tn{max}}}{V}$\\
	\tbf{Relative Luftfeuchtigkeit:}\\
	$\phi = \frac{f}{f_{\tn{max}}}$\\
	\tbf{Zusammenhang Dampfdruck/Luftfeuchtigkeit:}\\
	$p_{\ce{H2O}}  = \frac{n_{\wawer}RT}{V}$\\ 
	$= \frac{m_{\wawer}RT}{M_{\wawer}V} = f \cdot \frac{RT}{M_{\wawer}}$\\
\end{multicols}
\subsection*{Wind}
\vspace{-10pt}
\begin{multicols}{2}
	\begin{itemize}
		\item Druckgradientenkraft
		\item Corioliskraft
		\item Zentrifugalkraft
		\item Reibungskraft
	\end{itemize}
\end{multicols}
\vspace{-10pt}
\noindent
\tbf{Geostrophischer Wind:} Annahme: Nord-/Südkomponente der Corioliskraft und Druckgradienten heben sich auf, Reibung wird vernachlässigt. Parallel zu den Isobaren. Erklärt die Passatwinde und erklärt, warum Wind nicht direkt von Hoch nach Tief weht.\\
$F_D = -F_C \Leftrightarrow - \frac{1}{\rho} \frac{\Delta p}{\Delta x} = f_c \cdot v$\\
\tbf{Zyklostrophischer Wind:} Annahme: Nord-/Südkomponente der Druckgradientenkraft und der Zentrifugalkraft heben sich auf. $F_C$ und $F_R$ werden nicht berücksichtigt. Erklärt Tornados.\\
$F_D = F_Z \Leftrightarrow - \frac{1}{\rho} \frac{\Delta p}{\Delta x} = \frac{v^2}{R}$\\
\tbf{Gradientenwind / Geostropisch-Zyklostrophischer Wind:}\\ 
Berücksichtigt $F_D$, $F_C$ und $F_Z$. Reibung wird weiterhin vernachlässigt. Bestes Windmodell, welches trotzdem noch relativ genau rein durch Wetterkarten und Höhenwindmessungen vorhergesagt werden kann.\\
\tbf{Zyklonaler Gradientenwind:}\\
Die Luft dreht sich um ein Tiefdruckgebiet, es gilt:\\
$F_C + F_Z = F_D \Leftrightarrow f_c \cdot v + \frac{v^2}{R} = -\frac{1}{\rho} \frac{\Delta p}{\Delta x}$\\
\tbf{Antizyklonaler Gradientenwind:}\\
Die Luft dreht sich um ein Hochdruckgebiet, die Richtungen von $F_C$ und $F_D$ sind umgekehrt:\\
$F_C - F_Z = F_D \Leftrightarrow f_c \cdot v - \frac{v^2}{R} = -\frac{1}{\rho} \frac{\Delta p}{\Delta x}$\\
\subsection*{Wärme, Temperatur, Freiheitsgrade}
\vspace{-10pt}
\begin{multicols}{2}
	\centering
	\tbf{Stoßrate gegen eine Quaderförmige Wand bei Teilchendichte $D$:}\\
	$\dot N = \frac{D}{6} \cdot v$\\
	\tbf{Druck durch Impulsübertragung:}\\
	$p = \frac{F}{A} = \frac{\norm{\dot {\vec p}}}{A} = m \cdot \frac{\norm{\dot v}}{A} = \frac{1}{3} \frac{N}{V} mv^2$\\
	\tbf{Freiheitsgrade:}\\
	Im Allgemeinen $f = 3n$ für $n$-atomiges Molekül\\
	\tbf{Effektive Freiheitsgrade} (bei realistischen Temperaturen):
	\vspace{-10pt}
	\begin{multicols}{2}
		\begin{itemize}
			\item  He: $3$
			\item $\tn{N}_2, \tn{O}_2$: $5$
			\item $\tn{CO}_2$: $\sim 7$\\
			\item $\tn{H}_2\tn{O}$: $\sim 7$\\ 
		\end{itemize}
	\end{multicols}
	\vspace{-10pt}
	\tbf{Energie pro Teilchen pro Freiheitsgrad:}\\
	$E = \frac{1}{2}kT$\\
	\tbf{Ideale Gasgleichung:}\\
	$pV = nRT = N k T$, mit $n = \frac{N}{N_A}$\\
	\tbf{Energie für $N$ Teilchen mit jeweils $f$ Freiheitsgraden:}\\
	$U = f \cdot N \cdot E = \frac{f}{2} NkT = \frac{f}{2} nRT$\\
	\tbf{Mittlere freie Weglänge:}\\
	$\lambda = \frac{1}{\sqrt 2 \pi \frac{N}{V}d^2}$\\
	\tbf{Boltzmann-Verteilung:}\\
	$p(E) \sim e^{- \frac{E}{kT}}$, wahrscheinlichste Energie
	$E_{\kin_p} = kT$, aber mittlere Energie $E_{\kin_m} = \frac{3}{2}kT$!
	\tbf{Maxwell-Verteilung:}\\
	$p(v) \sim 4\pi v^2 \cdot e^{-\frac{mv^2}{2kT}}$, wahrscheinlichste Geschwindigkeit $v_p = \sqrt{\frac{2kT}{m}} = \sqrt{\frac{RkT}{M}}$
	\tbf{Arrheniusgleichung für Reaktionsgeschwindigkeit $c$:}\\
	$c = A \cdot e^{-\frac{E_A}{RT}}$
\end{multicols}
\subsection*{Thermodynamik}
\begin{multicols}{2}
	\centering
	\tbf{Abgeschlossenes (isoliertes) System:} Weder Energie-, noch Materialaustausch.\\
	\tbf{Geschlossenes System:} Energieaustausch, aber kein Materialaustausch.\\
	\tbf{Offenes System:} Energie- und Materialaustausch.\\
	\tbf{Adiabatisches System:} Kein Wärmeaustausch.\\
	\tbf{Arbeitsdichtessystem:} Kein Arbeitsaustausch.\\
	\tbf{Molare Wärmekapazität $[c_V] = 1 \frac{\J}{\tn{mol} \cdot \tn K}$ bei konstantem Volumen}:\\
	$c_V = \frac{f}{2} \cdot R$\\
	\tbf{Spezifische Wärmekapazität $[C_V] = 1 \frac{\J}{\kg \cdot \tn K}$ bei konstantem\\ Volumen}:\\
	$C_V = \frac{c_V}{M}$\\
	\tbf{Wärme / Thermische Energie:}\\
	$Q = m \cdot C \cdot \Delta T$\\
	\tbf{Wärmeströmung:}\\
	$\dot Q = \dot m \cdot C \cdot \Delta T$,
	$\dot m = \rho \cdot A \cdot v$
	\tbf{Erster Hauptsatz der Thermodynamik:}
	$\Delta U := \Delta E = \Delta Q + \Delta W$
\end{multicols}
\subsection*{Zustandsänderungen}
Für Phasenübergang ist Energie nötig! $H = C \cdot \Delta T$\\
\tbf{Isotherme Zustandsänderung:}\\
Keine Änderung der \tbf{Temperatur} - bei Expansion ist aber mehr Wärme nötig um die gleiche Temperatur beizubehalten.\\
Langsame Kompression von Luft mit gleichzeitiger Abkühlung,\\
 näherungsweise Beschreibung technischer Prozesse.\\
$\Delta T = 0 \implies \Delta U = 0 \implies \Delta Q = - \Delta W$\\
Für die Volumenarbeit folgt für ideale Gase $W = nRT \cdot \ln\lr(\frac{V_1}{V_2})$\\
\tbf{Isochore Zustandsänderung:}\\
Konstantes Volumen. Es gilt $\Delta W = 0$, also $\Delta U = \Delta Q$\\
\tbf{Isobare Zustandsänderung:}\\
Konstanter Druck. Chemische Reaktionen, Ausdehung von Luft und Wasser bei Erwärmung.
Ein Teil der Zugefügten Energie geht in Ausdehnung über, also muss für die Erwärmung mehr
Energie hinzugefügt werden als unter isochoren Verhältnissen.\\
\tbf{Wärmekapazität bei konstantem Druck:}\\
$c_P = c_V + R$, $C_P = C_V + \frac{R}{M}$\\
\tbf{Adiabatische Zustandsänderung:}\\
Keine Änderung der \tbf{Wärme}. $\Delta Q = 0$, also $\Delta U = \Delta W$ Beschreibt sehr gut isolierte Systeme, Verbrennungsmotoren, Ausbreitung von Schall, aufsteigende Luftmassen.\\
\tbf{Adiabatengleichung:} $T \cdot V^\frac{R}{c_V} = \tn{const}$, $p \cdot V^\frac{c_p}{c_v} = \tn{const}$\\
Es folgt $\frac{dp}{dT} = \frac{c_p p}{RT}$\\
\tbf{Aufstieg Trockener Luft:}\\
Durch Gleichsetzen von $dp$ in der Adiabatengleichung\\ und der Barometrischen Höhenformel folgt\\ 
$\frac{dT}{dh} = - \frac{M \cdot g}{c_p} = - \frac{g}{C_p}$\\
\tbf{Aufstieg feuchter Luft:}\\
Durch Adiabatische Abkühlung kann beim Aufstieg der Taupunkt erreicht werden, wodurch Wolken entstehen! Dabei wird Kondensationswärme frei und der Temperaturgradient $\frac{dT}{dh}$ wird geringer, desto feuchter die Luft war!
\subsection*{Entropie}
\tbf{Entropie:} $S = k \cdot \ln(W(n))$, wobei $W(n)$ die Gesamtzahl der\\ Mikrozustände der $n$ Teilchen ist. $\Delta S \geq \frac{\Delta Q}{T}$!!\\
\tbf{Zweiter Hauptsatz der Thermodynamik:} Wärme kann nicht von niedrigerer Temperatur zu höherer Tempatur fließen. Es gibt keine periodisch arbeitende Maschine, die Wärme perfekt in Energie umwandelt. Ausgleichs- und Mischungsvorgänge sind irreversibel. Entropie nimmt durch jeden spontan ablaufenden Prozess zu, ebenso durch Zufuhr von Wärme oder Materie. Entropie kann innerhalb eines Systems nur abnehmen, wenn Wärme oder Materie abgegeben wird.\\
\tbf{Wirkungsgrad einer reversiblen Wärmemaschine:} Angenommen, eine Maschine entnimmt Wärme von einer Stelle $A$ und überträgt sie auf eine andere Stelle $B$. Dann gilt $\Delta S_A = \frac{Q_A}{T_A}$, $\Delta S_B = \frac{Q_B}{T_B}$, und da die Entropie nicht abnimmt gilt $\Delta S_A = \Delta S_B$, also $\frac{Q_A}{T_A} = \frac{Q_B}{T_B}$. Da ein Teil der Wärme $Q_A$ zu Arbeit $W$ wird, und der Rest zu Wärme $Q_B$, gilt $Q_B = Q_A + W$. Es folgt $\eta := \frac{W}{Q_B} = 1 - \frac{T_A}{T_B} < 1$.\\
\tbf{Entropieänderung bei der freien Expansion ins Vakuum:}\\ 
$\Delta S = nR \cdot \ln\lr(\frac{V_2}{V_1})$
\clearpage
\subsection*{Enthalpie}
\tbf{Enthalpie} $H = U + p \cdot V$
Viele Prozesse in Umwelt und Technik sind isobar, dann gilt: $\Delta U = \Delta Q + \Delta W = \Delta Q - p\Delta V \implies \Delta Q = \Delta H$\\
Verdampfungsenthalpie $\Delta H_V$, Kondensationsenthalpie $\Delta H_K = - \Delta H_V$
\tbf{Freie Enthalpie / Gibbs-Energie:} $G = U + pV - TS$, jedes System verringert von alleine diese Größe, bis es nicht weiter geht.\\
\tbf{Phasenübergänge:} $\Delta G_1 = \Delta G_2 \implies \frac{\Delta p}{\Delta T} = \frac{S_2 - S_1}{V_2 - V_1}$\\
$\implies \frac{dp_s}{dT} = \frac{h_V}{\Delta V_m \cdot T} \approx \frac{h_v \cdot p_s}{R \cdot T^2}$, $\frac{dp_s}{p_s} \approx -\frac{h_V}{R} \cdot (\frac{1}{T_2} - \frac{1}{T_2})$\\
Folge: Wasserkapazität der Atmosphäre steigt pro Kelvin Erwärmung um ca. $7\%$
\subsection*{Strahlung}
Entstehung durch spontane Emmision, Bremsung von Ladungsträgern, Molekülschwingungen, zeitlich veränderlicher Strom, Paarvernichtung
\vspace{-10pt}
\begin{multicols}{2}
	$c = \lambda f$\\
	\tbf{Rydbergformel:}\\ 
	$\frac{1}{\lambda_{\tn{vac}}} = R \cdot \lr(\frac{1}{n_1^2} - \frac{1}{n_2^2})$\\
	\tbf{Energie im Photon:} $E = h \cdot f$\\
	\tbf{Lambertstrahler:} Gleich hell aus allen Richtungen, es gilt $I = I_{\tn{max}} \cdot \cos(\theta)$\\
	\tbf{Schwarzkörperstrahlung:}\\ 
	$P = \sigma \cdot A \cdot T^4$, Sonne ist in etwa ein Schwarzkörper\\
	\tbf{Abgestrahlte Energie pro Sekunde/Intervall/Winkel:}\\
	$L(f,T) = \frac{2hf^3}{c^2}\frac{1}{e^{hf/kT}-1}$\\
	\tbf{Atmosphärisches Fenster:} Bezeichnet den Wellenlängen- bereich, für den die Atmosphäre größtenteils durchlässig ist. Entsteht durch die Gaskomposition der Atmosphäre, insbesondere Wasserdampf und \ce{CO2}. Sichtbares Licht, knapp drüber, und kurzwellige Radiowellen.\\
	\tbf{Strahlungsbilanz:}\\ 
	Netto-Absorption $\to$ Erwärmung!\\
\end{multicols}
\subsection*{Treibhauspotential}
\tbf{Notation für Bestände und Flüsse:} Bestand $S$, Input $I$, Output $O$, $I = O + \Delta S$\\
Tiefsee und Tiefseeboden sind mit Abstand die größten \ce{CO2}-Speicher\\
\tbf{Anorganischer Kohlenstoffzyklus:}\\
\ce{CO2} (Luft) $\leftrightarrow$ \ce{HCO3-} (Wasser) $\leftrightarrow$ \ce{CaSiO3}, \ce{CaCO3} (Gestein)\\
\tbf{Organischer Kohlenstoffzyklus:}\\
\ce{CO2}, \ce{CH4} (Luft) $\leftrightarrow$ \ce{CH4} etc. (Organismen) $\leftrightarrow$ Kohle, Erdöl, etc. (Böden)\\
\tbf{Zerfall von Methan:} \ce{CH4} + 2\ce{O2} $\to$ \ce{CO2} + 2\ce{H2O}\\
\tbf{Radiative Forcing $F$:}\\
Änderung der Energiebilanz der Erde durch Änderung der Wirkung von Weltraumstrahlung, Gemessen in $\tn{W}/\m^2$. Eines der wichtigsten quantitativen Maße des Klimawandels.\\
$\Delta F \approx \alpha \cdot \ln \lr(\frac{C_0 + \Delta C}{C_0})$ mit $\alpha \approx 5,35 \frac{\tn{W}}{\m^2}$ bei Erhöhung des \ce{CO2}-Volumentanteils von $C_0$ um $\Delta C$.\\
\tbf{Spezifischer Strahlenantrieb:} $a_{\ce{CO2}} = \frac{\Delta F}{\Delta C} \approx \frac{\alpha}{C_0}$\\
\tbf{Treibhauspotential:} Spezifischer Strahlenantrieb mal Korrekturfaktor für weitere Zerfallsauswirkung mal Integral über die Konzentration eines Gases pro Zeit, zum Normieren durch das Treibhauspotential von \ce{CO2} geteilt.\\
\tbf{Korrekturfaktor}: $c_{\ce{CH4}} = (1 + f_1 + f_2)$, $f_1 = 0.5$ entspricht dem Abbau von Ozon durch die Reaktion \ce{CH4 + OH- -> CH3- + H2O}, $f_2 = 0.15$ entspricht der Entstehung von Wasser.
\subsection*{Corioliskraft}
Bewegung eines Vektors (Bei Rotationsachse $\Omega$ und Rotation um den Ursprung):
$\frac{d}{dt} \vv = \Omega \times \vv$\\
Aus der Produktregel folgt $\vec a_C = -2\vec \Omega \times \vec v$.\\
Bei Axialbewegung $v_A$ mit der Drehachse gibt es keine Beschleunigung.\\
Bei Radialbewegung $v_R$ weg von der Drehachse: $a_{\tn{Cor}} = 2 \cdot \omega \cdot v_R$ entgegen der Erdrotation (also nach Westen).
\newpar
Bei Tangentialbewegung $v_T$ mit der Drehung (Ost-West):\\ 
$a_{\tn{Cor}} = 2 \cdot \omega \cdot v_T$. Bei Bewegung mit der Erdrotation würde die Bewegung des Objekts es dazu bringen, sich von der Rotationsachse zu entfernen - wenn es wieder von der Schwerkraft nach unten bewegt wird, "rutscht" es dabei Richtung Äquator.
\newpar
Die Tangentialgeschwindigkeit ist bereits identisch zur Geschwindigkeit bei Bewegung nach Westen, also $v_T = -v_O$.\\
Die Radialgeschwindigkeit hängt vom Breitengrad ab -
am Breitengrad $\phi \in (-\pi, \pi)$ gilt $v_r = v \sin(\phi)$, es folgt:\\
$\binom{a_N}{a_O} = 2 \cdot \omega \cdot \sin(\phi) \cdot \binom{v_N}{-v_O} := f_C \cdot \binom{v_N}{-v_O}$\\
Insgesamt führt auf der Nordhalbkugel $(\sin(\phi) > 0)$ jede Bewegung zu einer Kraft nach Rechts, auf der Südhalbkugel umgekehrt.
\end{document}