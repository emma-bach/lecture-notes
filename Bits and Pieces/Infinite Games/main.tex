\documentclass{report}
\usepackage[a4paper,margin=1in]{geometry}
\usepackage{fancyhdr}
\usepackage[titles]{tocloft}
\usepackage[titletoc]{appendix}
\usepackage{tikz}
\usepackage{xcolor}

\usepackage{multicol}
\usepackage{amsmath}
\usepackage{amssymb}
\usepackage{amsthm}
\usepackage{pdfpages}
\usepackage{bm}
\usepackage{tikz-cd}
\usepackage{physics}
%\usepackage{bbm}
%\usepackage{biblatex}
%\addbibresource{bib.bib}

%hyperref should be last apparently
\usepackage{hyperref}

\renewcommand\cftsecdotsep{\cftdot}
\renewcommand\cftsubsecdotsep{\cftdot}
\renewcommand\epsilon{\varepsilon}

% Starts a new paragraph without indentation
% and with an empty line between paragraphs
\newcommand*{\newpar}{\par\vspace{\baselineskip}\noindent}
\newcommand{\trans}{\twoheadrightarrow}
\newcommand{\ttt}[1]{\texttt{#1}}
\newcommand{\tbf}[1]{\textbf{#1}}
\newcommand{\ul}[1]{\underline{#1}}


\newcommand{\bC}{\mathbb{C}}
\newcommand{\bF}{\mathbb{F}}
\newcommand{\bN}{\mathbb{N}}
\newcommand{\bQ}{\mathbb{Q}}
\newcommand{\bR}{\mathbb{R}}

\newcommand{\ve}{\vec{e}}
\newcommand{\vv}{\vec{v}}
\newcommand{\vw}{\vec{w}}
\newcommand{\vx}{\vec{x}}
\newcommand{\vy}{\vec{y}}
\newcommand{\vz}{\vec{0}}

\newcommand{\Mat}[3]{\text{Mat}^{#1}_{#2}\left(#3\right)}
\newcommand{\scalar}[2]{\left\langle #1, #2 \right\rangle}

%\renewcommand*\contentsname{Inhalt}
%\renewcommand*\proofname{Beweis}

\pagestyle{fancy} %allows headers

\lhead{Emma Bach}
\rhead{\today}


\begin{document}
% \newtheorem{codename}{printedname}[countedwith]
\newtheorem{lemma}{Lemma}[chapter]
\newtheorem{theorem}[lemma]{Satz}
\newtheorem{proposition}[lemma]{Proposition}
\newtheorem{corollary}[lemma]{Korollar}



\theoremstyle{definition}
\newtheorem{definition}[lemma]{Definition}
\newtheorem{beispiel}[lemma]{Beispiel}
\newtheorem{beobachtung}[lemma]{Beobachtung}
\newtheorem{anmerkung}[lemma]{Anmerkung}
\newtheorem{question}[lemma]{Frage}
\newtheorem{application}[lemma]{Anwendung}
\newtheorem{konsequenz}[lemma]{Konsequenz}
%
%
%
\begin{titlepage}
	\centering
	{\Large \textsc{Mitschrieb}\par}
	\vspace{0.5cm}
	{\huge\bfseries Numerik\par}
    \vspace{0.5cm}
	{\Large\itshape Emma Bach\par}
	\vfill
	
	Basierend auf:\par 
	\vspace{1cm}
	Vorlesungen Numerik I + II von\par
	Prof. Dr. Patrick  \textsc{Dondl}\par
	
	\vfill

% Bottom of the page
	{\large \today\par}
\end{titlepage}

\tableofcontents
\thispagestyle{fancy}
\clearpage
%
%
%
%
%
%
%
%
%
\section{Cantor Spaces and Baire Spaces}
Cantor Space: $2^\omega$ (Infinite sequences of binary Numbers)
\newpar
Baire Space: $\omega^\omega$ (Infinite Sequences of arbitrary Natural Numbers?)
\newpar
We consider spaces $X^\omega$, where $X$ is an arbitrary countable ordinal.
\newpar
Let $s \in X^{<\omega}$ then $N_s = \{x \in X^\omega : x \supseteq s\}$. We call \tbf{Baire Topology} the topology generated by these sets $N_s$.
\newpar
We set a \tbf{Lebesque Measure} $\mu$ on $2^\omega$ (and on $\omega^\omega$). In the case of a cantor space this is $\mu(N_s) = 2^{-|s|}$. In the case of $\omega^\omega$ finding a measure is more complicated.
\newpar
We have two notions of \tbf{small subsets} of a baire space:
\begin{itemize}
 \item $\mathcal{N} = \{X \subset 2^\omega : \mu(X) = 0\}$
 \item $\mathcal{M} = \{X \subset 2^\omega : X \text{is meager}\}$
\end{itemize}
%
\begin{definition}
 We say $A \subseteq 2^\omega$ is \tbf{nowhere dense} iff. 
 \begin{align*}
  \forall s \in 2^{<\omega} : \exists t \supseteq s : N_t \cap A = \emptyset
 \end{align*}
\end{definition}
\begin{definition}
 A set is called \tbf{meager} if it is a countable union of closed nowhere dense sets.
\end{definition}
%
\section{Infinite Games}
An \tbf{Infinite Game} is an infinite Sequence $x = (x(0), x(1), x(2), x(3), \hdots) \in X^\omega$, where player $I$ picks an element $x(0) \in X$, then player $II$ replies by picking an element $x(1) \in X$, etc.
\newpar
Consider a subspace $A \subseteq X^\omega$, which we call the set of outputs of the game.
\newpar
Given a sequence of moves $x \in X^\omega$, we say that $I$ \tbf{wins the game $G_X(A)$} iff. $x \in A$ and that $II$ wins the game iff. $x \notin A$.
\newpar
For player $I$, a \tbf{strategy} is a function 
\begin{align*}
 \sigma : \bigcup_{n \leq \omega} X^{2n} \to X
\end{align*}
We analogously define a trategy for player $II$ as a function:
\begin{align*}
 \tau : \bigcup_{n \leq \omega} X^{2n+1} \to X
\end{align*}
Intuitively, these functions assign to each point of the game (i.e. to each sequence $X^k$ of odd / even length) a move that the given player takes at that point.
\newpar
Let $y \in X^\omega$ enumerate the moves of player $II$. We denote $\sigma * y = (\sigma(\emptyset), \sigma(\sigma(\emptyset), y(0)), \hdots)$. Let $x \in X^\omega$ analogously denote the moves of player $I$.
\newpar
We call $\sigma$ a \tbf{winning strategy} for the game $G_X(A)$ iff. $\{\sigma * y\} \subseteq A$. We analogously call $\tau$ a winning strategy if $\{x * \tau\} \cap A = \emptyset$. If one of the players has a winning strategy, we say that the game is \tbf{determined}.
\newpar
Given a line of play $x \in X^\omega$, we write $x_I(n) = \{x(2n)\}$ and $x_{II}(n) = \{x(2n+1)\}$
\newpar
Is there always a winning strategy for one of the players? The answer is independent of ZF! However, it turns out that it is not independent of ZFC - The statement ``every game is determined'' is known as the Axiom of Determinancy, and it contradicts the full Axiom of Choice.)
\newpar
\begin{theorem}
 There exists $A$ such that $G_X(A)$ (where $X$ is countable) is not determined.
\end{theorem}
\begin{proof}
 Since $X$ is countable, the set $X^{<\omega}$ of finite sequences of $X$ is countable. Therefore the sets of strategies $\sigma$ and $\tau$ have size $2^{\aleph_0}$. Let $\{\sigma_\alpha : \alpha < 2^{\aleph_0}\}$ enumerate all strategies for $I$ and let $\tau_\alpha$ enumerate all strategies for $II$.
 \newpar
 Now we can define recursively two sets $A,B \subseteq X^\omega$ with $A = \{a_\alpha : \alpha < 2^{\aleph_0}\}$ and $B = \{b_\alpha : \alpha < 2^{\aleph_0}\}$ as follows:
 \begin{itemize}
  \item Assume $a_i$ and $b_i$ with $i < \alpha$ are already defined.
  \item Then define $\sigma_\alpha * y = b_\alpha$ for some $y \in X^\omega$ and set $b_\alpha \notin \{a_i : i < \alpha\}$. This is possible since $|\{\alpha_i : i < \alpha\}| < 2^{\aleph_0}$, but $|\sigma_\alpha  * y| = 2^{\aleph_0}$
  \item Now let $x * \tau_\alpha = a_\alpha$ for some $x \in X^\omega$ s.t. $a_\alpha \notin \{b_i : i < \alpha\}$.
  \item Now player $I$ cannot have a winning strategy for $G_X(A)$ and neither can player $II$.
 \end{itemize}
 (This proof implicitly uses the well-ordering theorem, so it isn't valid in ZF.)
\end{proof}
\begin{theorem}
 \emph{\tbf{Gale-Stewart:}} If $X$ is open or closed, then $X$ is determined.
\end{theorem}
\begin{proof}
 Let $B \subseteq X^\omega, s \in X^{<\omega}$. Define 
 \begin{align*}
  B / S := \{x \in X^\omega : s \circ x \in B\}
 \end{align*}
 Note that if $I$ has no winning strategy in $G_X(B / S)$, then $\forall i \in X : \exists j \in X$ s.t. $I$ has no winning strategy in $G_X(B / S \circ (i,j))$ either. We prove this by contradiction: If $\exists i \in X : \forall j \in X$ $I$ has a winning strategy $\sigma$ in $G_X(B/S \circ (i,j)$, then we would get a winning strategy for $I$ in $G_X(B/S)$ (by playing $i$ and the continuing according to $\sigma$).
 \newpar
 Now, suppose that $A \subseteq X^\omega$ is open. Assume $I$ has no winning strategy in $G_X(A)$
 . We can describe a strategy $\tau$ for $II$ thanks to our lemma such that for any partial play according to $\tau$, $I$ has no winning strategy in $G_X(A/S)$.
 \newpar
 By contradiction: Assume that $x$ is a line of play according to $\tau$ such that $x \in A$. Since $A$ is open, we have $N_{x \upharpoonright 2n} \subseteq A$. But then $I$ should have a winning strategy in $G_X(A / x \upharpoonright 2n)$.
 \newpar
 Therefore, we have $x \notin A$, which means $\tau$ is a winning strategy.
\end{proof}

%
%
%
%
%
%
%
%
%
%
%
\end{document}
