\documentclass{article}

\begin{document}
	\section{Maßtheorie}
	\begin{enumerate}
		\item Alle relevanten Definitionen, insbesondere:
		\begin{enumerate}
			\item $\sigma$-Algebra
			\item Maß
			\item Vollständiges Maß
			\item Messbare Menge
		\end{enumerate}
		\item Messbare Abbildungen (2.7, 2.9)
		\item Produkte von Messräumen (Universelle Eigenschaft!!)
	\end{enumerate}
	\section{Lebesquemaß}
	\begin{enumerate}
		\item Konstruktion eher nicht so wichtig
		\item $\lambda^n$-messbare Mengen (6.9)
		\item Charakterisierung 6.15,6.17,6.12
		\item Vitalimenge
		\item Cantormenge
	\end{enumerate}
	\section{Integration}
	\begin{enumerate}
		\item Approximation durch Treppenfunktionen
		\item Monotone und majorisierte Konvergenz
		\item Regelintegral = Lebesqueintegral (8.9)
		\item Differenzierbarkeit unter dem Integral (8.12)
	\end{enumerate}
	\section{$L^p$-Räume}
	\begin{enumerate}
		\item Höldersche Ungleichung
		\item Fischer-Riesz
		\item Partielle Integration!! (10.20)
		\item Radon-Nykodym
		\item Absolutstetigkeit
	\end{enumerate}
	\section{Transformationssatz}
	\section{Untermannigfaltigkeiten}
	\begin{enumerate}
		\item Regulärwerte
		\item Integralsatz von Gauß
		\item Integral auf Untermannigfaltigkeiten
	\end{enumerate}
\end{document}