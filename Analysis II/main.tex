\documentclass{report}
\usepackage[a4paper,margin=1in]{geometry}
\usepackage{fancyhdr}
\usepackage[titles]{tocloft}
\usepackage[titletoc]{appendix}
\usepackage{tikz}
\usepackage{xcolor}

\usepackage{multicol}
\usepackage{amsmath}
\usepackage{amssymb}
\usepackage{amsthm}
\usepackage{pdfpages}
\usepackage{bm}
\usepackage{tikz-cd}
\usepackage{physics}
%\usepackage{bbm}
%\usepackage{biblatex}
%\addbibresource{bib.bib}

%hyperref should be last apparently
\usepackage{hyperref}

\renewcommand\cftsecdotsep{\cftdot}
\renewcommand\cftsubsecdotsep{\cftdot}
\renewcommand\epsilon{\varepsilon}

% Starts a new paragraph without indentation
% and with an empty line between paragraphs
\newcommand*{\newpar}{\par\vspace{\baselineskip}\noindent}
\newcommand{\trans}{\twoheadrightarrow}
\newcommand{\ttt}[1]{\texttt{#1}}
\newcommand{\tbf}[1]{\textbf{#1}}
\newcommand{\ul}[1]{\underline{#1}}

\newcommand{\bC}{\mathbb{C}}
\newcommand{\bF}{\mathbb{F}}
\newcommand{\bN}{\mathbb{N}}
\newcommand{\bQ}{\mathbb{Q}}
\newcommand{\bR}{\mathbb{R}}

\newcommand{\ve}{\vec{e}}
\newcommand{\vv}{\vec{v}}
\newcommand{\vw}{\vec{w}}
\newcommand{\vx}{\vec{x}}
\newcommand{\vy}{\vec{y}}
\newcommand{\vz}{\vec{0}}

\newcommand{\Mat}[3]{\text{Mat}^{#1}_{#2}\left(#3\right)}
\newcommand{\scalar}[2]{\left\langle #1, #2 \right\rangle}

\renewcommand*\contentsname{Inhalt}
\renewcommand*\proofname{Beweis}

\pagestyle{fancy} %allows headers

\lhead{Emma Bach}
\rhead{\today}


\begin{document}
% \newtheorem{codename}{printedname}[countedwith]
\newtheorem{lemma}{Lemma}[chapter]
\newtheorem{theorem}[lemma]{Satz}
\newtheorem{proposition}[lemma]{Proposition}
\newtheorem{corollary}[lemma]{Korollar}



\theoremstyle{definition}
\newtheorem{definition}[lemma]{Definition}
\newtheorem{beispiel}[lemma]{Beispiel}
\newtheorem{beobachtung}[lemma]{Beobachtung}
\newtheorem{anmerkung}[lemma]{Anmerkung}
\newtheorem{question}[lemma]{Frage}
\newtheorem{application}[lemma]{Anwendung}
\newtheorem{konsequenz}[lemma]{Konsequenz}
%
%
%
\begin{titlepage}
	\centering
	{\Large \textsc{Study Notes}\par}
	\vspace{0.5cm}
	{\huge\bfseries Handbook of Logic in Computer Science\par}
    \vspace{1cm}
	{\Large\itshape Emma Bach\par}

% Bottom of the page
	{\large \today\par}
\end{titlepage}

\tableofcontents
\thispagestyle{fancy}
%
%
%
%
%
%
%
%
%
\chapter{Wiederholung}
%
%
%
%
%
%
%
%
%
\chapter{Der Euklidische Raum}
\begin{lemma}
 Sei $(V, \scalar{\_}{\_})$ ein euklidischer Vektorraum. Dann wird durch
 \begin{align*}
  \norm{u} = \sqrt{\scalar{u}{u}}
 \end{align*}
auf $V$ eine Norm erklärt. Diese bezeichnet man als die durch das Skalarprodukt induzierte Norm.
\end{lemma}
\begin{definition}
 Seu $(V, \scalar{\_}{\_})$ ein euklidischer Vektorraum, Die Vektoren $u, v \in V$ heißen \tbf{orthogonal}, wenn \begin{align*}
   \scalar{u}{v} = 0                                                                                                                 
 \end{align*}
 ist. Für $u, v \in V \setminus \{0\}$ Wird die reelle Zahl
 \begin{align*}
  \phi = \arccos \frac{\scalar{u}{v}}{\norm{u}\ \norm{v}}
 \end{align*}
 als der Winkel zwischen $u$ und $v$ bezeichnet.
\end{definition}
\begin{anmerkung}
 Es gilt
 \begin{align*}
  \frac{\abs{\scalar{u}{v}}}{\norm{u}\ \norm{v}} \leq 1
 \end{align*}
\end{anmerkung}
\begin{lemma}
\label{lemma:normequiv}
 Für $X = (x_1, \hdots, x_n) \in \bR^n$ sei
 \begin{align*}
  \norm{X}_{\max} := \max\{\abs{x_1}, \hdots, \abs{x_n}\}
 \end{align*}
 Dann ist $||\_||_{\max}$ eine Norm auf $\bR^n$ und es gilt
 \begin{align*}
  \norm{X}_{\max} \leq \norm{X} \leq \sqrt{n}\norm{X}_{\max}
 \end{align*}
\end{lemma}
\begin{theorem}
Die Menge $\bQ^n$ der Punkte mit rational Koordinaten ist dicht in $\bR^n$.
\end{theorem}
\begin{proof}
Sei $X \in \bR^n$ und $\epsilon \in \bR^+$. Da $\bQ$ dicht in $\bR$ ist gilt
\begin{align*}
 \forall i \in \{1, \hdots, n\} : \exists y_i \in \bQ : \abs{x_i - y_i} \leq \frac{\epsilon}{\sqrt{n}}
\end{align*}
Durch Lemma \ref{lemma:normequiv} folgt:
\begin{align*}
 \norm{x-y} \leq \sqrt{n}\norm{X - Y} < \epsilon
\end{align*}
\end{proof}
\begin{theorem}
\label{theorem:componentcauchy}
 Sei $(X_k)_{k \in \bN}$ eine Folge aus $\bR^n$. Sei $X_k = (x_1^{(k)}, \hdots, x_n^{(k)})$. Dann gilt:
 \begin{align*}
  \lim_{k \to \infty} X_k = X \Leftrightarrow \forall i : \lim_{k \to \infty} x_i^{(k)} = x_i
 \end{align*}
 Insbesondere ist $X_k$ eine Cauchyfolge, wenn die Komponenten Cauchyfolgen sind.
\end{theorem}
\begin{proof}
 $X_k \to X$, $i \in \{1, \hdots, n\}$, $\epsilon \in \bR^+$. Dann gilt
 \begin{align*}
  \exists k_o \in \bN : \forall k \geq k_0 : \norm{X_k - X} \leq \epsilon \implies \forall i : \abs{x_i^{(k)} - x_i} < \epsilon \implies \lim_{k \to \infty} x_i^{(k)} = x_i
 \end{align*}
 Und umgekehrt:
 \begin{align*}
  \forall i : x_i^{(k)} \to x_i, \epsilon \in \bR^+ \implies \exists k_0^i \in \bN : \forall k \geq k_0^i \abs{x_i^{(k)} - x_i} \leq \frac{\epsilon}{\sqrt{n}}
 \end{align*}
 \[
  k_0 := \max\{k_0^n, \hdots, k_0^n\} \implies \forall k \geq k_0 : \abs{x_i^{(k)} - x_i} < \frac{\epsilon}{\sqrt{n}} \implies \norm{X_k - X} \leq \sqrt{n}\norm{X_k - X} < \epsilon
 \]
\end{proof}
\begin{theorem}
 Für konvergente Folgen $(X_k),(Y_k) \in \bR^n$, $(\lambda_k) \in \bR$ gilt:
\begin{align}
 \lim_{k \to \infty} (X_k + Y_k) = \lim_{k \to \infty} X_k + \lim_{k \to \infty} Y_k\\
 \lim_{k \to \infty} \lambda_k X_k = \left(\lim_{k \to \infty} \lambda_k\right)\left(\lim_{k \to \infty}
 X_k\right)\\
 \lim_{k \to \infty} \scalar{X_k}{Y_k} = \scalar{\lim_{k \to \infty} X_k}{\lim_{k \to \infty} Y_k}
\end{align}
\end{theorem}
\begin{theorem}
 $\bR^n$ ist vollständig.
\end{theorem}
\begin{proof}
Ist $X_k$ eine Cauchyfolge in $\bR^n$, so sind nach Satz \ref{theorem:componentcauchy} alle Teilfolgen Cauchy in $\bR$. Also:
\begin{align*}
 \exists x_i \in \bR : x_i^{(k)} \to x_i \implies \exists X \in \bR^n : X_k \to X
\end{align*}
\end{proof}
\begin{theorem}
\emph{\textbf{(Bolzano-Weierstrass:)}} Jede beschränkte Folge in $\bR^n$ besitzt eine konvergente Teilfolge.
\end{theorem}
\begin{proof}
 Sei $(X_k)$ eine beschränkte Folge in $\bR^n$. Nach \ref{lemma:normequiv} müssen die Komponentenfolgen ebenfalls beschränkt sein. Nach dem eindimensionalen Fall des Satzes von Bolzano-Weierstrass existieren also konvergente Teilfolgen der Koordinatenfolgen. Angenommen, die konvergente Teilfolge der ersten Komponente ist gegeben durch $x_1^{(k_n)} \to x_1$. So ist $x_2^{(k_n)}$ ebenfalls eine beschränkte Teilfolge, also existiert eine Teilfolge $x_2^{(k_n)_m}$ welche in den ersten beiden Komponenten konvergiert. Führt man dieses Verfahren induktiv fort, erhält man eine konvergente Teilfolge von $(X_k)$.
\end{proof}
\begin{theorem}
 Sei $(A_i)_{i \in \bN}$ eine Folge abgeschlossener beschränkter nichtleerer Teilmengen des $\bR^n$, sodass $A_1 \supseteq A_2 \supseteq \hdots$ Dann ist $\bigcap_{i \in \bN} \neq \emptyset$
\end{theorem}
\begin{proof}
 $A_i \neq \emptyset \implies \exists X_i \in A$ sd. $(X_i)_{i \in \bN}$ eine Folge ist. Da $A_i$ beschränkt ist ist $(X_i)_{i \in \bN}$ beschränkt, also hat $X_i$ eine konvergente Teilfolge $X_{i_k}$ mit Limes $X$. Es gilt $X_{i_k} \in A_{i_k} \subseteq A_i$, also ist $X$ ein Berührpunkt von $A_i$, also $X \in A_i$.
\end{proof}
\begin{theorem}
 Jede abgeschlossene beschränkte Teilmenge des $\bR^n$ ist kompakt.
\end{theorem}
\begin{proof}
 Analog zur eindimensionalen Version, wobei statt Intervallen $[a_i,b_i]$ Hyperwürfel $[a_i^{(1)}, b_i^{(1)}] \times \hdots \times [a_i^{(n)}, b_i^{(n)}]$ genutzt werden müssen.
\end{proof}
\begin{theorem}
\label{theorem:allnormsequiv}
 Seien $\norm{\_}_1$ und $\norm{\_}_2$ Normen auf $\bR^n$. So existieren $k, K \in \bR^+$ mit
 \begin{align*}
  \forall X \in \bR^n : k\norm{X}_1 \leq \norm{X}_2 \leq K\norm{X}_1
 \end{align*}
\end{theorem}
\begin{proof}
Diese Normenäquivalenz bildet eine Äquivalenzrelation. Es reicht also, zu zeigen, dass jede Norm $||\_||_2$ äquivalent zu einer spezifischen Norm $\norm{\_}_1$ ist. Wir wählen $\norm{\_}_{\max}$.\\
Sei $(E_i)$ die Standardbasis des $\bR^n$. Wir definieren:
\begin{align*}
 K := \norm{E_1}_2 + \hdots + \norm{E_n}_2
\end{align*}
Dann gilt:
\begin{align*}
 \norm{X}_2 &= \norm{x_1 E_1 + \hdots + x_n E_n}\\
         &\leq \abs{x_1}\norm{E_1}_2 + \hdots + \abs{x_n} \norm{E_n}_2\\
         &\leq \norm{X}_{\max} K \quad ^\text{[citation needed]}
\end{align*}
Es bleibt die Rückrichtung zu zeigen. 
\begin{lemma}
 $f(X) := \norm{X}_2$ ist stetig.
\end{lemma}
\begin{proof}
\begin{align*}
 \abs{\norm{X}_2 - \norm{Y}_2} \leq \norm{X - Y}_2 \leq K\norm{X - Y}_{\max} \leq K \norm{X - Y}
\end{align*}
Also ist $\norm{\_}_2$ stetig bezüglich der euklidischen Norm $\norm{\_}$.
\end{proof}
Wir definieren nun:
\begin{align*}
 A := \{X \in \bR^n \mid ||X||_{\max} = 1\}
\end{align*}
Diese Menge ist beschränkt. Wir wollen Zeigen, dass sie außerdem abgeschlossen ist. Sei $X_i \to X$, $X_i \in A$. Es gilt:
\begin{align*}
  \abs{\norm{X_i}_{\max} - \norm{X}_{\max}} \leq \norm{X_i - X}_{\max} \leq \norm{X_i - X}
\end{align*}
Also konvergiert jede Menge, also ist $A$ kompakt, also auch abgeschlossen. Dementsprechend muss $f$ auf $A$ ein Minimum $k$ annehmen. Wir wissen $f \geq 0$, also ist$k \geq 0$. Es gilt sogar $k > 0$, da keiner der Vektoren in $A$ der Nullvektor ist. Nun gilt also $\forall X \in A : ||X||_2 \geq k$. Wir definieren:
\begin{align*}
 \lambda := \frac{1}{\norm{X}_{\max}}
\end{align*}
\begin{align*}
 \norm{\lambda X}_{\max} = \abs{\lambda} \norm{X}_{\max} = 1 
\end{align*}
\begin{align*}
 |\lambda| \norm{X}_2 = \norm{\lambda X}_2 \geq k \implies \norm{X_2} \geq k \norm{X}_{\max}
\end{align*}
\end{proof}
\begin{anmerkung}
 Im unendlichdimensionalen Fall gilt Satz \ref{theorem:allnormsequiv} nicht.
\end{anmerkung}
%
%
%
%
%
%
%
%

%
%
%
%
\section{Abbildungen und Koordinatenfunktionen auf $\bR^n$}
In diesem Abschnitt betrachten wir Funktionen $F: \bR^n \to \bR^k$. Betrachten wir zuerst den Spezialfall Linearer Funktionen, also $\forall X,Y \in \bR^n : \forall \lambda, \mu \in \bR : F(\lambda X + \mu Y) = \lambda F(X) + \mu F(Y)$.
\newpar
Sei $(E_i)$ die Standardbasis des $\bR^n$ und sei $(E_i')$ die Standardbasis des $\bR^k$. Nun gilt:
\begin{align*}
 F(E_j) = \sum_{i=1}^k a_{ij} E_i'
\end{align*}
Daraus erhalten wir Koeffizienten $a_{ij}$, welche eine Matrix bilden. Umgekehrt können wir aus den Koeffizienten die Abbildung $F$ rekonstruieren, indem wir definieren:
\begin{align*}
 F(X) &= F\left(\sum_{j=1}^n x_j E_j\right) \\
 &= \sum x_j F(E_j) \\
 &= \sum_{j=1}^n x_j \sum_{i=1}^k a_{ij} E_i'\\
 &= \sum_{i=1}^k \left(\sum_{j=1}^n a_{ij} x_j\right) E_i'
\end{align*}
%
%
%
%
%
%
%
%
%
%
%
%
%
%
%
\newpage
[missing stuff here]
\newpage
\begin{definition}
 Wir bezeichnen als $p_i : M \to k$ die Projektion eines Vektors auf die $i$-te Komponente.
\end{definition}
\begin{theorem}
\label{theorem:componentwisecontinuous}
 Sei $M$ ein metrischer Raum, $F : M \to \bR^n$ eine Abbildung und $x \in M$. Dann ist $F$ stetig in $x$ genau dann, wenn $p_i \circ F$ stetig für alle $i$ ist.
\end{theorem}
\begin{proof}
\begin{enumerate}
 \item $p_i$ ist stetig. Ist also $F$ stetig folgt direkt, dass auch $p_i \circ F$ stetig ist.
 \item Angenommen, $p_i \circ F$ ist stetig $\forall i$, $\epsilon \in \bR^+$. Da $p_i \circ F$ stetig ist existiert eine Umgebung $U_i$ von $x$, sodass $|f_i(x) - f_i(y)| < \frac{\epsilon}{\sqrt{n}} \forall y \in U_i$. Ebenso für die anderen Komponenten. Nun gilt:
 \begin{align*}
  \norm{F(y) - F(x)} \leq \sqrt{n} \norm{F(x) - F(y)}_{\max} \leq \epsilon
 \end{align*}
\end{enumerate} 
\end{proof}
Analog gilt das Selbe für Stetigkeit auf $M$, gleichmäßige Stetigkeit, etc.
\begin{definition}
 Sei $M \subseteq \bR^n$, $F : M \to \bR^k$ eine Abbildung, $x_0$ ein Häufungspunkt, $y \in \bR^k$. Dann definieren wir:
 \begin{align*}
  \lim_{x \to x_0} F(x) = y \Leftrightarrow \forall \epsilon \in \bR^+ : \exists \delta \in \bR^+ : \forall x \in M \setminus \{x_0\} : \norm{x - x_0} \leq \delta \implies \norm{F(x) - y} < \epsilon
 \end{align*}
 $F$ ist stetig in $x_0$ genau dann, wenn $\lim_{x \to x_0} F(x) = F(x_0)$.
\end{definition}
\begin{theorem}
 Sei $M \subseteq \bR^n$, $F : M \to \bR^k$ eine Abbildung, $X_0 \in M$ ein Häufungspunkt, $Y \in \bR^k$ und $f_i = p_i \circ F$. Dann gilt:
 \begin{align*}
  \lim_{X \to X_0} F(X) = Y \Leftrightarrow \forall i : \lim_{X \to X_0} f_i(X) = y_i
 \end{align*}
\end{theorem}
\begin{proof}
 Analog zu Beweis \ref{theorem:componentwisecontinuous}.
\end{proof}
\begin{corollary}
 \begin{align*}
  F(X) \to Y, G(X) \to Z \implies F(X) + G(X) \to Y + Z
 \end{align*}
\end{corollary}
\section{Mehrdimensionale Ableitungen}
\begin{beispiel}
 Sei $f : M \to \bR$ definiert auf einer offenen Menge $M \subseteq \bR^n$.
 \begin{align*}
  f(X) = f(x_1, \hdots, x_n) \text{ bzgl. der Standardbasis }
 \end{align*}
 Wir können aber auch $X = \sum x_i' E_i'$ bezüglich einer beliebigen anderen Basis darstellen. Also:
 \begin{align*}
  f(X) = f(x_1, \hdots, x_n) = g(x_1', \hdots, x_n')
 \end{align*}
 Da $f$ in der Regel nicht linear ist, ist ein solcher Basiswechsel sehr viel komplizierter als in der Linearen Algebra! Wo möglich ist es also besser, über $f(X)$ zu reden.
\end{beispiel}
\begin{definition}
 Sei $f : \bR^n \to \bR$, $\overline{X} \in M$. Betrachte die Abbildung \[t \to f(\overline{x}_1, \hdots \overline{x}_{i-1}, t, \overline{x}_{i+1}, \hdots, \overline{x}_n),\] welche eine Mehrdimensionale Funktion $f(x_1, \hdots x_n)$ auf eine eindimensionale Funktion $f(t)$ abbildet. \ul{Achtung:} Wir nehmen hier implizit eine Darstellung bezüglich der Standardbasis an!
\end{definition}
\begin{beispiel}
 Betrachte folgende Funktion:
 \begin{align*}
  f(x,y) = \begin{cases}
            \frac{xy}{x^2 + y^2} & (x,y) \neq (0,0)\\
            0 & (x,y) = (0,0)\\
           \end{cases}
 \end{align*}
 $f$ ist an $(0,0)$ partiellen differenzierbar, die Partiellen Ableitungen sind $0$. Allerdings gilt \begin{align*}
  \forall x : f(x,x) = \frac{1}{2}
 \end{align*}
 Also ist $f$ an $0$ nicht stetig! Es existieren also Funktionen, die an einem Punkt partiell Differenzierbar sind, an dem sie nicht stetig sind.
\end{beispiel}
\ul{Idee:} Fordere partielle Differenzierbarkeit bezüglich jeder möglichen Basis, also partielle Differenzierbarkeit in jedem Vektor.
\begin{beispiel}
 \begin{align*}
  f(x,y) = \begin{cases}
            \frac{x^2y}{x^4 + y^2} & (x,y) \neq (0,0)\\
            0 & (x,y) = (0,0)\\
           \end{cases}
 \end{align*}
 Wir betrachten die ``Linearisierung'' $t \to f(t, \alpha t)$. Einsetzen liefert:
 \begin{align*}
  f(t, \alpha t) = \frac{\alpha t}{t^2 + a^2}
 \end{align*}
 Diese Funktion ist differenzierbar, also ist $f$ differenzierbar bezüglich beliebiger Basen. Das reicht jedoch immer noch nicht:
 \begin{align*}
  f(a, a^2) = \frac{a^2a^2}{a^4+a^4} = \frac{1}{2}
 \end{align*}
 Also ist $f$ immer noch nicht stetig - es ist stetig für Folgen, welche den Nullpunkt durch Geraden erreichen, aber nicht, wenn wir durch kompliziertere Pfade gegen den Nullpunkt gehen.
\end{beispiel}
\newpar
Wir wollen die Begriffe aus der Analysis I über Stetigkeit und Ableitbarkeit retten, also brauchen wir einen komplizierteren Ableitungsbegriff.
\section{Differenzierbarkeit}
Sei $f$ eine beliebige Funktion $\bR \to \bR$. Die Ableitung gibt uns die Tangente der Funktion an einem beliebigen Punkt, also die beste affine Approximation der Funktion an diesem Punkt.
\begin{definition}
 Eine Funktion $F : \bR^n \to \bR^k$ heißt \tbf{affin}, wenn es eine Lineare Funktion $L : \bR^n \to \bR^k$ und eine Konstante $Z \in \bR^k$ gibt, sodass:
 \begin{align*}
  F(X) = L(X) + Z
 \end{align*}
\end{definition}
\newpar
Sei $g : \bR \to \bR$ affin, also $g(x) = cx + t$ für $c,t \in \bR$. Sei $f : \bR \to \bR$. Wir wollen eine beliebige Funktion $f$ an der Stelle $x_0$ approximieren. Für eine gute Approximation wollen wir $f(x_0) = g(x_0)$, also erhalten wir:
\begin{align*}
 g(x) = c(x - x_0) + f(x_0).
\end{align*}
Schreibe $x = x_0 + h$ und lasse $h$ gegen $0$ gehen.
\begin{align*}
 h \to f(x_0 + h) - g(x_0 + h) = f(x_0 + h) - f(x_0) - ch
\end{align*}
Wir sagen, die Approximation ist gut, wenn $f(x_0 + h) - f(x_0) - ch$ schneller gegen $0$ geht als $h$ selbst, also:
\begin{align}
\label{eq:goodapprox}
 \lim_{h \to 0} \frac{f(x_0 + h) - f(x_0) - ch}{h} = 0
\end{align}
Was äquivalent ist zu:
\begin{align*}
 c = \lim_{h \to 0} \frac{f(x_0 + h) - f(x_0)}{h}\\
\end{align*}
Wir sagen also, $f$ ist in $x_0$ differenzierbar, genau dann, wenn eine lineare Abbildung $L$ existiert, sodass:
\begin{align*}
 \lim_{h \to 0} \frac{f(x_0 + h) - f(x_0) - L(h)}{h} = 0
\end{align*}
Diese geometrische Intuition, nach der die Ableitung die beste affine Approximation ist, können wir auf den $\bR^n$ übertragen. 
\begin{definition}
 Sei $M \subset \bR^n$ offen, $F : M \to \bR^k$ eine Abbildung, sei $X_0 \in M$. Die Abbildung $F$ heißt \tbf{differenzierbar} am Punkt $X_0$, wenn es eine Lineare Abbildung $L : \bR^n \to \bR^k$ gibt, sodass:
 \begin{align*}
  \lim_{H \to 0} \frac{F(X_0 + H) - F(X_0) - L(H)}{\norm{H}} = 0.
 \end{align*}
 Wir nennen sie das \tbf{Differenzial von $F$ im Punkt $X_0$} und notieren sie als $DF_{X_0}$. $F$ heißt differenzierbar, wenn sie differenzierbar an jedem Punkt $X \in M$ ist.
\end{definition}
\begin{theorem}
 Gibt es ein Differential, ist es eindeutig bestimmt.
\end{theorem}
\begin{proof}
 Seien $L_1, L_2$ Differentiale. Es folgt:
 \begin{align*}
  \lim_{H \to 0} \frac{L_1(H) - L_2(H)}{\norm{H}} = 0
 \end{align*}
 Sei $X \in \bR^n \setminus \{0\}$. Dann gilt:
 \begin{align*}
  \lim_{t \to 0} \frac{L_1(tX) - L_2(tX)}{\norm{tX}} = 0
 \end{align*}
 %also:
 \begin{align*}
  \implies \frac{L_1(X) - L_2(X)}{\norm{X}} = 0
 \end{align*}
 %also:
 \begin{align*}
  \implies L_1(X) - L_2(X) = 0
 \end{align*}
 also sind die beiden Differentiale identisch.
\end{proof}
\begin{anmerkung}
 Unserer Differenzierbarkeitsbegriff wird insbesonders in der älteren Literatur oft als \tbf{total Differenzierbarkeit} bezeichnet.
\end{anmerkung}

%
%
%
%
%
%
%
%
%
%
%
%
%
%
%
%
%
\end{document}
