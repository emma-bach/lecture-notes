\documentclass{report}

% custom margins
\usepackage[a4paper,margin=1.5in]{geometry}
\renewcommand{\baselinestretch}{1.2}

% emma's long list of custom macros and universally used packages
%AMS packages - Symbols, "Theorem" and "Proof" Environments
\usepackage{amsmath}
\usepackage{amssymb}
\usepackage{amsthm}
\usepackage{physics}
\renewcommand{\div}{\tn{div}~}

% Nicer Headers and Footers
\usepackage{fancyhdr}
\usepackage[yyyymmdd]{datetime}
\renewcommand{\dateseparator}{-}

% Table of Contents
\usepackage[titles]{tocloft}
\usepackage[titletoc]{appendix}

% Tikz and Graphics
\usepackage{tikz}
\usepackage{tikz-cd}
\usepackage{xcolor}
\usepackage[many]{tcolorbox}

% Nicer Underlining
\usepackage{contour}
\usepackage{ulem}

% Multiple Text Columns
\usepackage{multicol}

% FloatBarrier
\usepackage{placeins} 

% hyperref should be last apparently
\usepackage{hyperref}

%--- RENEWED COMMANDS ---

% Variant Greek Letters
\renewcommand\epsilon{\varepsilon}
\renewcommand\phi{\varphi}

% Nicer Table of Contents
\renewcommand\cftsecdotsep{\cftdot}
\renewcommand\cftsubsecdotsep{\cftdot}

%--- TEXT FORMATING ---

% nice underlining
\renewcommand{\ULdepth}{1.6pt}
\contourlength{0.8pt}
\newcommand{\ul}[1]{%
	\uline{\phantom{#1}}%
	\llap{\contour{white}{#1}}%
}

% shorthand
\newcommand{\tn}[1]{\textnormal{#1}}
\newcommand{\tbf}[1]{\textbf{#1}}
\newcommand{\tit}[1]{\textit{#1}}
\newcommand{\ttt}[1]{\texttt{#1}}

\newcommand{\mc}[1]{\mathcal{#1}}

\newcommand{\ol}[1]{\overline{{#1}}}

% Underlined, bold, non-cursive Theorem Name
\newcommand{\theoremname}[1]{\ul{\textnormal{\tbf{#1}}}}

% Starts a new paragraph without indentation
% and with an empty line between paragraphs
\newcommand*{\newpar}{\par\vspace{\baselineskip}\noindent}

% Make inf height match sup height
\renewcommand{\inf}{\mathop{\mathrm{inf}\vphantom{\mathrm{sup}}}}

% Make tildes more readable
\renewcommand{\tilde}{\widetilde}

%--- RELATION SYMBOLS AND OPERATORS ---
\newcommand{\surj}{\twoheadrightarrow}
\newcommand{\inj}{\hookrightarrow}
\newcommand{\iso}{\overset{\sim}{\rightarrow}}
\newcommand{\symdiff}{\vartriangle}
\newcommand{\trans}{\twoheadrightarrow}
\newcommand{\tensor}{\otimes}
\newcommand{\bigmid}{~\middle|~}

%--- STUFF IN FANCY BRACKETS ---
\newcommand{\scalar}[2]{\left\langle #1, #2 \right\rangle}
\newcommand{\angles}[1]{\left\langle #1 \right\rangle}
\newcommand{\lr}{\qty}


%--- LETTERS ---
\newcommand{\dmu}{\ d\mu}
\newcommand{\dist}{\textnormal{dist}}
\newcommand{\spt}{\textnormal{spt}}

% mathbb
\newcommand{\bC}{\mathbb{C}}
\newcommand{\bF}{\mathbb{F}}
\newcommand{\bN}{\mathbb{N}}
\newcommand{\bQ}{\mathbb{Q}}
\newcommand{\bR}{\mathbb{R}}
\newcommand{\bZ}{\mathbb{Z}}

% mathcal
\newcommand{\cA}{\mathcal{A}}
\newcommand{\cB}{\mathcal{B}}
\newcommand{\cC}{\mathcal{C}}
\newcommand{\cD}{\mathcal{D}}
\newcommand{\cE}{\mathcal{E}}
\newcommand{\cF}{\mathcal{F}}
\newcommand{\cH}{\mathcal{H}}
\newcommand{\cI}{\mathcal{I}}
\newcommand{\cL}{\mathcal{L}}
\newcommand{\cM}{\mathcal{M}}
\newcommand{\cN}{\mathcal{N}}
\newcommand{\cO}{\mathcal{O}}
\newcommand{\cP}{\mathcal{P}}
\newcommand{\cQ}{\mathcal{Q}}
\newcommand{\cR}{\mathcal{R}}
\newcommand{\cS}{\mathcal{S}}
\newcommand{\cT}{\mathcal{T}}
\newcommand{\cW}{\mathcal{W}}
\newcommand{\cX}{\mathcal{X}}
\newcommand{\cZ}{\mathcal{Z}}

% mathfrac
\newcommand{\fa}{\mathfrak{a}}

% real numbers in fancy costumes
\newcommand{\barR}{\ol{\mathbb{R}}}
\newcommand{\bRnn}{\mathbb{R}^{n \times n}}
\newcommand{\bRmn}{\mathbb{R}^{m \times n}}
\newcommand{\bRnm}{\mathbb{R}^{n \times m}}

% vectors
\renewcommand{\va}{\vec{a}}
\renewcommand{\vb}{\vec{b}}
\newcommand{\vc}{\vec{c}}
\newcommand{\ve}{\vec{e}}
\newcommand{\vF}{\vec{F}}
\newcommand{\vh}{\vec{h}}
\newcommand{\vp}{\vec{p}}
\newcommand{\vr}{\vec{r}}
\newcommand{\vs}{\vec{s}}
\newcommand{\vT}{\vec{T}}
\renewcommand{\vu}{\vec{u}}
\newcommand{\vv}{\vec{v}}
\newcommand{\vw}{\vec{w}}
\newcommand{\vW}{\vec{W}}
\newcommand{\vx}{\vec{x}}
\newcommand{\vy}{\vec{y}}
\newcommand{\vz}{\vec{z}}
\newcommand{\vzero}{\vec{0}}

\newcommand{\veta}{\vec{\eta}}

\renewcommand{\grad}{\vec{\nabla}}

% bold letters
\newcommand{\tbA}{\mathbf{A}}
\newcommand{\tbB}{\mathbf{B}}
\newcommand{\tbC}{\mathbf{C}}
\newcommand{\tbD}{\mathbf{D}}
\newcommand{\tbE}{\mathbf{E}}
\newcommand{\tbY}{\mathbf{Y}}
\newcommand{\tbZ}{\mathbf{Z}}

% sequences
\newcommand{\an}{(a_n)_{n \in \bN}}
\newcommand{\bn}{(b_n)_{n \in \bN}}
\newcommand{\sn}{(s_n)_{n \in \bN}}
\newcommand{\iinI}{_{i \in I}}
\newcommand{\iinN}{_{i \in \bN}}

% special functions
\newcommand{\at}{\textnormal{at}}
\newcommand{\ggT}{\textnormal{ggT}}
\newcommand{\kgV}{\textnormal{kgV}}
\newcommand{\id}{\textnormal{id}}
\newcommand{\Id}{\textnormal{Id}}
\newcommand{\im}{\textnormal{im}}
\newcommand{\inv}{\textnormal{inv}}
\newcommand{\ord}{\textnormal{ord}\ }
\newcommand{\rang}{\textnormal{rang}\ }
\renewcommand{\tr}{\textnormal{tr}\ }
\newcommand{\vol}{\textnormal{vol}}
\newcommand{\cond}{\textnormal{cond}}
\newcommand{\sgn}{\textnormal{sgn}}
\renewcommand{\char}{\textnormal{char}}
\newcommand{\Frac}{\textnormal{Frac}}
\newcommand{\Irr}{\textnormal{Irr}}

% groups and sets
\newcommand{\GL}{\text{GL}}
\newcommand{\SO}{\text{SO}}
\newcommand{\Hess}[1]{\text{Hess}(#1)}

\newcommand{\Grp}{\textnormal{Grp}}
\newcommand{\Mag}{\textnormal{Mag}}
\newcommand{\Mon}{\textnormal{Mon}}
\newcommand{\Ens}{\textnormal{Ens}}
\newcommand{\Hom}{\textnormal{Hom}}
\newcommand{\Aut}{\textnormal{Aut}}

\newcommand{\Mat}{\textnormal{Mat}}
\newcommand{\Matnn}{\textnormal{Mat}(n \times n)}
\newcommand{\MatBB}[3]{\textnormal{Mat}^{#1}_{#2}\left(#3\right)}



% --- THEOREM AND PROOF TYPES ---
% \newtheorem{codename}{printedname}[countedwith]
\newtheorem{lemma}{Lemma}[section]
\newtheorem{theorem}[lemma]{Satz}
\newtheorem{proposition}[lemma]{Proposition}
\newtheorem{corollary}[lemma]{Korollar}

\theoremstyle{definition}
\newtheorem{definition}[lemma]{Definition}
\newtheorem{summary}[lemma]{Zusammenfassung}
\newtheorem{example}[lemma]{Beispiel}
\newtheorem{counterexample}[lemma]{Gegenbeispiel}
\newtheorem{beobachtung}[lemma]{Beobachtung}
\newtheorem{anmerkung}[lemma]{Anmerkung}
\newtheorem{question}[lemma]{Frage}
\newtheorem{application}[lemma]{Anwendung}
\newtheorem{reminder}[lemma]{Erinnerung}
\newtheorem{konsequenz}[lemma]{Konsequenz}

% --- CUSTOM ENVIRONMENTS ---

% sketchy proofs marked with untrustworthy QED symbol
\newenvironment{proofsketch}{\begin{proof}[Beweisskizze]\renewcommand*{\qedsymbol}{\("\square"\)}}{\end{proof}}

% align with exactly one number for labeling
\NewDocumentEnvironment{nalign}{}{\equation\aligned}{\endaligned\endequation}

\makeatletter
\let\amsmath@bigm\bigm

% colored box behind proofs
\tcolorboxenvironment{proof}{
	colback=white,
	boxrule=0pt,
	leftrule=0.5mm,
	before skip=0.75cm,
	after skip=0.75cm,
	sharp corners,
	breakable,
	enhanced,
}

\renewcommand*\contentsname{Inhalt}
\renewcommand*\proofname{Beweis}

\pagestyle{fancy} %allows headers

\lhead{Emma Bach}
\rhead{\today}

\begin{document}

\begin{titlepage}
	\centering
	{\Large \textsc{Mitschrieb}\par}
	\vspace{0.5cm}
	{\huge\bfseries Numerik\par}
    \vspace{0.5cm}
	{\Large\itshape Emma Bach\par}
	\vfill
	
	Basierend auf:\par 
	\vspace{1cm}
	Vorlesungen Numerik I + II von\par
	Prof. Dr. Patrick  \textsc{Dondl}\par
	
	\vfill

% Bottom of the page
	{\large \today\par}
\end{titlepage}

\tableofcontents
\thispagestyle{fancy}

\chapter{Elementare Zahlentheorie}
\section{Primzahlen}
\begin{definition}
	$H \subseteq \bZ$ heißt Untergruppe von $\bZ$, wenn 
	\begin{align*}
		n \in H \implies -n \in H,
	\end{align*} und 
	\begin{align*}
		m, n \in H \implies m + n \in H
	\end{align*}
\end{definition}
\begin{theorem}
	Es gibt eine Bijektion 
	\begin{align*}
		\bN &\leftrightarrow \{\textnormal{Untergruppen von ($\bZ, + $)}\}\\
		n &\mapsto n\bZ
	\end{align*}
\end{theorem}
\begin{proof}
	Sei $H \subseteq \bZ$ eine Untergruppe. Entweder $H = \{0\} = 0\bZ$, oder $H \neq \{0\}$.
	\newpar
	Sei also $H \neq \{0\}$. Dann gibt es ein kleinstes Element $m$ der Menge $\{n \in H \mid n > 0\}$.
	Aus den Untergruppenaxiomen folgt $m\bZ \subseteq H$. Gleichzeitig kann kein Element $n \notin m \bZ$ in $H$ enthalten sein, denn sonst wäre auch $r = n \mod m \neq 0$ in $H$ enthalten. Dann hätten wir aber $r < m$, was ein Widerspruch ist.
	\newpar
	Als Umkehrfunktion wählen wir das kleinste positive Element von $H$.
\end{proof}
\begin{definition}
	Eine \tbf{Primzahl} ist eine natürliche Zahl $p \in \bN_{\geq 2}$, die nicht als Produkt zweier Zahlen $a,b < p$ geschrieben werden kann.
\end{definition}
\begin{theorem}
	Es gibt unendlich viele Primzahlen.
\end{theorem}
\begin{proof}
	Angenommen, es gäbe endlich viele Primzahlen. Sei also $p_1, \hdots, p_r$ eine vollständige Liste aller Primzahlen. Dann wäre aber
	\begin{align*}
		q = 1 + \prod_{i = 1}^{r} p_i
	\end{align*}
	durch keine Primzahl teilbar, also selbst eine Primzahl. Widerspruch!
\end{proof}
\begin{theorem}
	Jede Zahl $n \in \bN$ kann als Produkt von Primzahlen geschrieben werden:
	\begin{align*}
		n = p_1 \cdot \hdots \cdot p_r \quad (r \geq 0)
	\end{align*}
\end{theorem}
\begin{proof}
	Der Fall $n = 1$ gilt per Konvention durch das leere Produkt.
	\newpar
	Sei $n \geq 2$ gegeben. Es gilt entweder:
	\begin{itemize}
		\item $n$ ist eine Primzahl.
		\item $n$ ist von der Gestalt $n = a \cdot b$, mit $a,b < n$.
	\end{itemize}
	Der Satz folgt durch Induktion über die entstehende Baumstruktur - nach Induktionsannahme haben $a$ und $b$ eine Primfaktorzerlegung. Also hat auch $n$ eine Primfaktorzerlegung. 
\end{proof}
\begin{definition}
	Der \tbf{größte gemeinsame Teiler} von $a,b \in \bZ$ mit $a \neq 0$ oder $b \neq 0$ ist die Zahl:
	\begin{align*}
		\ggT(a,b) = \max\{d \in \bN : d \mid a \wedge d \mid b\}
	\end{align*}
\end{definition}
\begin{theorem}
	\theoremname{Über den größten gemeinsamen Teiler}: Seien $a,b \in \bZ$. So gibt es $r,s \in \bN$ mit 
	\begin{align*}
		\ggT(a,b) = ra + sb
	\end{align*}
	Gegeben $d \mid a$ und $d \mid b$ gilt außerdem $d \mid \ggT(a,b)$.
\end{theorem}
\begin{proof}
	Die Menge
	\begin{align*}
		H := \{ra + sb \mid r,s \in \bZ\} = a\bZ + b\bZ
	\end{align*}
	bildet eine Untergruppe von $Z$, ist also eine Gruppe der Form $m \bZ$ mit $m > 0$. Da $a \in m\bZ$ und $b \in m\bZ$ ist $m$ ein gemeinsamer Teiler. Da $m$ ein Element in $H$ ist existiert außerdem per Definition eine Darstellung $m = r'a + s'b$. Es gilt also 
	\begin{align*}
		(d \mid a) \wedge (d \mid b) \implies d \mid r'a + s'b \implies d \mid m.
	\end{align*}
	\newpar
	Aus $(d \mid a) \wedge (d \mid b) \implies d \mid m$ folgt nun, dass jeder Teiler von $a$ und $b$ $m$ teilt, also kann es keinen Teiler von $a$ und $b$ geben, welcher größer als $m$ ist.
\end{proof}
\noindent Die Existenz der Darstellung $\ggT(a,b) = ra + sb$ ist auch als das \tbf{Lemma von Bézout} oder die \tbf{Bézoutsche Identät} bekannt.
\begin{lemma}
	\theoremname{Lemma von Euklid:} Sei $p$ eine Primzahl und $a,b \in \bZ$.
	Dann gilt:
	\begin{align*}
		p \mid ab \implies (p \mid a) \vee (p \mid b)
	\end{align*}
\end{lemma}
\begin{proof}
	Es reicht zu Zeigen:
	\begin{align*}
		(p \nmid a) \wedge (p \mid ab) \implies p \mid b
	\end{align*}
	Aus $p \nmid a$ folgt $\ggT(p,a) = 1$. Nach dem Lemma von Bézout können wir also $1$ darstellen als:
	\begin{align*}
		1 = rp + sa
	\end{align*}
	also:
	\begin{align*}
		b = rpb + sab
	\end{align*}
	Es gilt trivial $p \mid prb$, außerdem gilt per Annahme $p \mid ab$. Es folgt $p \mid rpb + sab = b$.
\end{proof}
\begin{theorem}
	\theoremname{Eindeutigkeit der Primfaktorzerlegung im Ring $\bZ$:}
	Sei $n \in \bN_{\geq 1}$ und
	\begin{align*}
		n = \prod_{i = 1}^r p_i = \prod_{i = 1}^s q_i
	\end{align*}
	wobei alle $q_i$ und $r_i$ Primzahlen sind. So gilt $r = s$ und es gilt eine Permutation $\sigma \in S_r$ mit $p_i = q_{\sigma(i)}$.
\end{theorem}
\noindent Äquivalente Formulierungen:
\begin{itemize}
	\item Falls die $p_i$ und $q_i$ Aufsteigend oder Absteigend sortiert sind, gilt $\forall i : p_i = q_i$
	\item Es existiert eine Bijektion zwischen endlichen Multimengen von Primzahlen und $\bN_{\geq 1}$.
\end{itemize} 
\begin{proof}
	Per Induktion folgt aus dem Lemma von Euklid schnell:
	\begin{align*}
		p_1 \mid p_1 \implies \bigvee_{i = 1}^s p_1 \mid q_i
	\end{align*}
	Also existiert ein $q_i$ mit $p_1 = q_i$. Teilen wir nun beide Seiten durch $p_1$, folgt die Aussage durch die Induktionsannahme.
\end{proof}
\begin{definition}
	Euklidischer Algorithmus
\end{definition}
\chapter{Gruppen}
\begin{definition}
	\phantom{}
	\begin{enumerate}
		\item Eine Menge $M$ mit Verknüpfung $\top : M \times M \to M$ heißt \tbf{Magma}.
		\item Ein Magma mit neutralem Element heißt \tbf{unitäres Magma}.
		\item Ein unitäres Magma mit assoziativer Verknüpfung heißt \tbf{Monoid}.
		\item Ein Monoid mit inversen Elementen heißt \tbf{Gruppe}.
		\item Eine Gruppe mit kommutativer Gruppe heißt \tbf{kommutative Gruppe} oder \tbf{abelsche Gruppe}.
	\end{enumerate}
\end{definition}
\begin{example}
	\phantom{}
	\begin{enumerate}
		\item $(\bZ, -)$ ist ein nichtunitäres Magma.
		\item ?
		\item $(\bN, +)$ und $(\bN, \cdot)$ sind Monoide.
		\item $(\bZ, +)$ ist eine abelsche Gruppe.
	\end{enumerate}
\end{example}
\begin{theorem}
	Das neutrale Element einer Gruppe ist eindeutig.
\end{theorem}
\begin{itemize}
	\item Es gibt keine Gruppe mit $0$ Elementen, da die Leere Menge kein neutrales Element hat.
	\item Es gibt genau eine Gruppe mit einem Element, die Gruppe $\{e\}$.
\end{itemize}
\begin{theorem}
	Es gibt genau eine Gruppe mit zwei Elementen, nämlich die Gruppe $\{e,a\}$, mit Verknüpfungstabelle:
	\begin{align*}
		\begin{array}{c|c|c}
			&e&a\\\hline
			e&e&a\\\hline
			e&a&e\\
		\end{array}
	\end{align*}
\end{theorem}
\begin{proof}
	Durch die Definition von $e$ sind drei Teile der Tabelle schon eindeutig festgelegt. $a \cdot a = e$, da sonst $a$ kein Inverses hätte.
\end{proof}
\begin{theorem}
	Gilt für ein Gruppenelement $a \in G$ $a^2 = a$, so gilt $a = e$.
\end{theorem}
\begin{proof}
	\begin{align*}
		a = ae = aaa^{-1} = aa^{-1} = e
	\end{align*}
\end{proof}
\begin{theorem}
	Aus $a \cdot x = a \cdot y$ folgt $x = y$.
\end{theorem}
\begin{proof}
	\begin{align*}
		ax &= ay\\
		a^{-1}ax &= a^{-1}ay\\
		ex &= ey\\
		x &= y
	\end{align*}
\end{proof}
Es folgt, dass in einer Gruppentabelle in jeder Spalte und jeder Zeile kein Element doppelt vorkommen kann.
\begin{corollary}
	Es gibt nur eine Gruppe mit drei Elementen, nämlich:
	\begin{align*}
		\begin{array}{c|c|c|c}
			&e&a&b\\\hline
			e&e&a&b\\\hline
			a&a&b&e\\\hline
			b&b&e&a\\
		\end{array}
	\end{align*}
\end{corollary}
\begin{theorem}
	Es gibt zwei Gruppen mit vier Elementen, nämlich:
	\begin{align*}
		\begin{array}{c|c|c|c|c}
			 &e&a&b&c\\\hline
			e&e&a&b&c\\\hline
			a&a&e&c&b\\\hline
			b&b&c&a&e\\\hline
			c&c&b&e&a
		\end{array} &\simeq (\bZ / 2\bZ) \times (\bZ / 2\bZ)\\
		\begin{array}{c|c|c|c|c}
			&e&a&b&c\\\hline
			e&e&a&b&c\\\hline
			a&a&b&c&e\\\hline
			b&b&c&e&a\\\hline
			c&c&e&a&b
		\end{array} &\simeq \bZ / 4\bZ
	\end{align*}
\end{theorem}
\begin{definition}
	Ein Gruppenhomomorphismus ist eine Abbildung
	\begin{align*}
		\phi : (G, \top) \to (H, \bot)
	\end{align*}
	mit
	\begin{align*}
		\phi(a \top b) = \phi(a) \bot \phi(b)
	\end{align*}
	Wir schreiben die Menge der Gruppenhomomorphismen von $G$ nach $H$ als $\Grp(G,H)$.
\end{definition}
\begin{theorem}
	$\Grp((\bZ, +), G) \simeq G$
\end{theorem}
\begin{proof}
	Ordne jedem Gruppenelement $g$ die Abbildung $\phi(n) : g \to g^n$ zu.
\end{proof}
\begin{theorem}
	Sind $G$ und $H$ Gruppen, so gilt $\Mag(G,H) = \Mon(G,H) = \Grp(G,H)$.
\end{theorem}
\begin{definition}
	Eine Teilmenge $H \subseteq G$ einer Gruppe $G$ heißt \tbf{Untergruppe}, wenn sie so mit einer Gruppenstruktur versehen werden kann, dass die Einbettungsabbildung $i : H \to G$ ein Gruppenhomomorphismus ist.
\end{definition}
\noindent Dies ist eine sehr allgemeine Definition, die analog für Untervektorräume, Untermagmas etc. funktioniert. In der Praxis verwendet man meistens folgendes Kriterium:
\begin{theorem}
	$H$ ist genau dann eine Untergruppe, wenn $e \in H$ und wenn für jedes $a,b \in H$ auch $ab \in H$ und $a^{-1}, b^{-1} \in H$.
\end{theorem}
\begin{theorem}
	\theoremname{Satz von Lagrange:} Sei $G$ eine endliche Gruppe und $H$ eine Untergruppe. So ist die Zahl der Elemente von $H$ ein Teiler der Zahl der Elemente von $G$.
\end{theorem}
\noindent Der Beweis folgt nach einigen weitern Definitionen und Sätzen.
\begin{definition}
	Sei $G$ eine Gruppe mit Untergruppe $H$. Wir definieren:
	\begin{align*}
		gH &= \{gh \mid h \in H\} \subseteq G\\
		G / H &= \{gH \mid g \in G\} \subseteq \cP(G)
	\end{align*}
\end{definition}
\begin{lemma}
	Sei $h \in H$. Dann gilt $hH = H$.
\end{lemma}
\begin{lemma}
	Je zwei Nebenklassen $gH$ und $g'H$ sind entweder gleich oder disjunkt.
\end{lemma}
\noindent Aus diesen beiden Lemmas folgt $\abs{G} = \abs{G / H} \cdot \abs{H}$, was den Satz von Lagrange impliziert.
\begin{theorem}
	Der Schnitt beliebig vieler Untergruppen bildet eine Untergruppe.
\end{theorem}
\begin{definition}
	Sei $G$ eine Gruppe und $T \subseteq G$ eine Teilmenge. Die \tbf{durch $T$ erzeugte Untergruppe} $\angles{T}$ ist die kleinste Untergruppe, welche $T$ enthält.
\end{definition}
\noindent $\angles{T}$ besteht aus allen Elementen von $G$, welche durch beliebig häufige Anwendung von Inversionen und Gruppenoperationen entstehen kann, wobei wir die "leere Gruppenoperation" als $e$ definieren (für den Fall $\angles{\emptyset} = \{e\}$.)
\begin{theorem}
	Sei $\varphi : G \to H$ ein Gruppenhomomorphismus. Sei $A$ eine Untergruppe von $G$ und $B$ eine Untergruppe von $H$. So ist $\phi(A)$ eine Untergruppe von $H$ und $\phi^{-1}(B)$ eine Untergruppe von $G$.
\end{theorem}
\begin{theorem}
	\theoremname{Universelle Eigenschaft surjektiver Gruppenhomomorphismen:} Sei $\phi : G \surj H$ ein surjektiver Gruppenhomomorphismus. Sei $\psi : G \to K$ ein Gruppenhomomorphismus, sodass $\ker \psi \supseteq \ker \phi$. So existiert genau ein Gruppenhomomorphismus $\ol{\psi}$, sodass:
	\begin{align*}
		\psi = \ol{\psi} \circ \phi
	\end{align*}
	Also sodass das folgende Diagramm kommutiert:
	\[\begin{tikzcd}
		G & H \\
		&\\
		{} & K
		\arrow["\phi", two heads, from=1-1, to=1-2]
		\arrow["\psi"', from=1-1, to=3-2]
		\arrow["{\exists!\overline{\psi}}", dashed, from=1-2, to=3-2]
	\end{tikzcd}\]
\end{theorem}
\begin{proof}
	Sei $\phi(x) = \phi(y)$, also $\phi^{-1}(\phi(x)) = \phi^{-1}(\phi(y))$. Dann gilt 
	\begin{align*}
		\phi(xy^{-1}) = \phi(1) = 1,
	\end{align*} 
	also $xy^{-1} \in \ker \phi$, also $y \in x \cdot \ker \phi$. 
	Somit gilt \begin{align*}
		\phi^{-1}(\phi(g)) = g \cdot \ker \phi.
	\end{align*} 
	Es gilt außerdem
	\begin{align*}
		\psi(g \cdot \ker \phi) = \psi(g) \cdot \psi(\ker(\phi)) = \phi(g),
	\end{align*}
 	also ist $\psi$ konstant auf den Fasern (also den Urbildern einzelner Elemente) von $\phi$. Somit existiert ein $\ol{\psi}$ Aufgrund der "universellen Eigenschaft surjektiver Funktionen", welche noch nicht eingeführt wurde :), folgt die Existenz eines $\ol{\psi}$ mit den gefragten Eigenschaften.
 	\newpar
 	Zu zeigen ist noch, dass $\ol{\psi}$ ein Gruppenhomomorphismus ist.\\
 	Wähle hierfür $g \in G, g' \in G$ und $\phi(g) = h$ und $\phi(g') = h'$. Dann gilt
 	\begin{align*}
 		\ol{\psi}(hh') = \ol{\psi}(\phi(gg')) = \psi(gg') = \psi(g) \cdot \psi(g') = \ol{\psi}(h) \cdot \ol{\psi}(h')
 	\end{align*}
\end{proof}
% https://q.uiver.app/#q=WzAsNCxbMSwwLCJHIl0sWzAsMiwiSCJdLFsxLDIsIksiXSxbMiwyLCJIIl0sWzAsMSwiXFxwaGkiXSxbMCwyLCJcXHBzaSIsMl0sWzAsMywiXFxwaGkiLDJdLFsxLDIsIlxcZXhpc3RzIVxcb3ZlcmxpbmV7XFxwc2l9Il0sWzIsMywiXFxleGlzdHMhXFxvdmVybGluZXtcXHBoaX0iXSxbMSwzLCJcXGlkX0giLDEseyJjdXJ2ZSI6Mn1dXQ
\[\begin{tikzcd}
	& G \\
	\\
	H & K & H
	\arrow["\phi", from=1-2, to=3-1]
	\arrow["\psi"', from=1-2, to=3-2]
	\arrow["\phi"', from=1-2, to=3-3]
	\arrow["{\exists!\overline{\psi}}", dashed, from=3-1, to=3-2]
	\arrow["{\exists!\overline{\phi}}", dashed, from=3-2, to=3-3]
\end{tikzcd}\]
Die "Moral" dieses Satzes: Ein surjektiver Gruppenhomomorphismus $\phi : G \surj H$ wird durch $(G, \ker \phi)$ "im wesentlichen eindeutig" festgelegt.
\newpar
Die Frage ist nun, ob jede Untergruppe einer Gruppe $G$ als Kern eines surjektiven Gruppenhomomorphismus festgelegt ist. Die Antwort auf diese Frage: Nein! Die einzigen Untergruppen, welche als solche Kerne auftreten, sind sogenannte \tbf{Normalteiler}.
\begin{definition}
	Sei $G$ eine Gruppe. Eine Untergruppe $N \subseteq G$ heißt \tbf{Normalteiler}, falls
	\begin{align*}
		\forall g \in G : gN = Ng
	\end{align*}
\end{definition}
\begin{example}
	Wir betrachten die Gruppe $Q \subset GL(2,\bR)$ aller linearen Abbildungen, welche ein Quadrat auf sich selbst abbilden. Prof. Soergel nennt diese Gruppe auch gerne die "Bierdeckelgruppe $B$". Die Gruppe hat acht Elemente:
	\begin{itemize}
		\item Die Rotationen $d_0$, $d_1$, $d_2$ und $d_3$ um Vielfache von $90$°,
		\item die Spiegelungen $s_{x}$ und $s_{y}$ an den Koordinatenachsen,
		\item und die Spiegelungen $s_+$ und $s_-$ an den Diagonalachsen.
	\end{itemize}
	Diese Gruppe ist nicht kommutativ! Zum Beispiel ist $d_1s_+ = s_y$, aber $s_+ d_1 = s_x$ (Wir notieren die Elemente als Abbildungen, also wird links zuerst $s_+$ und dann $d_1$ angewandt.
	\newpar
	Wir betrachten nun die Untergruppe $H = \angles{s_+}$. So gilt $d_1H = \{s_y, d_1\}$, aber $Hd_1 = \{s_x, d_1\}$
\end{example}
\begin{theorem}
	Der Kern eines Gruppenhomomorphismus ist immer ein Normalteiler.
\end{theorem}
\begin{proof}
	Sei $\phi : G \to H$. So ist $(\ker \phi) x = \phi^{-1}(\phi(x)) = x (\ker \phi)$.
\end{proof}
\begin{theorem}
	Für jeden Normalteiler $N \subseteq G$ gibt es einen surjektiven Gruppenhomomorphismus $\phi : G \surj H$ mit $\ker \phi = N$.
\end{theorem}
\begin{proof}
	Wir wählen $H := G / N = \{xN : x \in G\}$. Für je zwei Teilmengen $A, B \subseteq G$ definieren wir $AB = \{ab : a \in A, b \in B\}$ und erhalten eine assoziative Verknüpfung auf $\cP(G)$. Für einen Normalteiler $N \subseteq G$ ist $G / N$ unter dieser Verknüpfung stabil (geschlossen), denn durch Abuse of Notation folgt $xNyN = xyNN = xyN$. Unter dieser Verknüpfung wird $G/N$ eine Gruppe mit neutralem Element $1N = N$ und $x^{-1}N = (xN)^{-1}$.
	\newpar
	Dadurch wird dann $\phi : G \to G/N$, $x \to xN$ ein Gruppenhomomorphismus mit $\ker \phi = N$.
\end{proof}
\begin{example}
	Die Menge $D \subsetneq Q$ der Drehungen eines Quadrats ist ein Normalteiler der vollen Symmetriegruppe $Q$.
\end{example}
\begin{proof}
	Sei $s$ eine Beliebige Spiegelung. So gilt $Ds = \{s_+, s_-, s_x, s_y\} = sD$. Einzelne Spiegelungen kommutieren also nicht mit einzelnen Gruppenelementen, es ist jedoch trotzdem eine Art "Kommutativität mit der Gruppe als Ganzes" vorhanden.
\end{proof}
\noindent Die Menge der Drehungen ist desweiteren gegeben als der Kern der Determinante, welche ein Gruppenhomomorphismus in die Gruppe mit zwei Elementen ist (genauer in die Gruppe $\{\pm 1, \cdot\}$).
\begin{theorem}
	Sei $\phi : G \to H$ ein Gruppenhomomorphismus. So induziert $\phi$ einen Isomorphismus $G / \ker \phi \to \im \phi$.
\end{theorem}
\noindent Sei $G$ eine Gruppe mit $\abs{G} = 5$. Gemäß Lagrange hat $G$ nur die triviale Gruppe und sich selbst als Untergruppe. Sei $g \in G \setminus \{1\}$. So haben wir einen Gruppenhomomorphismus
\begin{align*}
	\phi : (\bZ, +) &\to G\\
	         n &\mapsto g^n
\end{align*}
Wir wissen nun, das das Bild von $\phi$ eine Untergruppe von $G$ bildet. Da $g \neq 1$ handelt es sich nicht um die triviale Gruppe, also ist $G = \im \phi$.
\newpar
$\ker \phi \subsetneq \bZ$ ist eine Untergruppe, also existiert genau ein $n \in \bN$, sodass $\ker \phi = \bZ n$. Da $G$ $5$ Elemente hat, muss $n = 5$. Die einzige Gruppe mit $5$ Elementen ist also $\bZ / 5\bZ$.
\begin{corollary}
	Ist $p$ eine Primzahl, so ist jede Gruppe mit $p$ Elementen isomorph zu $\bZ/ p\bZ$.
\end{corollary}
\begin{lemma}
	Sei $\pi : G \to H$ ein surjektiver Gruppenhomomorphismus und $N \subseteq G$ ein Normalteiler. So ist $\pi(N)$ ein Normalteiler von $H$.
\end{lemma}
\begin{theorem}
	\theoremname{Noetherscher Isomorphiesatz:} Sei $G$ eine Gruppe und seien $H$ und $K$ Normalteiler. So ist $H / K$ ein Normalteiler von $G / K$ und die Komposition kanonischer Abbildungen $G \surj G / K \surj (G / K) / (H / K)$ induziert einen Isomorphismus $G / H \to (G / K) / (H / K)$
\end{theorem}
\begin{definition}
	Sei $G$ eine Gruppe und $g \in G$. Wir definieren die Ordnung eines Elements als:
	\begin{align*}
		\ord(g) = \inf\{n \in \bN_{\geq 1} \mid g^n = e\}
	\end{align*}
	Erinnerung: Das Infimum der leeren Menge ist $\infty$, also gilt $\ord(g) = \infty$ falls kein solches $n$ existiert.
\end{definition}
\begin{proposition}
	Sei $G$ eine Gruppe und $g \in G$.
	\begin{enumerate}
		\item $\ord(g) = \abs{\angles{g}}$
		\item $\angles{g}$ ist isomorph zu $\bZ / (\ord(g))\bZ$, falls die Ordnung endlich ist, und zu $\bZ$, falls die Ordnung unendlich ist.
		\item Falls $\ord(g) < \infty$, so gilt $g^n = e \Leftrightarrow \ord(g) \mid n$
	\end{enumerate}
\end{proposition}
\begin{definition}
	Sei $G$ eine Gruppe. So heißt $\abs{G}$ die \tbf{Ordnung} der Gruppe.
\end{definition}
\begin{theorem}
	Sei $g \in G$ ein Element einer endlichen Gruppe. So ist $\ord(g)$ ein Teiler von $\abs{G}$.
\end{theorem}
\begin{proof}
	$\ord(g) = \abs{\angles{g}}$. Nach Lagrange teilt die Größe der Untergruppe $\angles{g}$ die Größe der Gruppe $G$.
\end{proof}
\noindent Im Allgemeinen gibt es aber nicht für jeden Teiler ein Element mit der jeweiligen Ordnung. Zum Beispiel fehlt bei jeder nicht zyklischen Gruppe bereits die Ordnung selbst. Wohl aber gilt:
\begin{theorem}
	\theoremname{Cauchy:} Sei $p$ eine Primzahl mit $p \mid \abs{G}$. So gibt es ein $g \in G$ mit $\ord(g) = p$.
\end{theorem}
\noindent Ein Beweis folgt aber erst später.
\begin{corollary}
	Sei $g \in G$ ein Element einer endlichen Gruppe. So gilt 
	\begin{align*}
		g^\abs{G} = e.
	\end{align*}
\end{corollary}
\begin{proof}
	$g^\abs{G} = g^{r \cdot \abs{\ord(g)}} = e^r = e.$
\end{proof}
\begin{theorem}
	\theoremname{Kleiner Satz von Fermat:} Sei $p$ eine Primzahl und $a \in \bZ$. So gilt
	\begin{align*}
		a^p \equiv a \mod p
	\end{align*}
\end{theorem}
\begin{proof}
	Wir betrachten $\bZ/p\bZ$. Falls $\alpha \equiv 0 \mod p$ ist die Aussage trivial. Da außerdem die Multiplikative Gruppe die Ordnung $p-1$ hat gilt im Fall $\alpha \not\equiv 0$ ebenfalls trivial $\alpha^{p-1} \equiv 1 \mod p$, also $\alpha^p \equiv \alpha \mod p$.
\end{proof}
\noindent Alternativ kann der Beweis auch auf "Allgemeinbildungsniveau" geführt werden:
\begin{proof}
	(Ohne Gruppentheorie): Sei $a \in \{0, \hdots, p-1\}$. So sind die Zahlen $a, 2a, 3a, \hdots, (p-1)a$ paarweise verschieden, also ist nach Schubfachprinzip ein Element jeder Restklasse $\mod p$ enthalten. Also gilt:
	\begin{align*}
		\prod_{k = 1}^{p-1} ka &\equiv \prod_{k = 1}^{p-1} k \mod p\\
		\implies (p-1)!a^{p-1} &\equiv (p-1)! \mod p.
	\end{align*}
	Allerdings ist $(p-1)!$ teilerfremd zu $p$, also können wir durch $(p-1)!$ teilen und erhalten $a^{p-1} \equiv 1 \mod p$.
\end{proof}
\begin{theorem}
	\label{thrm:sunziprereq}
	Seien $a,b \in \bN_{\geq 1}$ teilerfremd. So induziert
	\begin{align*}
		\phi : \bZ &\to \bZ / a\bZ \times \bZ / b\bZ\\
		n &\mapsto (n + a\bZ, n + b\bZ)
	\end{align*}
	einen Isomorphismus:
	\begin{align*}
		\bZ / ab\bZ &\overset{\sim}{\rightarrow} \bZ / a\bZ \times \bZ / b\bZ\\
	\end{align*}
\end{theorem}
\begin{proof}
	Nach Bézout gibt es $x,y \in \bZ$ mit
	\begin{align*}
		xa + yb = 1.
	\end{align*}
	Nun gilt:
	\begin{align*}
		n &= xy = 1 - yb \overset{\phi}{\mapsto} (0,1)\\
		m &= yb = 1 - xy \overset{\phi}{\mapsto} (1,0)
	\end{align*}
	Diese Tupel bilden eine Basis des Bildraums, also ist $\phi$ surjektiv:
	\begin{align*}
		\phi(\alpha n + \beta m) &= \phi(\alpha n) + \phi(\beta m)\\
								 &= \alpha\phi(n) + \beta\phi(m)\\
								 &= (\beta, \alpha)
	\end{align*}
	Da beide Seiten gleich viele Elemente haben handelt es sich sogar um einen Isomorphismus.
\end{proof}
\begin{example}
	Das Kongruenzensystem
	\begin{align*}
		n &\equiv 3 \mod 17\\
		n &\equiv 5 \mod 9\\
		n &\equiv 1 \mod 12
	\end{align*}
	hat keine Lösung, denn aus $n \equiv 5 \mod 9$ folgt $n \equiv 2 \mod 3$, also $n \not \equiv 1 \mod 12$. Ersetzen wir jedoch $12$ durch $13$, so sind die Teiler teilerfremd und der chinesische Restsatz garantiert eine Lösung.
\end{example}
\begin{theorem}
	\theoremname{Chinesischer Restsatz / Satz von Sunzi / Satz von Aryabhata:} Satz \ref{thrm:sunziprereq} funktioniert auch für mehr als zwei Zahlen - sind $q_1, \hdots, q_r \in \bN_{\geq 1}$ teilerfremd, so ist
	\begin{align*}
		\bZ \to \bZ / q_1\bZ \times \hdots \times \bZ / q_r \bZ
	\end{align*}
	surjektiv und induziert eines Isomorphismus
	\begin{align*}
		\bZ / q_1 \hdots q_r \bZ \iso \bZ / q_1\bZ \times \hdots \times \bZ / q_r\bZ
	\end{align*}
\end{theorem}
\noindent Genauer kann man isomorph jedem Element $n$ das Tupel $(n \mod q_1, \hdots, n \mod q_r)$ zuordnen - so kommt man zurück zur "klassischen" Formulierung des Chinesischen Restsatzes.
\begin{corollary}
	Die Gruppe $\bZ / q_1\bZ \times \hdots \times \bZ / q_r\bZ$ wird durch das Element $(1, \hdots, 1)$ erzeugt und ist somit zyklisch.
\end{corollary}
\begin{lemma}
	\theoremname{Zentrales Lemma der RSA-Verschlüsselung:} Seien $p$ und $q$ prim und $st \equiv 1 \mod (p-1)(q-1)$. So ist $(a^t)^s \equiv \mod a \mod pq$.
\end{lemma}
\begin{proof}
	Wir wissen $a^x = a \mod p$ falls $x \equiv 1 \mod (p-1)$. 
	\newpar
	Also $a^x = a \mod pq$, wenn $x \equiv 1 \mod (p-1)$ und $x \equiv 1 \mod (q-1)$, also insbesondere $a^x = a \mod p$ wenn $x \equiv 1 \mod (p-1)(q-1)$.
\end{proof}
\noindent In der RSA-Verschlüsselung werden dann letztendlich $pq$ und $t$ veröffentlicht, $a^t$ als Nachricht zurückgeschickt, und dann durch Exponentiation mit $s$ die Nachricht entschlüsselt.
\begin{definition}
	Sei $A$ eine abelsche Gruppe und $p$ eine Primzahl. Wir definieren:
	\begin{align*}
		A(p) = \{a \in A \mid \exists n \in \bN : \ord(a) = p^n\}
	\end{align*}
\end{definition}
\begin{theorem}
	Sei $(A, +)$ eine abelsche Gruppe. Für jede Primzahl $p$ ist $A(p) \subseteq A$ eine Untergruppe.
\end{theorem}
\begin{proof}
	Seien $x,y \in A(p)$. So existieren $r,s \in \bN_{\geq 0}$, sodass $p^rx = p^sy = 0$. Da $A$ abelsch ist folgt
	\begin{align*}
		p^{r+s}(x+y) = p^rx + p^sy = 0
	\end{align*}
	Also $x + y \in A(p)$.
	\newpar
	Es gilt außerdem $p0 = 0$, also $0 \in A(p)$, und $p^r(-x) = -(p^rx) = -0 = 0$, also $-x \in A(p)$.
\end{proof}
\begin{theorem}
	Sei $(A, +)$ eine abelsche Gruppe und seien $p_1, \hdots, p_r$ paarweise verschiedene Primzahlen. So liefert die Gruppenverknüpfung eine Injektion
	\begin{align*}
		A(p_1) \times \hdots \times A(p_r) \inj A
	\end{align*}
	mit Bild alle Elemente endlicher Ordnung, in deren Ordnung nur die Primfaktoren $p_1, \hdots, p_r$ vorkommen.
\end{theorem}
\begin{proof}
	Sei sonst $(a_1, \hdots, a_r) \neq (0, \hdots, 0)$ ein Tupel mit $a_i \in A(p_i)$ und 
	\begin{align*}
		a_1 + \hdots + a_r = 0.
	\end{align*} Sei OBdA $a_1 \neq 0$. So gilt:
	\begin{align*}
		-a_1 = a_2 + \hdots + a_r
	\end{align*}
	Für hinreichend großes $n$ gilt nun
	\begin{align*}
		-(p_2\hdots p_r)^na_1 = (p_2\hdots p_r)^n(a_2 + \hdots + a_r) = 0
	\end{align*}
	Allerdings gilt
	\begin{align*}
		(p_2\hdots p_r)^na_1 \neq 0
	\end{align*}
	Da die Primzahlen $p_2, \hdots, p_r$ alle die Ordnung von $a_1$ nicht Teilen. Widerspruch!
	\newpar
	Sei nun $a \in A$, $\ord(a) = p_1^{\alpha_1}\hdots p_r^{\alpha_r}$. So gilt:
	\begin{align*}
		\angles{a} &\cong  \bZ / \ord(a)\bZ\\
				   &= \bZ / p_1^{\alpha_1}\hdots p_r^{\alpha_r}\bZ\\
				   &\cong \bZ / p_1^{\alpha_1}\bZ \times \hdots \times \bZ / p_r^{\alpha_r}\bZ
	\end{align*}
\end{proof}
\noindent Dieser Satz ist analog zur Hauptraumzerlegung eines Vektorraums - wir zerlegen $A$ in eine direkte Summe aus Untergruppen. Es gibt in der kommutativen Algebra einen Satz über Module über Ringe, welcher sowohl diesen Satz als auch die Hauptraumzerlegung gemeinsam verallgemeinert.
\begin{corollary}
	Sei $E$ eine endliche abelsche Gruppe. Seien $p_1, \hdots, p_r$ die Primfaktoren der Kardinalität $\abs{E}$ von $E$. So gilt:
	\begin{enumerate}
		\item Die Verknüpfung liefert eine Bijektion
		\begin{align*}
			E(p_1) \times \hdots \times E(p_r) \iso E
		\end{align*}
		\item Für alle $i$ ist $\abs{E(p_i)}$ eine Potenz von $p_i$. 
	\end{enumerate}
\end{corollary}
\begin{proof}
	\phantom{}
	\begin{enumerate}
		\item Folgt direkt aus dem vorherigen Satz, da die Elemente endlicher Ordnung, in deren Ordnung nur die Primfaktoren $p_1, \hdots, p_r$ vorkommen, nun die ganze Gruppe ausmachen.
		\item Folgt aus Induktion über $\abs{E_p}$. Der Fall $\abs{E_p} = 1$ ist klar:
		\begin{align*}
			\abs{E_p} = 1 = p^0
		\end{align*}
		Für $\abs{E_p} > 1$ existiert ein Element $x \neq 0$. Es gilt $\abs{\angles{x}} = p^r$ nach Annahme. Dann folgt
		\begin{align*}
			\abs{E(p) / \angles{x}} = p^s
		\end{align*}
		nach Induktionsannahme, da der Quotient kleiner ist. Somit folgt nach Lagrange
		\begin{align*}
			\abs{E_p} = \abs{E(p) / \angles{x}} \cdot \abs{\angles{x}} = p^s \cdot p^r
		\end{align*}
	\end{enumerate}
\end{proof}
\begin{proposition}
	Sei $E$ eine endliche abelsche Gruppe, sodass die Ordnung jedes $x \in E$ eine Potenz von $p$ prim ist. So ist $\abs{E}$ ebenfalls eine $p$-Potenz.
\end{proposition}
\begin{proposition}
	Es existiert eine Bijektion zwischen der Menge der endlichen abelschen Gruppen bis auf Isomorphismus und den endlichen Multimengen echter Primpotenzen. Genauer hat jede endliche abelsche Gruppe die Form
	\begin{align*}
		\bZ / p_1\bZ \times \hdots \times \bZ / p_r \bZ
	\end{align*}
	Für $p_1, \hdots, p_r$ prim.
\end{proposition}
\noindent Zum Beispiel ist jede abelsche Gruppe der Ordnung $8$ isomorph zu $\bZ / 8\bZ$, $\bZ / 4\bZ \times \bZ / 2\bZ$, oder $\bZ / 2\bZ \times \bZ / 2\bZ \times \bZ / 2\bZ$.
\begin{theorem}
	Sei $K$ ein Körper und $K^\times$ dessen multiplikative Gruppe. So ist jede endliche Untergruppe von $K^\times$ zyklisch.
\end{theorem}
\begin{definition}
	Sei $G$ eine Gruppe und $X$ eine Menge. Eine Operation von $G$ auf $X$ ist eine Abbildung
	\begin{align*}
		G \times G &\to X\\
		(g,x) &\mapsto gx
	\end{align*}
	Sodass: 
	\begin{enumerate}
		\item $\forall g,h \in G : (gh)x = g(hx)$
		\item $1x = x$
	\end{enumerate}
\end{definition}
\begin{example}
	Sei $X$ eine Menge und
	\begin{align*}
		\Ens^\times(X) := \{f : X \to X \mid f \text{ bijektiv}\}
	\end{align*}
	So ist $\Ens^\times$ eine Gruppe, welche durch Anwendung der jeweiligen Funktion auf kanonische Weise auf $X$ operiert:
	\begin{align*}
		\Ens^\times(X) \times X &\to X\\
		(f, x) &\mapsto f(x)
	\end{align*}
\end{example}
\begin{example}
	$\cS_n$ operiert durch Anwendung der jeweiligen Permutation auf $\{1, \hdots, n\}$.
\end{example}
\begin{example}
	Analog operiert $\GL(n,\bR)$ auf $\bR^n$.
\end{example}
\begin{definition}
	Sei $G$ eine Gruppe, welche auf eine Menge $X$ operiert. So nennen wir für $x \in X$ die Menge
	\begin{align*}
		Gx = \{gx \mid g \in G\}
	\end{align*}
	die \tbf{Bahn} von $x$.
\end{definition}
\begin{example}
	Wir lassen die Gruppe $\SO(2)$ der zweidimensionalen Rotationsmatrizen auf $\bR^2$ operieren. So ist die Bahn eines Vektors $\vv$ der Kreis mit Radius $\norm{\vv}$.
\end{example}
\begin{example}
	Sei $H$ eine Untergruppe einer Gruppe $G$. So operiert $H$ durch die Gruppenoperation auf $G$ und die Bahnen $Hg$ sind genau die Nebenklassen.
\end{example}
\begin{proposition}
	Die Bahnen der Operation einer Gruppe auf einer Menge sind paarweise disjunkt.
\end{proposition}
\begin{proposition}
	Die Bahnen der Operation eines Monoids auf einer Menge sind nicht umbedingt paarweise disjunkt.
\end{proposition}
\begin{definition}
	Sei $G$ eine Gruppe, welche auf einer Menge $X$ operiert. Sei $x \in X$. Wir nennen die Menge
	\begin{align*}
		G_x := \{g \in G \mid gx = x\}
	\end{align*}
	die \tbf{Standgruppe} oder \tbf{Isotropiegruppe} von $x$.
\end{definition}
\begin{lemma}
	Die Abbildung $G \to X : g \to gx$ induziert eine Bijektion
	\begin{align*}
		G / G_x \iso Gx
	\end{align*} 
\end{lemma}
\begin{proofsketch}
	Für jede $G_x$-Linksnebenklasse $L \subset G$ besteht die Menge $Lx$ nur aus einem Punkt. Falls $L = gG_x$ haben wir genauer $Lx = gG_xx = \{gx\}$. Wir können nun unsere Bijektion definieren, indem wir jeder Linksnebenklasse $L \in G/G_x$ das einzige Element von $Lx$ zuordnen. Sie ist trivial surjektiv, gleichzeitig ist sie injektiv, denn aus $gG_x x = hG_x x$ folgt $gx = hx$, also $h^{-1}g \in G_x$, also $gG_x = hG_x$.
\end{proofsketch}
\begin{corollary}
	\theoremname{Bahnformel:}
	\begin{align*}
		\abs{G} = \abs{Gx} \cdot \abs{G_x} = \abs{G / G_x} \cdot \abs{G_x}
	\end{align*}
\end{corollary}
	\noindent Frage: Was sind die endlichen Untergruppen der dreidimensionalen Drehgruppe $\SO_3$? 
	\newpar
	Vorüberlegung: Jede Gruppe $G$ mit $x \in G$ operiert auf sich selber durch $g.x = gxg^{-1}$, auch notiert als $(\textnormal{int} g)(x)$ für "interior", da dies einen sogenannten \tbf{inneren Automorphismus} bildet. Es gilt nämlich:
	\begin{align*}
		\textnormal{int} : G &\to \Aut(G)\\
		g &\mapsto (x \mapsto gxg^{-1})
	\end{align*}
	
	\noindent Wir nennen diese Operation die "\textbf{Operation durch Konjugation}".
	\newpar
	Dabei gilt:
	\begin{align*}
		(hg).x &= hgx(hg)^{-1}\\
			   &= hgxg^{-1}h^{-1}\\
			   &= h(gxg^{-1})h^{-1}\\ 
			   &= h.(g.x)
	\end{align*}
	und außerdem:
	\begin{align*}
		g.(xy) = gxyg^{-1} = gxg^{-1}gyg^{-1} = (g.x)(g.y)
	\end{align*}
	Intuitiv sind Drehungen, welche zueinander konjugiert werden können, "quasi die gleichen" - die Symmetriegruppe, welche den Raum an einer Achse um $180$ Grad dreht, ist die gleiche, welche den Raum an einer anderen Achse um $180$ Grad dreht. Es ist also sinnvoll, die Frage einzuschränken auf "Was sind die endlichen Untergruppen von $\SO_3$ \tit{bis auf Konjugation}? Die Antwort ist nun:
	\begin{theorem}
		Die endlichen Untergruppen der Drehgruppe $\SO_3$ bis auf Konjugation sind genau:
		\begin{enumerate}
			\item Die zyklischen Gruppen $C_k$, für $k \geq 1$
			\item Die \tbf{Diedergruppen} $D_k$, für $k \geq 2$ (die Verallgemeinerungen unserer Quadratgruppe)
			\item Die Symmetriegruppen des Würfels (= die Symmetriegruppe des Oktaeders), des Tetraeders, und des Ikosaeders (= die Symmetriegruppe des Dodekaeders)
		\end{enumerate}
	\end{theorem}
	\begin{proof}
		\begin{proposition}
			$g \in \SO_3 \setminus \id$ hält genau eine Gerade punktweise fest. Diese sticht an zwei Punkten durch die Einheitssphäre $S^2$.
		\end{proposition}
		\noindent Wir nennen diese Punkte die Pole $P \subset S^2$ von $g$. Sei also $G \subset \SO(3)$ eine endliche Untergruppe. Wir definieren
		\begin{align*}
			M \in \{(g,p) \mid g \in G \setminus \id, p \in P\}
		\end{align*}
		Als die Menge der Pole der Drehachsen der Gruppe. Es gilt
		\begin{align*}
			\abs{M} &= 2(\abs{G} - 1)\\
			\abs{M} &= \sum_{p \in P} \left(\abs{G_p} - 1\right)\\
		\end{align*}
		Die Gruppe operiert außerdem auf der Menge der Pole ($G \searrow P$). Wir können die Menge der Pole somit Zerlegen in ihre $G$-Bahnen, also $P \to P_1 \sqcup \hdots \sqcup P_r$. Die Ordnung der Pole einer Bahn ist konstant, wir definieren also die \tbf{Polordnung} der Pole der $i$-ten Bahn als
		\begin{align*}
			\forall p \in P_i : n_i := \abs{G_p}
		\end{align*}
		Und erhalten
		\begin{align*}
			\sum_{p \in P} \left(\abs{G_p} - 1\right) &= \sum_{i = 1}^r \abs{P_i} \cdot (n_i - 1)\\
		\end{align*}
		Aus der Bahnformel erhalten wir
		\begin{align*}
			\abs{P_i} \cdot n_i = \abs{G}
		\end{align*}
		Also gilt:
		\begin{align*}
			2(\abs{G} - 1)  &=\sum_{i = 1}^r \abs{P_i} \cdot (n_i - 1)\\
			\implies 2 - \frac{2}{\abs{G}} &=\sum_{i = 1}^r 1 - \frac{1}{n_i}\\ 
		\end{align*}
		\begin{description}
			\item[Fall $r = 0$:] Es gibt keine Pole, also ist die Gruppe trivial. Die triviale Gruppe kann auch betrachtet werden als die zyklische Gruppe $C_1$ mit nur einem Element.
			\item[Fall $r = 1$:] Dann gilt:
			\begin{align*}
				2 - \frac{2}{\abs{G}} &= 1 - \frac{1}{n_1}
			\end{align*}
			Die Gruppe muss mindestens $2$ Elemente haben, also ist der linke Term größer als $1$ und der rechte Term kleiner als $1$, was ein Widerspruch ist.
			\item[Fall $r = 2$:] Dann gilt:
			\begin{align*}
				2 - \frac{2}{\abs{G}} &= 2 - \frac{1}{n_1} - \frac{1}{n_2}
			\end{align*}
			Es gilt $n_i \leq \abs{G}$, also ist die Gleichung genau dann erfüllt, wenn $n_1 = n_2 = \abs{G}$. Wir haben also eine Gruppe mit zwei Polen, deren Ordnung die Ordnung der gesamten Gruppe ist. Die ganze Gruppe hat also nur eine Symmetrieachse, die Gruppe muss dann eine zyklische Gruppe sein.
			\item[Fall $r = 3$:] Dann gilt:
			\begin{align*}
				2 - \frac{2}{\abs{G}} &= 3 - \frac{1}{n_1} - \frac{1}{n_2} - \frac{1}{n_3}\\
			\implies
				 1 + \frac{2}{\abs{G}} &= \frac{1}{n_1} + \frac{1}{n_2} + \frac{1}{n_3}
			\end{align*}
			Sei OBdA $n_1 \leq n_2 \leq n_3$. Behauptung: Dann muss $n_1 = 2$ sein.
			\begin{description}
				\item[Fall $3a$: $n_2 = 2$:] Dann gilt $2n_3 = \abs{G}$. Nach Bahnformel folgt $\abs{P_3} = 2$. Bei einer Polbahn mit zwei Elementen müssen sich die Pole gegenüberliegen. Wir erhalten die Diedergruppe $D_{n_3}$ - $n_3$ entspricht der zentralen Drehsymmetrie und $n_1$ und $n_2$ entsprechen den seitlichen Symmetrieachsen durch die Ecken und Kanten.
				\item[Fall $3a$: $n_2 > 2$:]
				\begin{align*}
					\frac{1}{2} + \frac{1}{n_2} + \frac{1}{n_3} > 1 + \frac{2}{\abs{G}}
				\end{align*}
				Also $n_2 = 3$ und $n_3 = 3$ oder $n_3 = 4$ oder $n_3 = 5$. Der erste Fall liefert den Tetraeder, der zweite Fall den Würfel und der dritte Fall den Ikosaeder.
			\end{description}
			\item[Fall $r \geq 4$:] Die Gleichungen sind nicht mehr lösbar.
		\end{description}
	\end{proof}
	\begin{definition}
		Eine Gruppe $G \neq 1$ heißt \tbf{einfach}, falls ihre einzigen Normalteiler die triviale Gruppe $1$ und die Gruppe $G$ selbst sind.
	\end{definition}
	\begin{proposition}
		Jede Gruppe mit Primzahlordnung ist einfach (also $\bZ/ p\bZ$ für $p$ prim). Diese Gruppen sind die einzigen abelschen einfachen Gruppen.
	\end{proposition}
	\begin{proof}
		Satz von Lagrange.
	\end{proof}
	\begin{proposition}
		Für fast alle Körper $k$ ist $GL(n,k) / k^\times$ einfach.
	\end{proposition}
	\begin{theorem}
		Die Ikosaedergruppe $I$ ist einfach.
	\end{theorem}
	\begin{proof}
		Wir haben:
		\begin{itemize}
			\item Ein Element der Ordnung $1$ (die Identität)
			\item $15$ Elemente der Ordnung $2$ (Drehungen um Kantenmitten, da wir $30$ Kanten haben und je zwei gegenüberliegende Kanten die selben Drehungen haben)
			\item $20$ Elemente der Ordnung $3$ (Drehungen um Flächenmitten, da wir $20$ Flächen haben)
			\item Keine Elemente der Ordnung $4$
			\item $24$ Elemente der Ordnung $5$ (Drehungen an Ecken, da wir $12$ Ecken haben, an jeder Ecke vier Drehungen, aber je zwei gegenüberliegende Ecken die gleichen Drehungen haben)
		\end{itemize}
		Wir wollen die Elemente nun in Konjugationsklassen zerlegen. Die Elemente der Ordnung $2$ und der Ordnung $3$ bilden jeweils eine Konjugationsklasse. Die Drehungen um Ecken können jedoch nicht alle in der selben Konjugationsklasse liegen, da $24$ kein Teiler von $60$ ist. Die Drehungen um Ecken zerfallen also in zwei Konjugationsklassen - es handelt sich um die Drehungen um $2\pi/5$ ($72$°, "Eine Ecke weiter"), und die Drehungen um $4\pi/5$ ($144°$, "zwei Ecken weiter"). Größere Drehungen existieren nicht, da eine Drehung drei Ecken weiter einer Drehung um zwei Ecken in die andere Richtung entspricht und eine Drehung vier Ecken weiter einer Drehung um eine Ecke in die andere Richtung entspricht.
		\newpar
		Sei also nun $N$ ein Normalteiler von $I$. So muss $\abs{N}$ ein Teiler von $\abs{I}$ sein, also $1,2,3,4,5,6,10,12,15,20,30,60$.
		\begin{proposition}
			Sei $N$ ein Normalteiler einer Gruppe $G$. So ist $N$ eine Vereinigung von Konjugationsklassen von $G$.
		\end{proposition}
		Aus der Proposition folgt, dass $\abs{I}$ außerdem eine Summe von $1,20,15,12,12$ sein muss, wobei die $1$ in der Sumume enthalten sein muss (da $N$ sonst nicht das neutrale Element enthalten würde). Die einzigen Möglichkeiten sind also $\abs{N} = 1 = 1$ und $\abs{N} = 60 = 1 + 20 + 15 + 12 + 12$. Also sind die einzigen Normalteiler von $I$ die triviale Gruppe und $I$ selbst.
	\end{proof}
	\begin{proposition}
		Die \tbf{alternierenden Gruppen}
		\begin{align*}
			A_r := \ker(\sgn : \cS_r \to \{\pm 1\})
		\end{align*} sind für $r \geq 5$ einfach. (Wobei der Kern genau den Permutationen mit $\sgn(\sigma) = 1$ entspricht). Insbesondere ist $A_5$ isomorph zur Ikosaedergruppe, und diese ist die kleinste nichtabelsche einfache Gruppe.
	\end{proposition}
	\begin{proof}
		Ein Ikosaeder hat die selbe Symmetriegruppe wie ein Dodekaeder, da die beiden Körper dual zueinander sind (also können die Ecken des einen den Flächenmitten des anderen zugeordnet werden und umgekehrt).
		\begin{proposition}
			Ein Dodekaeder enthält fünf eingeschriebene Würfel.
		\end{proposition}
		Wir erhalten einen Gruppenhomomorphismus $\phi : I \to \cS_5$, welcher beschreibt, wie eine Symmetrie des Ikosaeders/Dodekaeders die eingeschriebenen Würfel permutiert. So ist $\sgn \circ \phi$ ein Homomorphismus $I \to \{\pm 1\}$. Da $I$ einfach ist und $\abs{I} \geq 2$ ist $\ker(\sgn \circ \phi) = I$. 
		\newpar
		$\ker(\phi)$ ist ein von $I$ verschiedener Normalteiler, also ist $\ker(\phi) = 1$, also ist $\phi$ injektiv. Es gilt $\abs{I} = 60$ und $\abs{\cS_5} = 120 \implies \abs{A_5} = 60$, also ist $\phi$ auch surjektiv, also bijektiv, also ein Isomorphismus.
	\end{proof}
	\begin{definition}
		Eine \tbf{Kompositionsreihe} einer Gruppe $G$ ist eine Folge von Untergruppen
		\begin{align*}
			1 = G_0 \subset G_1 \subset \hdots \subset G_{r-1} \subset G_r = G
		\end{align*}
		sodass jedes $G_i$ Normalteiler von $G_{i+1}$ ist und alle \tbf{Subquotienten} $G_i / G_{i-1}$ einfach sind.
	\end{definition}
	\begin{theorem}
		\theoremname{Jordan-Hölder:} Je zwei Kompositionsreihen einer endlichen Gruppe $G$ haben die selbe Länge und bis auf Reihenfolge die selben Subquotienten.
	\end{theorem}
	\noindent Dieser Satz gibt uns eine "Primfaktorzerlegung" von Gruppen, in der die "Primfaktoren" die Subquotienten sind.
	\begin{example}
		Sei $G = \bZ / 20\bZ$. Eine mögliche Kompositionsreihe ist nun:
		\begin{align*}
			\bZ/20\bZ \supset 2\bZ / 20\bZ \supset 10\bZ / 20\bZ \supset 0
		\end{align*}
		Die Subquotienten hier sind dann $\bZ/2\bZ$, $\bZ/5\bZ$ und $\bZ/2\bZ$.
	\end{example}
	\begin{proposition}
		Es existiert eine universelle Eigenschaft surjektiver linearer Abbildungen, welche exakt analog zur universellen Eigenschaft surjektiver Gruppenhomomorphismen funktioniert. 
	\end{proposition}
	\noindent Eine lineare Abbildung ist also im Wesentlichen eindeutig durch ihren Kern bestimmt - alle surjektiven linearen Abbildungen mit dem selben Kern können durch Verkettung mit einem eindeutigen Isomorphismus ineinander überführt werden.
	\begin{proposition}
		Für jeden Untervektorraum $U \subset V$ gibt es eine surjektive lineare Abbildung $s : V \surj W$ mit $\ker s = U$. Dies ist genau die Abbildung $V \surj V / U$.
	\end{proposition}
	\begin{proposition}
		Für Vektorräume gilt $\dim V = \dim U + \dim V / U$. Diese Formel kann als Analogon zum Satz von Lagrange gesehen werden.
	\end{proposition}
	\begin{example}
		Betrachte einen $3$-dimensionalen Vektorraum $V$ über $\bF_7$. Eine mögliche Kompositionsreihe ist
		\begin{align*}
			V \supset W \supset U \supset \{0\}
		\end{align*}
		mit $\dim W = 2$ und $\dim U = 1$. Die Subquotienten $V/W$, $W/U$ und $U/\{0\}$ sind jeweils eindimensionale Untervektorräume von $V$, also isomorph zur additiven Gruppe von $\bF_7$.
	\end{example}
	\begin{proof}[Beweis (Jordan-Hölder):]
		Per Induktion über $\abs{G}$. Ist $\abs{G} = 1$, ist $G$ die triviale Gruppe und somit insbesondere bereits alleine die vollständige Kompositionsreihe und die Aussage gilt trivial.
		\newpar
		Seien nun $G \supset M \supset \hdots$ und $G \supset N \supset \hdots$ Kompositionsreihen. $M$ und $N$ müssen beide Ordnungen kleiner $G$ haben, falls also $M = N$ ist gilt die Aussage per Induktion. Ist $N \neq M$, existiert ein surjektiver Homomorphismus $N \surj G / M$ mit Kern $N \cap M$. Somit geht $N / N \cap M$ isomorph auf $G / M$. 
		\newpar
		Wir wissen also, dass der Subquotient $G / M$ isomorph zu $N / N \cap M$ ist. Analog ist der Subquotient $G / N$ isomorph zu $M / M \cap N$. Also sind
		$G \supset M \supset M \cap N \supset \hdots$ und $G \supset N \supset M \cap N \supset \hdots$ valide Kompositionsreihen, welche per Induktion äquivalent zu den ursprünglichen Kompositionsreihen sind. Somit folgt auch die gesuchte Aussage per Induktion.
	\end{proof}
	\begin{definition}
		Sei $p$ eine Primzahl. Eine endliche Gruppe $G$ heißt \tbf{$p$-Gruppe}, wenn
		\begin{align*}
			\abs{G} = p^n.
		\end{align*}
	\end{definition}
	\begin{definition}
		Sei $G$ eine Gruppe. Das \tbf{Zentrum} $Z$ einer Gruppe ist die Menge der Elemente, welche mit allen anderen Gruppenelementen kommutieren:
		\begin{align*}
			Z(G) = \{z \in G \mid \forall g \in G : zg = gz\}
		\end{align*}
	\end{definition}
	\begin{theorem}
		Sei $G$ eine nichttriviale $p$-Gruppe. So ist auch das Zentrum von $G$ nichttrivial.
	\end{theorem}
	\begin{proof}
		Sei $\abs{G} = p^r$. Wir zerlegen $G$ in ihre Konjugationsklassen $C_1, \hdots, C_s$. $\abs{C_i}$ teilt $\abs{G} = p^r$, also haben insbesondere alle Konjugationsklassen eine $p$-Potenz als Ordnung.
		\newpar
		Ist $\abs{C_i} = 1$, gilt $C_i = \{z\}$ mit $z \in Z(G)$. Also:
		\begin{align*}
			\abs{G}  \equiv \abs{Z(G)} \mod p
		\end{align*}
		Ist $\abs{G} \neq 1$, folgt $\abs{Z(G)} \neq 1$.
	\end{proof}
	\begin{theorem}
		Sei $p$ prim und $\abs{G} = p^2$. So ist $G$ abelsch und isomorph zu $\bZ / p^2\bZ$ oder $\bZ / p\bZ \times \bZ / p\bZ$.
	\end{theorem}
	\begin{proof}
		Wir wissen $Z(G) \neq 1$. Wäre $Z(G) \neq G$, so wäre $\abs{Z(G)} = p$ und für $x \in G \setminus Z(G)$ wäre
		\begin{align*}
			\angles{Z(G), x} = G
		\end{align*}
		Aber diese erzeugte Gruppe ist abelsch! Widerspruch!!!!!!
	\end{proof}
	\begin{definition}
		Sei $G$ eine endliche Gruppe, $p$ eine Primzahl, und $p^r$ die größte $p$-Potenz, die $\abs{G}$ teilt. Eine Untergruppe $S \subset G$ mit $\abs{S} = p^r$ heißt \tbf{$p$-Sylow} von $G$
	\end{definition}
	\begin{theorem}
		In jeder $p$-Gruppe $G \neq 1$ gibt es ein Element $g \in G$ der Ordnung $p$.
	\end{theorem}
	\begin{theorem}
		Jede $p$-Gruppe ist nilpotent.
	\end{theorem}
	\begin{theorem}
		\theoremname{Sylow-Sätze:}
		Sei $G$ eine endliche Gruppe und $p$ prim, sodass $p^r$ $G$ maximal teilt.. So gilt:
		\begin{enumerate}
			\item Es gibt $p$-Sylows in $G$.
			\item Je zwei $p$-Sylows in $G$ sind zueinander konjugiert.
			\item Jede Untergruppe $H \subset G$, deren Ordnung eine $p$-Potenz ist, liegt in einer $p$-Sylow.
			\item Die Zahl der $p$-Sylows von $G$ ist kongruent zu $1$ modulo $p$ und teilt $\abs{G}/p^r$
		\end{enumerate}
	\end{theorem}
	\begin{example}
		In der Würfelgruppe $W$ ist $\abs{W} = 24$. Ein $2$-Sylow ist also eine Untergruppe mit Ordnung $2^3 = 8$. Davon gibt es drei, und zwar die Untergruppen, die Isomorph zur Diedergruppe $D_4$ sind.
	\end{example}
	\begin{proof}
		[Beweis (Erster Sylowsatz):]
		Seien $C_1, \hdots, C_r$ die Konjugationsklassen von $G$. Wir können diese weiter Zerlegen in $C_1, \hdots, C_s$ mit $s < r$ und $Z(G)$. Wir wählen ein $g_i \in C_i$ und erhalten (warum?)
		\begin{align*}
			\abs{G} = \abs{G / Z_G(g_1)} + \hdots + \abs{G / Z_G(g_s)} + \abs{Z(G)}
		\end{align*}
		Wobei $Z_G(x) = \{g \in G : gxg^{-1} = x\}$.
		\newpar
		Wir führen nun einen Beweis durch Induktion durch. Gibt es ein $i$, sodass $p^r$ $\abs{Z_G(g_i)}$ teilt, dann folgt die Aussage direkt per Induktion.
		\newpar
		Andernfalls teilt $p$ $\abs{G / Z_G(g_i)}$ für alle $i$, also teilt $p$ auch $\abs{Z(G)}$, also existiert ein $z \in Z(G)$ mit $\ord(z) = p$. 
		\newpar
		Wir wissen $\abs{G} = k p^r$ mit $k < p$, also gilt $\abs{G / \angles{z}} = k p^{r-1}$. Insbesondere existiert eine Untergruppe $\bar S \subset G / \angles{z}$ mit $\abs{\bar S} = p^{r-1}$, falls $k = 1$ ist dies die Gruppe $G / \angles{z}$ selbst. Wir definieren nun $S := \pi^{-1}(\bar S) \subset G$ - also ist $S$ die Gruppe, die durch die Quotientenabbildung auf $\bar S$ abgebildet wird - und es folgt $\abs{S} = p^r$.
	\end{proof}
	\begin{lemma}
		Sei $G$ eine Gruppe. Sei
		\begin{align*}
			\cS= \{Q \subset G : \abs{Q} = p^r\}
		\end{align*}
		die Menge der $p$-Sylows von $G$. Ist $Q \in \cS$ schreiben wir $g.Q = gQg^{-1}$ oder ist $x \in\cS$ schreiben wir einfach $gx = gxg^{-1}$.
		\newpar
		Ist $H \subset G$ eine $p$-Gruppe und $x = Q\in \cS$ ein Fixpunkt von $H$ in der Menge $\cS$ der $p$-Sylows, so gilt $H \subset Q$.
	\end{lemma}
	\begin{proof}
		$\forall h \in H : hx = x$ bedeuted $\forall h \in H : hQ = Qh$. Daraus folgt, dass $HQ$ eine Untergruppe von $G$ ist.
		\newpar
		Also gilt gemäß Bahnformel $\abs{HQ} = \abs{QH / H} \cdot \abs{H}$. Wir wissen $\abs{H} = p^r$, da $QH/H$ eine $Q$-Bahn ist, gilt außerdem $\abs{QH/H} = p^s$. Also ist die Ordnung von $HQ$ ebenfalls eine $p$-Potenz. Da die $p$-Potenz, die die Ordnung von $Q$ teilt, bereits maximal ist, folgt $\abs{HQ} = \abs{Q}$, also $HQ = Q$.
	\end{proof}
	\begin{proof}[Beweis (Zweiter Sylowsatz, Dritter Sylowsatz):]
		Sei $P = x \in \cS$ eine beliebige $p$-Sylow. Es gilt $P \subset G_P$, also gemäß Bahnformel $\abs{GP} = \abs{G} / \abs{G_P}$, also teilt $p$ nicht die Ordnung von $GP$. Ist jetzt also $H \subset G$ eine $p$-Gruppe, können wir $GP$ in $H$-Bahnen $B_1, \hdots, B_r$ zerlegen. Da $H$ eine $p$-Gruppe ist, ist für alle $i$ die Ordnung $\abs{B_i}$ eine $p$-Potenz. Da $p$ aber nicht $\abs{GP}$ teilt, muss es $B_i$ mit $\abs{B} = 1$ geben. Also existiert ein $y \in GP \subset \cS$, welche Fixpunkte von $H$ sind. Also existiert ein $Q$ der Gestalt $gPg^{-1}$, welches ein Fixpunkt von $H$ ist, sodass $H \subset Q$.
	\end{proof}
	\begin{proof}[Beweis (Vierter Sylowsatz):]
		Ganz $\cS$ ist eine $G$-Bahn $GP$ mit $P \in \cS$ beliebig. Da es nur einen Fixpunkt gibt, und da alle Bahnen $p$-Potenzordnung haben, gilt $\abs{\cS} \equiv 1 \mod p$. Es gilt außerdem $\abs{GP} = \abs{G} / \abs{G_p} \mid \abs{G} / \abs{P} = \abs{G} / p^r$.
	\end{proof}
	\begin{corollary}
		Sei $G$ eine endliche Gruppe und $p$ eine Primzahl, die die Gruppenordnung teilt. So gibt es ein Element von $G$ der Ordnung $p$.
	\end{corollary}
	\begin{proof}
		Gemäß der Sylowsätze existiert ein $p$-Sylow und darin ein Element der Ordnung $p$.
	\end{proof}
	\begin{corollary}
		Jede Gruppe mit $6$ Elementen ist isomorph zu $\cS_3$ oder $\bZ / 6 \bZ$.
	\end{corollary}
	\begin{proof}
		Gemäß Sylowsätzen existiert ein $a \in G$ mit $\abs{\angles{a}} = 2$ und ein $b \in G$ mit $\abs{\angles{b}} = 3$. Also gilt
		\begin{align*}
			\angles{a} \times \angles{b} \inj G,
		\end{align*}
		also
		\begin{align*}
			G = \{1,b,b^2,a,ab,ab^2\}
		\end{align*}
		Es folgt, dass entweder $ba = ab$ oder $ba = ab^2$. Wählen wir eine dieser beiden Möglichkeiten ist bereits die gesamte Gruppe festgelegt. Der erste Fall ergibt $\bZ / 6\bZ$, der zweite Fall ergibt $\cS_3$.
	\end{proof}
	\chapter{Ringe}
	\begin{example}
		Die Abbildung $\bZ \to \bZ$, welche jedes Element auf $0$ abbildet, ist kein Ringhomomorphismus, da das Einselement auf das Nullelement abgebildet wird. Wohl aber ist für einen beliebigen Ring $R$ die Abbildung $R \to 0$ in den Nullring ein Ringhomomorphismus.
	\end{example}
	\begin{lemma}
		Es gibt keinen Ringhomomorphismus $\phi : \bQ \to \bZ$.
	\end{lemma}
	\begin{proof}
		Es müsste gelten $\phi(1) = 1$, also $\phi(2) = 2$, und $\phi(2) \cdot \phi(\frac{1}{2}) = 2 \cdot \phi(\frac{1}{2})= 1$. Allerdings gibt es kein solches Element in $\bZ$.
	\end{proof}
	\begin{theorem}
		Für jeden Ring $R$ gibt es genau einen Ringhomomorphismus $\bZ \to R$.
	\end{theorem}
	\begin{proof}
		Analog wie für einen Gruppenhomomorphismus aus einer zyklischen Gruppe:
		\newpar
		Es gibt genau einen Gruppenhomomorphismus $\phi :(\bZ,+) \to (R,+)$ mit $1_\bZ \to 1_R$. Dieser bildet auch einen Ringhomomorphismus, denn für natürliches $a,b$ gilt:
		\begin{align*}
			\phi(a \cdot b) 
			&= \sum_{i = 1}^{ab} 1_R\\
			&= \sum_{i = 1}^a 1_R \cdot \sum_{i = 1}^b 1_R &\textnormal{(Induktiv per Distributivgesetz)}\\
			&= \phi\lr(\sum_{i = 1}^a 1) \cdot \phi\lr(\sum_{i = 1}^b 1)\\
			&= \phi(a) \cdot \phi(b)
		\end{align*}
		Und für negative $a,b \in \bZ$ folgt das Selbe, wenn man eine Summe bis zu einer negativen Zahl als Summe der Inversen definiert.
	\end{proof}
	\begin{theorem}
		Sei $R$ ein Ring. So ist $\ker(Z \to R) = n\bZ$ für ein $n \in \bN$.  Wir nennen dieses $n$ die \tbf{Charakteristik} von $R$.
	\end{theorem}
	\begin{proof}
		Jeder Ringhomomorphismus bildet einen Gruppenhomomorphismus. Der Kern muss also eine Untergruppe von $\bZ$ sein, also die Form $n\bZ$ haben.
	\end{proof}
	\begin{theorem}
		Die Charakteristik von $R$ ist das kleinste $m \geq 1$ mit $\displaystyle \sum_{i = 1}^m 1_R = 0_R$, falls es ein solches $m$ gibt, ansonsten $0$.
	\end{theorem}
	\begin{example}
		\phantom{}
		\begin{itemize}
			\item Es gilt $\char(\bZ) = \char(bQ) = \char(\bR) = 0$.
			\item Der einzige Ring mit Charakteristik $1$ ist der Nullring.
			\item $\char (\bF_p) = p$. 
		\end{itemize}
	\end{example}
	\begin{theorem}
		Sei $R$ ein Kring (kommutativer Ring) der Charakteristik $p$, mit $p$ prim. So ist der \tbf{Frobeniushomomorphismus}
		\begin{align*}
			Fr : R &\to R\\
				 a &\mapsto a^p := \prod_{i = 1}^p a
		\end{align*}
		ein Ringhomomorphismus.
	\end{theorem}
	\begin{proof}
		Es gilt:
		\begin{itemize}
			\item $\phi(1) = 1)$
			\item $\phi(ab) = \phi(a)\phi(b)$
			\item 
			$\displaystyle
				\phi(a + b) 
				= (a+b)^p
				= \sum_{k = 0}^p \binom{p}{k}a^kb^{p-k}
			$
			\newpar
			In der letzten Summe haben alle Terme außer dem ersten und dem letzten nun durch den Binomialkoeffizienten einen Faktor $p$ und sind im Ring somit $0$. Also
			\begin{align*}
				\sum_{k = 0}^p \binom{p}{k}a^kb^{p-k} 
				&= a^p + b^p\\
				&= \phi(a) + \phi(b)
			\end{align*}
		\end{itemize}
	\end{proof}
	\begin{theorem}
		\theoremname{Universelle Eigenschaft surjektiver Ringhomomorphismen:} Sei $s : R \surj Q$ ein surjektiver Ringhomomorphismus und $\phi : R \to S$ ein beliebiger Ringhomomorphismus. So existiert genau dann ein Ringhomomorphismus $\bar\phi$ mit $\bar\phi \circ s = \phi$, wenn $\ker \phi \supseteq \ker s$. Falls $\bar\phi$ existiert, ist es eindeutig.
	\end{theorem}
	\begin{proof}
		Es ist bereits bekannt, dass ein eindeutiger Gruppenhomomorphismus $\bar\phi$ zwischen den additiven Gruppen existiert. Zu zeigen ist, dass $\bar\phi$ ein Ringhomomorphismus ist. Dies ist jedoch simpel - es gilt:
		\begin{itemize}
			\item $\bar\phi(1_Q) = ( \bar\phi \circ s)(1_R) = \phi(1_Q) = 1_S$
			\item Da $s$ surjektive ist, reicht für die Multiplikativität $\bar\phi(s(a) \cdot s(b)) = \bar\phi(s(a)) \cdot \bar\phi(s(b))$.
			\begin{align*}
				\bar\phi(s(a) \cdot s(b))
				&= \bar\phi(s(ab)\\
				&= \phi(ab)\\
				&= \phi(a) \cdot \phi(b)\\
				&= \bar\phi(s(a)) \cdot \bar\phi(s(b))
			\end{align*}
		\end{itemize}
	\end{proof}
	\noindent
	Wie im Fall für Gruppen existiert also genau ein Ringisomorphismus zwischen den Bildern zweier Ringhomomorphismen mit gleichem Kern. Wir fragen uns nun: Gegeben einen Ring $R$ - welche Untergruppen von $(R,+)$ sind Kerne von Ringhomomorphismen? Die Antwort sind \tbf{Ideale}.
	\begin{definition}
		Eine Untergruppe $I$ der Additiven Gruppe $(R,+)$ eines Rings $R$ heißt \tbf{Ideal}, falls
		$RI \subset I$ und $IR \subset I$ (also wenn das Produkt eines Ideals mit einem beliebigen Ringelement immer im Ideal liegt)
	\end{definition}
	\begin{definition}
		Ein zyklisches Ideal heißt \tbf{Hauptideal}.
	\end{definition}
	\begin{lemma}
		Sei $\phi : R \to S$ ein Ringhomomorphismus. So ist $\ker \phi \subset R$ ein Ideal.
	\end{lemma}
	\begin{proof}
		$\phi(x) = 0 \implies \phi(ax) = \phi(xa) = 0$
	\end{proof}
	\begin{theorem}
		Gegeben einen Ring $R$ mit Ideal $I$ gibt es einen surjektiven Ringhomomorphismus $s : R \surj Q$ mit Kern $I$.
	\end{theorem}
	\begin{proof}
		Betrachte die Gruppe $(R / I, +)$. Wir behaupten, dass es darauf genau eine Multiplikation gibt, die verträglich ist mit $\phi : R \to R / I$. Zu zeigen ist nur, dass die Abbildung $s \circ \cdot_R$ konstant auf den Fasern von $s \times s : R \times R \to R/I \times R/I$ ist. Diese Fasern haben genau die Form $(a + I) \times (b + I)$. Es gilt:
		\begin{align*}
			(a+I) \circ (b+I) &= ab + aI + Ib + II\\
			\implies s(ab + aI + Ib + II) &= ab + I
		\end{align*}
		(Da $aI + Ib + II \subset I$.)
	\end{proof}
	\begin{theorem}
		$n \in \bN$ lässt beim Teilen durch $9$ den selben Rest wie seine Quersumme und beim Teilen durch $11$ den selben Rest wie seine alternierende Quersumme.
	\end{theorem}
	\begin{proof}
		$10^n \equiv 1^n \mod 9$ und $10^n \equiv -1^n \mod 11$
	\end{proof}
	\begin{definition}
		Sei $R$ ein Kring.
		\begin{enumerate}
			\item Seien $a,b \in R$. Wir sagen, $a$ \tbf{teilt} $b$, falls ein $c \in R$ existiert, sodass $ac = b$.
			\item Ein Element $a \in R$ heißt \tbf{kürzbar}, wenn die Multiplikation $a \cdot : R \to R$ injektiv ist, also wenn $ax = ay \implies x = y$.
			\item $R$ heißt \tbf{Integritätskring}, wenn genau ein nicht kürzbare Element in $R$ existiert (Dieses ist dann zwingend die $0_R$).
		\end{enumerate}
	\end{definition}
	\begin{example}
		\phantom{}
		\begin{itemize}
			\item Alle $a \in \bZ \setminus 0$ sind kürzbar.
			\item $\bZ / 12\bZ$ ist kein Integritätsring, da $3 \cdot 4 = 3 \cdot 8 = 3 \cdot 0$. Im Allgemeinen ist $\bZ / m\bZ$ genau dann ein Integritätsring, wenn $m$ prim ist.
			\item Jeder Körper ist ein Integritätsring.
			\item Jeder Teilring eines Körpers ist ein Intergritätsring.
		\end{itemize}
	\end{example}
	\begin{corollary}
		Die Ringmultiplikation ist genau dann injektiv, wenn ihr Kern trivial ist, also wenn der Ring Nullteilerfrei ist.
	\end{corollary}
	\begin{lemma}
		Jeder endliche kommutative Integritätsring $R$ ist ein Körper.
	\end{lemma}
	\begin{proof}
		Sei $a \in R \setminus 0$ kürzbar. Dann ist die Multiplikation mit $a$ injektiv. Da $R$ endlich ist, ist die Multiplikation auch bijektiv, also existiert ein multiplikatives Inverses. Jeder kommutative Ring mit multiplikativen Inversen ist per Definition ein Körper.
	\end{proof}
	\begin{corollary}
		Die Charakteristik eines Körpers ist $0$ oder eine Primzahl.
	\end{corollary}
	\begin{proof}
		$\bZ / (\char K) \bZ \inj K$ ist injektiv. Dies ist nur möglich, wenn $(\char K)$ $0$ oder prim ist.
	\end{proof}
	\begin{definition}
		Sei $R$ ein kommutativer Ring. $a \in R$ heißt \tbf{Einheit}, wenn es ein multiplikatives inverses gibt. Einheiten sind also quasi die "Einsteiler". Wir schreiben die Menge der Einheiten als $R^\times$ (da die Einheiten die größte multiplikativen Gruppe des Rings bilden).
	\end{definition}
	\begin{example}
		\phantom{}
		\begin{itemize}
			\item $\bZ^\times = \{-1,1\}$
			\item $\bQ^\times = \bQ \setminus 0$
			\item $0^\times = 0$
			\item $\bR[X]^\times = \bR^\times$
		\end{itemize}
	\end{example}
	\noindent In der Physik hat man einen eindimensionalen $\bR$-Vektorraum $V$ der Länge, der Zeit, etc. Die Einheiten sind genau die Basen dieses Vektorraums, also bildet die Multiplikation eine Bijektion $\bR \iso V$. Analog ist die Multiplikation mit einer Einheit eine Bijektion $R \iso R$.
	\begin{corollary}
		In jedem endlichen kommutativen Ring sind die Einheiten genau die kürzbaren Elemente.
	\end{corollary}
	\begin{example}
		\phantom{}
		\begin{itemize}
			\item $(\bZ / 12\bZ)^\times = \{1,5,7,11\}$
			\item $(\bZ / m\bZ)^\times = \{a \mid \ggT(a,m) = 1\}$
		\end{itemize}
	\end{example}
	\begin{definition}
		\theoremname{Diffie-Hellman-Verschlüsselung:}
		\begin{enumerate}
			\item Alice wählt eine beliebige Gruppe $G$ und eine natürliche Zahl $a \in \bN$. Sie wählt ein Element $g \in G$ und veröffentlicht $G, g^a$ und $g$.
			\item Daraufhin wählt Bob ein $b \in \bN$ und sendet $g^b$.
			\item Alice rechnet nun $(g^b)^a$ und Bob rechnet $(g^a)^b$. Dies ist der gemeinsame geheime Schlüssel.
		\end{enumerate}
		Die Sicherheit des Algorithmus beruht darauf, dass der diskrete Logarithmus (also das berechnen von $a$ aus $g^a$) im Allgemeinen sehr ineffizient ist. Die häufigste Wahl ist $G = (\bZ / p \bZ)^\times$ für eine sehr große Primzahl $p$, in besonders sicherheitskritischen Fällen wird als Gruppe oft eine elliptische Kurve über einem endlichen Körper genommen.
	\end{definition}
	\begin{definition}
		Sei $R$ ein Ring und $R[X]$ der Polynomring. Formell kann $R[X]$ auch definiert werden als die Menge aller Abbildungen $\bN \to R$, die höchstens an endlich vielen Stellen $\neq 0$ sind (diese Stellen entsprechen dann den Koeffizienten.) Wir unterscheiden zwischen \tit{Polynomen} und \tit{Polynomfunktionen} - Im Allgemeinen ist die kanonische Abbildung $\bF_p[X] \inj \Ens(\bF_p, \bF_p)$ nämlich nicht injektiv!
	\end{definition}
	\begin{lemma}
		Im Fall eines endlichen Körpers $\bF_p$ ist die kanonische Abbildung $\bF_p[X] \surj \Ens(\bF_p, \bF_p)$ surjektiv. Der Kern dieser Abbildung ist das Ideal, welches von $(x^p - x)$ erzeugt wird.
	\end{lemma}
	\begin{proposition}
		Sei $R$ ein kommutativer Ring und $c \in R$. So gibt es genau einen Ringhomomorphismus
		\begin{align*}
			E_c : R[X] \to R,
		\end{align*}
		sodass $E_c(X) = c$ und $E_c \circ \imath = \id$. (Wobei $\imath$ die kanonische Einbettung $R \to R[X]$ ist, welche ein Element als konstantes Polynom auffasst. Professor Soergel schreibt diese auch als $\textnormal{can}$.)
		\newpar
		Wir nennen $E_c$ den \tbf{Einsetzungshomomorphismus} und schreiben abkürzend 
		\begin{align*}
			P(c) := E_c(P)
		\end{align*}
	\end{proposition}
	\begin{proposition}
		Sei $\phi : R \to S$ ein Ringhomomorphismus und $c \in S$ (nicht $c \in R$!) ein Element, welches für alle $r \in R$ mit $\phi(r)$ bezüglich der Multiplikation kommutiert. So gibt es genau einen Ringhomomorphismus 
		\begin{align*}
			E_{c, \phi} : R[X] \to S
		\end{align*} mit $E_{c, \phi}[X] = c$ und $E_{c, \phi} \circ \imath = \phi$.
	\end{proposition}
	\noindent Diese Proposition verallgemeinert die Vorherige, indem sie uns zum Beispiel erlaubt, Matrizen oder komplexe Zahlen in reelle Polynome einzusetzen.
	\begin{proposition}
		Jeder Ringhomomorphismus $\phi : R \to S$ liefert einen Ringhomomorphismus $\phi[X] : R[X] \to S[X]$.
	\end{proposition}
	\begin{definition}
		Sei $R$ ein kommutativer Ring und $P \in R[X]$. Ein Element $\lambda \in R$ heißt \tbf{Nullstelle von $P$}, falls $P(\lambda) = 0$.
	\end{definition}
	\begin{definition}
		Der \tbf{Grad} eines Polynoms ist definiert als:
		\begin{align*}
			\deg(P) = 
			\begin{cases}
				n & \textnormal{Falls $P = a_nx^n + \hdots + a_0x^0, a_n \neq 0$}\\
				-\infty & \textnormal{Falls $P = 0$.}
			\end{cases}
		\end{align*}
	\end{definition}
	\begin{corollary}
		Sei $R$ ein Integritätsring und $P, Q \in R[X]$. Dann gilt
		\begin{align*}
			\deg(PQ) = \deg(P) + \deg(Q)
		\end{align*}
	\end{corollary}
	\noindent Unser Ziel ist es, folgenden Satz zu beweisen:
	\begin{theorem}
		Sei $R$ ein kommutativen Integritätsring und $P \in R[X] \setminus 0$. Dann hat $P$ höchstens $\deg(P)$ Nullstellen.
	\end{theorem}
	\begin{proofsketch}
		Die kommutativen Integritätsringe sind genau die Ringe, über denen die gewohnte Zerlegung in Linearfaktoren funktioniert.
	\end{proofsketch}
	\begin{counterexample}
		Sei $R = \Ens(M, \bR)$. Dann ist $R$ ein Ring, aber kein Integritätsring. So hat das Polynom $X^2 - X$ unendlich viele Nullstellen, nämlich alle Abbildungen $f : M \to \{0,1\}$.
	\end{counterexample}
	\begin{counterexample}
		Ebenso gilt der Satz nicht für nichtkommutative Ringe. Es existieren zum Beispiel Zahlreiche Matrixringe, welche idempotente Matrizen enthalten, also Matrizen $M$ mit $M^n = I$. Somit sind alle skalaren Vielfachen dieser Matrizen Nullstellen von $X^n$.
	\end{counterexample}
	\begin{theorem}
		Sei $k$ ein kommutativer Ring und $P,Q \in k[X]$, wobei $Q$ normiert ist (also Leitkoeffizient $1$ hat.) Dann existieren eindeutige $A,B \in k[X]$, sodass
		\begin{align*}
			P = A \cdot Q + B
		\end{align*}
		Wobei $\deg B \leq (\deg Q) - 1$
	\end{theorem}
	\begin{example}
		Der Quotientenring $\bR[X] / \angles{X^2 + 1}$ ist isomorph zu den Komplexen Zahlen (Da $X^2 + 1 = 0 \implies X^2 = -1 \implies i := X$
	\end{example}
	\begin{theorem}
		Sei $P \in \bR[X]$ ein Polynom vom Grad $2$ ohne reelle Nullstellen, und sei $\alpha \in \bC$ mit $P(\alpha) = 0$. Dann gilt $\bR[X]/\angles{P} \iso \bC$.
	\end{theorem}
	\begin{theorem}
		Sei $R$ ein Ring, $\fa_1, \hdots, \fa_r \subset R$ Ideale mit $i \neq j \implies \fa_i + \fa_j = R$. So ist die durch Restklassen gegebene Abbildung $\phi$ eine Surjektion $R \surj R/\fa_1 \times \hdots \times R/\fa_j$.
	\end{theorem}
	\begin{proof}
		Die Abbildung ist klar ein Ringhomomorphismus. Es reicht also zu zeigen, dass alle "Einheitsvektoren" $(0, \hdots, 0, 1, 0, \hdots, 0)$ im Bild liegen, da dann der Rest durch Linearkombination folgt.
		\newpar
		OBdA reicht $(1, 0, \hdots, 0) \in \im(\phi)$. Es gilt:
		\begin{align*}
			&\fa_1 + \fa_i = R\\ \implies &\exists a_i \in \fa_1, b_i \in \fa_i : a_i + b_i = 1\\
			\implies &1 - a_i = b_i\\
			\implies &\phi_1(1 - a_i) = [1 - a_i]_{\fa_1} = 1,\\ &\phi_i(1 - a_i) = [b_i]_{\fa_i} = 0\\
			\implies &\phi\lr(\prod_{i \geq 2}(1 - a_i)) = (1, \hdots, 0)
		\end{align*}
	\end{proof}
	\begin{corollary}
		\theoremname{Polynominterpolation:}
		Gegeben $x_1, \hdots, x_r \in \bR$ paarweise verschieden und $y_1, \hdots, y_r \in \bR$ beliebig gibt es ein Polynom $P \in \bR[T]$, sodass $\forall i : P(x_i) = y_i$.
	\end{corollary}
	\begin{proof}
		% https://q.uiver.app/#q=WzAsMyxbMCwwLCJcXGJSW1RdIl0sWzIsMCwiXFxiUiJdLFsxLDIsIlxcYlJbVF0gLyBcXGFuZ2xlc3tUIC0geH0iXSxbMCwxLCJFX3giXSxbMCwyLCIiLDIseyJzdHlsZSI6eyJoZWFkIjp7Im5hbWUiOiJlcGkifX19XSxbMiwxLCJcXHNpbSIsMix7InN0eWxlIjp7ImJvZHkiOnsibmFtZSI6ImRhc2hlZCJ9fX1dXQ==
		\[\begin{tikzcd}
			{\bR[T]} && \bR \\
			\\
			& {\bR[T] / \angles{T - x}}
			\arrow["{E_x}", from=1-1, to=1-3]
			\arrow[two heads, from=1-1, to=3-2]
			\arrow["\sim"', dashed, from=3-2, to=1-3]
		\end{tikzcd}\]
		The rest should be entirely obvious and is left as an exercise to the reader. (Maybe coming later :)
	\end{proof}
	Führt man eine Polynominterpolation durch, findet man die "Basisvektoren", indem man pro $x_i$ für alle $x_j$ die Geraden durch $(x_i, y_0)$ und $(x_j, 0)$ miteinander multipliziert. Diese werden dann aufsummiert. Letztendlich führt man also die selbe Rechnung durch wie in unserem abstrakteren Beweis über Ideale.
	\begin{theorem}
		Es gilt zusätzlich $\ker \phi = \bigcap_{i = 1}^r \fa_i$. Ist $R$ kommutativ, ist dies außerdem äquivalent zu $\angles{\fa_1 \fa_2 \hdots \fa_r}$.
	\end{theorem}
	\begin{proof}
		Seien $\fa_1, \fa_2 \subset R$ und $R$ kommutativ mit $\fa_1 + \fa_2 = R$. Wir zeigen $\fa_1 \cap \fa_2 = \angles{\fa_1 \fa_2}$. Die Richtung $\fa_1 \cap \fa_2 \supseteq \angles{\fa_1 \fa_2}$ ist klar.
		\newpar
		Es existieren $a_1 \in \fa_1$, $a_2 \in \fa_2$, sodass $a_1 + a_2 = 1$. Dann gilt insbesondere 
		\begin{align*}
			\forall x \in \fa_1 \cap \fa_2 :\\
			 a_1 x + a_2 x = x
		\end{align*}
		Also $\fa_1 \cap \fa_2 \subseteq \angles{\fa_1 \fa_2}$, also auch $\fa_1 \cap \fa_2 = \angles{\fa_1 \fa_2}$. Der Rest folgt durch Induktion.
	\end{proof}
	\begin{corollary}
		\begin{align*}
			\bR[T] / \angles{X^2 - 1} \cong \bR[T] / \angles{X + 1} \times \bR[T] / \angles{X - 1} \cong \bR \times \bR
		\end{align*}
	\end{corollary}
\end{document}