\documentclass{report}
\usepackage[a4paper,margin=1.5in]{geometry}
\usepackage{fancyhdr}
\usepackage[titles]{tocloft}
\usepackage[titletoc]{appendix}
\usepackage{tikz}
\usepackage{xcolor}

\usepackage{multicol}
\usepackage{amsmath}
\usepackage{amssymb}
\usepackage{amsthm}
\usepackage{pdfpages}
\usepackage{bm}
\usepackage{tikz-cd}
\usepackage{physics}
\usepackage{placeins}

%\usepackage{setspace}

%\onehalfspacing

%hyperref should be last apparently
\usepackage{hyperref}

\renewcommand\cftsecdotsep{\cftdot}
\renewcommand\cftsubsecdotsep{\cftdot}
\renewcommand\epsilon{\varepsilon}

% Starts a new paragraph without indentation
% and with an empty line between paragraphs
\newcommand*{\newpar}{\par\vspace{\baselineskip}\noindent}
\newcommand{\trans}{\twoheadrightarrow}
\newcommand{\ttt}[1]{\texttt{#1}}
\newcommand{\tbf}[1]{\textbf{#1}}
\newcommand{\ul}[1]{\underline{#1}}

\newcommand{\Hess}[1]{\text{Hess}(#1)}

\newcommand{\bC}{\mathbb{C}}
\newcommand{\bF}{\mathbb{F}}
\newcommand{\bN}{\mathbb{N}}
\newcommand{\bQ}{\mathbb{Q}}
\newcommand{\bR}{\mathbb{R}}
\newcommand{\bZ}{\mathbb{Z}}

\newcommand{\cA}{\mathcal{A}}
\newcommand{\cB}{\mathcal{B}}
\newcommand{\cC}{\mathcal{C}}
\newcommand{\cE}{\mathcal{E}}
\newcommand{\cM}{\mathcal{M}}
\newcommand{\cP}{\mathcal{P}}
\newcommand{\cX}{\mathcal{X}}

\newcommand{\ve}{\vec{e}}
\newcommand{\vh}{\vec{h}}
\newcommand{\vv}{\vec{v}}
\newcommand{\vw}{\vec{w}}
\newcommand{\vx}{\vec{x}}
\newcommand{\vy}{\vec{y}}
\newcommand{\vz}{\vec{0}}
\newcommand{\zz}{\vec{z}}

\newcommand{\tbA}{\mathbf{A}}
\newcommand{\tbB}{\mathbf{B}}
\newcommand{\tbC}{\mathbf{C}}
\newcommand{\tbD}{\mathbf{D}}
\newcommand{\tbE}{\mathbf{E}}
\newcommand{\tbY}{\mathbf{Y}}
\newcommand{\tbZ}{\mathbf{Z}}

\newcommand{\an}{(a_n)_{n \in \bN}}
\newcommand{\bn}{(b_n)_{n \in \bN}}
\newcommand{\sn}{(s_n)_{n \in \bN}}

\newcommand{\ggT}{\text{ggT}}
\newcommand{\kgV}{\text{kgV}}

\renewcommand{\tr}{\text{tr}\ }
\newcommand{\rang}{\text{rang}\ }

\newcommand{\Mat}[3]{\text{Mat}^{#1}_{#2}\left(#3\right)}
\newcommand{\scalar}[2]{\left\langle #1, #2 \right\rangle}

\renewcommand*\contentsname{Inhalt}
\renewcommand*\proofname{Beweis}

\pagestyle{fancy} %allows headers

\lhead{Emma Bach}
\rhead{\today}

\NewDocumentEnvironment{nalign}{}{\equation\aligned}{\endaligned\endequation}

\begin{document}

% \newtheorem{codename}{printedname}[countedwith]
\newtheorem{lemma}{Lemma}[chapter]
\newtheorem{theorem}[lemma]{Satz}
\newtheorem{proposition}[lemma]{Proposition}
\newtheorem{corollary}[lemma]{Korollar}

\theoremstyle{definition}
\newtheorem{definition}[lemma]{Definition}
\newtheorem{beispiel}[lemma]{Beispiel}
\newtheorem{beobachtung}[lemma]{Beobachtung}
\newtheorem{anmerkung}[lemma]{Anmerkung}
\newtheorem{question}[lemma]{Frage}
\newtheorem{application}[lemma]{Anwendung}
\newtheorem{konsequenz}[lemma]{Konsequenz}

\newenvironment{proofsketch}{\begin{proof}[Beweisskizze]\renewcommand*{\qedsymbol}{\("\square"\)}}{\end{proof}}

%
%
%
\begin{titlepage}
	\centering
	{\Large \textsc{Mitschrieb}\par}
	\vspace{0.5cm}
	{\huge\bfseries Numerik\par}
    \vspace{0.5cm}
	{\Large\itshape Emma Bach\par}
	\vfill
	
	Basierend auf:\par 
	\vspace{1cm}
	Vorlesungen Numerik I + II von\par
	Prof. Dr. Patrick  \textsc{Dondl}\par
	
	\vfill

% Bottom of the page
	{\large \today\par}
\end{titlepage}

\tableofcontents
\thispagestyle{fancy}
%
%
%
%
%
\chapter{Primzahlen}
\begin{definition}
	$H \subseteq \bZ$ heißt Untergruppe von $\bZ$, wenn 
	\begin{align*}
		n \in H \implies -n \in H,
	\end{align*} und 
	\begin{align*}
		m, n \in H \implies m + n \in H
	\end{align*}
\end{definition}
\begin{theorem}
	Es gibt eine Bijektion 
	\begin{align*}
		\bN &\leftrightarrow \{\text{\emph{Untergruppen von ($\bZ, + $)}}\}\\
		n &\mapsto n\bZ
	\end{align*}
\end{theorem}
\begin{proof}
	Sei $H \subseteq \bZ$ eine Untergruppe. Entweder $H = \{0\} = 0\bZ$, oder $H \neq \{0\}$.
	\newpar
	Sei also $H \neq \{0\}$. Dann gibt es ein kleinstes Element $m$ der Menge $\{n \in H \mid n > 0\}$.
	Aus den Untergruppenaxiomen folgt $m\bZ \subseteq H$. Gleichzeitig kann kein Element $n \notin m \bZ$ in $H$ enthalten sein, denn sonst wäre auch $r = n \mod m \neq 0$ in $H$ enthalten. Dann hätten wir aber $r < m$, was ein Widerspruch ist.
	\newpar
	Als Umkehrfunktion wählen wir das kleinste positive Element von $H$.
\end{proof}
\begin{definition}
	Eine \tbf{Primzahl} ist eine natürliche Zahl $p \in \bN_{\geq 2}$, die nicht als Produkt zweier Zahlen $a,b < p$ geschrieben werden kann.
\end{definition}
\begin{theorem}
	Es gibt unendlich viele Primzahlen.
\end{theorem}
\begin{proof}
	Angenommen, es gäbe endlich viele Primzahlen. Sei also $p_1, \hdots, p_r$ eine vollständige Liste aller Primzahlen. Dann wäre aber
	\begin{align*}
		q = 1 + \prod_{i = 1}^{r} p_i
	\end{align*}
	durch keine Primzahl teilbar, also selbst eine Primzahl. Widerspruch!
\end{proof}
\begin{theorem}
	Jede Zahl $n \in \bN$ kann als Produkt von Primzahlen geschrieben werden:
	\begin{align*}
		n = p_1 \cdot \hdots \cdot p_r \quad (r \geq 0)
	\end{align*}
\end{theorem}
\begin{proof}
	Der Fall $n = 1$ gilt per Konvention durch das leere Produkt.
	\newpar
	Sei $n \geq 2$ gegeben. Es gilt entweder:
	\begin{itemize}
		\item $n$ ist eine Primzahl.
		\item $n$ ist von der Gestalt $n = a \cdot b$, mit $a,b < n$.
	\end{itemize}
	Der Satz folgt durch Induktion über die entstehende Baumstruktur - nach Induktionsannahme haben $a$ und $b$ eine Primfaktorzerlegung. Also hat auch $n$ eine Primfaktorzerlegung. 
\end{proof}
\begin{definition}
	Der \tbf{größte gemeinsame Teiler} von $a,b \in \bZ$ mit $a \neq 0$ oder $b \neq 0$ ist die Zahl:
	\begin{align*}
		\ggT(a,b) = \max\{d \in \bN : d \mid a \wedge d \mid b\}
	\end{align*}
\end{definition}
\begin{theorem}
	\ul{\emph{Über den größten gemeinsamen Teiler}}: Seien $a,b \in \bZ$. So gibt es $r,s \in \bN$ mit 
	\begin{align*}
		\ggT(a,b) = ra + sb
	\end{align*}
	Gegeben $d \mid a$ und $d \mid b$ gilt außerdem $d \mid \ggT(a,b)$.
\end{theorem}
\begin{proof}
	Die Menge
	\begin{align*}
		H := \{ra + sb \mid r,s \in \bZ\} = a\bZ + b\bZ
	\end{align*}
	bildet eine Untergruppe von $Z$, ist also eine Gruppe der Form $m \bZ$ mit $m > 0$. Da $a \in m\bZ$ und $b \in m\bZ$ ist $m$ ein gemeinsamer Teiler. Da $m$ ein Element in $H$ ist existiert außerdem per Definition eine Darstellung $m = r'a + s'b$. Es gilt also 
	\begin{align*}
		(d \mid a) \wedge (d \mid b) \implies d \mid r'a + s'b \implies d \mid m.
	\end{align*}
	\newpar
	Aus $(d \mid a) \wedge (d \mid b) \implies d \mid m$ folgt nun, dass jeder Teiler von $a$ und $b$ $m$ teilt, also kann es keinen Teiler von $a$ und $b$ geben, welcher größer als $m$ ist.
\end{proof}
\noindent Die Existenz der Darstellung $\ggT(a,b) = ra + sb$ ist auch als das \tbf{Lemma von Bézout} oder die \tbf{Bézoutsche Identät} bekannt.
\begin{lemma}
	\ul{\emph{Lemma von Euklid:}} Sei $p$ eine Primzahl und $a,b \in \bZ$.
	Dann gilt:
	\begin{align*}
		p \mid ab \implies (p \mid a) \vee (p \mid b)
	\end{align*}
\end{lemma}
\begin{proof}
	Es reicht zu Zeigen:
	\begin{align*}
		(p \nmid a) \wedge (p \mid ab) \implies p \mid b
	\end{align*}
	Aus $p \nmid a$ folgt $\ggT(p,a) = 1$. Nach dem Lemma von Bézout können wir also $1$ darstellen als:
	\begin{align*}
		1 = rp + sa
	\end{align*}
	also:
	\begin{align*}
		b = rpb + sab
	\end{align*}
	Es gilt trivial $p \mid prb$, außerdem gilt per Annahme $p \mid ab$. Es folgt $p \mid rpb + sab = b$.
\end{proof}
\begin{theorem}
	\ul{\emph{Eindeutigkeit der Primfaktorzerlegung im Ring $\bZ$:}}
	Sei $n \in \bN_{\geq 1}$ und
	\begin{align*}
		n = \prod_{i = 1}^r p_i = \prod_{i = 1}^s q_i
	\end{align*}
	wobei alle $q_i$ und $r_i$ Primzahlen sind. So gilt $r = s$ und es gilt eine Permutation $\sigma \in S_r$ mit $p_i = q_{\sigma(i)}$.
\end{theorem}
\noindent Äquivalente Formulierungen:
\begin{itemize}
	\item Falls die $p_i$ und $q_i$ Aufsteigend oder Absteigend sortiert sind, gilt $\forall i : p_i = q_i$
	\item Es existiert eine Bijektion zwischen endlichen Multimengen von Primzahlen und $\bN_{\geq 1}$.
\end{itemize} 
\begin{proof}
	Per Induktion folgt aus dem Lemma von Euklid schnell:
	\begin{align*}
		p_1 \mid p_1 \implies \bigvee_{i = 1}^s p_1 \mid q_i
	\end{align*}
	Also existiert ein $q_i$ mit $p_1 = q_i$. Teilen wir nun beide Seiten durch $p_1$, folgt die Aussage durch die Induktionsannahme.
\end{proof}
\end{document}