\documentclass{report}

% custom margins
\usepackage[a4paper,margin=1.5in]{geometry}

% emma's long list of custom macros and universally used packages
%AMS packages - Symbols, "Theorem" and "Proof" Environments
\usepackage{amsmath}
\usepackage{amssymb}
\usepackage{amsthm}
\usepackage{physics}
\renewcommand{\div}{\tn{div}~}

% Nicer Headers and Footers
\usepackage{fancyhdr}
\usepackage[yyyymmdd]{datetime}
\renewcommand{\dateseparator}{-}

% Table of Contents
\usepackage[titles]{tocloft}
\usepackage[titletoc]{appendix}

% Tikz and Graphics
\usepackage{tikz}
\usepackage{tikz-cd}
\usepackage{xcolor}
\usepackage[many]{tcolorbox}

% Nicer Underlining
\usepackage{contour}
\usepackage{ulem}

% Multiple Text Columns
\usepackage{multicol}

% FloatBarrier
\usepackage{placeins} 

% hyperref should be last apparently
\usepackage{hyperref}

%--- RENEWED COMMANDS ---

% Variant Greek Letters
\renewcommand\epsilon{\varepsilon}
\renewcommand\phi{\varphi}

% Nicer Table of Contents
\renewcommand\cftsecdotsep{\cftdot}
\renewcommand\cftsubsecdotsep{\cftdot}

%--- TEXT FORMATING ---

% nice underlining
\renewcommand{\ULdepth}{1.6pt}
\contourlength{0.8pt}
\newcommand{\ul}[1]{%
	\uline{\phantom{#1}}%
	\llap{\contour{white}{#1}}%
}

% shorthand
\newcommand{\tn}[1]{\textnormal{#1}}
\newcommand{\tbf}[1]{\textbf{#1}}
\newcommand{\tit}[1]{\textit{#1}}
\newcommand{\ttt}[1]{\texttt{#1}}

\newcommand{\mc}[1]{\mathcal{#1}}

\newcommand{\ol}[1]{\overline{{#1}}}

% Underlined, bold, non-cursive Theorem Name
\newcommand{\theoremname}[1]{\ul{\textnormal{\tbf{#1}}}}

% Starts a new paragraph without indentation
% and with an empty line between paragraphs
\newcommand*{\newpar}{\par\vspace{\baselineskip}\noindent}

% Make inf height match sup height
\renewcommand{\inf}{\mathop{\mathrm{inf}\vphantom{\mathrm{sup}}}}

% Make tildes more readable
\renewcommand{\tilde}{\widetilde}

%--- RELATION SYMBOLS AND OPERATORS ---
\newcommand{\surj}{\twoheadrightarrow}
\newcommand{\inj}{\hookrightarrow}
\newcommand{\iso}{\overset{\sim}{\rightarrow}}
\newcommand{\symdiff}{\vartriangle}
\newcommand{\trans}{\twoheadrightarrow}
\newcommand{\tensor}{\otimes}
\newcommand{\bigmid}{~\middle|~}

%--- STUFF IN FANCY BRACKETS ---
\newcommand{\scalar}[2]{\left\langle #1, #2 \right\rangle}
\newcommand{\angles}[1]{\left\langle #1 \right\rangle}
\newcommand{\lr}{\qty}


%--- LETTERS ---
\newcommand{\dmu}{\ d\mu}
\newcommand{\dist}{\textnormal{dist}}
\newcommand{\spt}{\textnormal{spt}}

% mathbb
\newcommand{\bC}{\mathbb{C}}
\newcommand{\bF}{\mathbb{F}}
\newcommand{\bN}{\mathbb{N}}
\newcommand{\bQ}{\mathbb{Q}}
\newcommand{\bR}{\mathbb{R}}
\newcommand{\bZ}{\mathbb{Z}}

% mathcal
\newcommand{\cA}{\mathcal{A}}
\newcommand{\cB}{\mathcal{B}}
\newcommand{\cC}{\mathcal{C}}
\newcommand{\cD}{\mathcal{D}}
\newcommand{\cE}{\mathcal{E}}
\newcommand{\cF}{\mathcal{F}}
\newcommand{\cH}{\mathcal{H}}
\newcommand{\cI}{\mathcal{I}}
\newcommand{\cL}{\mathcal{L}}
\newcommand{\cM}{\mathcal{M}}
\newcommand{\cN}{\mathcal{N}}
\newcommand{\cO}{\mathcal{O}}
\newcommand{\cP}{\mathcal{P}}
\newcommand{\cQ}{\mathcal{Q}}
\newcommand{\cR}{\mathcal{R}}
\newcommand{\cS}{\mathcal{S}}
\newcommand{\cT}{\mathcal{T}}
\newcommand{\cW}{\mathcal{W}}
\newcommand{\cX}{\mathcal{X}}
\newcommand{\cZ}{\mathcal{Z}}

% mathfrac
\newcommand{\fa}{\mathfrak{a}}

% real numbers in fancy costumes
\newcommand{\barR}{\ol{\mathbb{R}}}
\newcommand{\bRnn}{\mathbb{R}^{n \times n}}
\newcommand{\bRmn}{\mathbb{R}^{m \times n}}
\newcommand{\bRnm}{\mathbb{R}^{n \times m}}

% vectors
\renewcommand{\va}{\vec{a}}
\renewcommand{\vb}{\vec{b}}
\newcommand{\vc}{\vec{c}}
\newcommand{\ve}{\vec{e}}
\newcommand{\vF}{\vec{F}}
\newcommand{\vh}{\vec{h}}
\newcommand{\vp}{\vec{p}}
\newcommand{\vr}{\vec{r}}
\newcommand{\vs}{\vec{s}}
\newcommand{\vT}{\vec{T}}
\renewcommand{\vu}{\vec{u}}
\newcommand{\vv}{\vec{v}}
\newcommand{\vw}{\vec{w}}
\newcommand{\vW}{\vec{W}}
\newcommand{\vx}{\vec{x}}
\newcommand{\vy}{\vec{y}}
\newcommand{\vz}{\vec{z}}
\newcommand{\vzero}{\vec{0}}

\newcommand{\veta}{\vec{\eta}}

\renewcommand{\grad}{\vec{\nabla}}

% bold letters
\newcommand{\tbA}{\mathbf{A}}
\newcommand{\tbB}{\mathbf{B}}
\newcommand{\tbC}{\mathbf{C}}
\newcommand{\tbD}{\mathbf{D}}
\newcommand{\tbE}{\mathbf{E}}
\newcommand{\tbY}{\mathbf{Y}}
\newcommand{\tbZ}{\mathbf{Z}}

% sequences
\newcommand{\an}{(a_n)_{n \in \bN}}
\newcommand{\bn}{(b_n)_{n \in \bN}}
\newcommand{\sn}{(s_n)_{n \in \bN}}
\newcommand{\iinI}{_{i \in I}}
\newcommand{\iinN}{_{i \in \bN}}

% special functions
\newcommand{\at}{\textnormal{at}}
\newcommand{\ggT}{\textnormal{ggT}}
\newcommand{\kgV}{\textnormal{kgV}}
\newcommand{\id}{\textnormal{id}}
\newcommand{\Id}{\textnormal{Id}}
\newcommand{\im}{\textnormal{im}}
\newcommand{\inv}{\textnormal{inv}}
\newcommand{\ord}{\textnormal{ord}\ }
\newcommand{\rang}{\textnormal{rang}\ }
\renewcommand{\tr}{\textnormal{tr}\ }
\newcommand{\vol}{\textnormal{vol}}
\newcommand{\cond}{\textnormal{cond}}
\newcommand{\sgn}{\textnormal{sgn}}
\renewcommand{\char}{\textnormal{char}}
\newcommand{\Frac}{\textnormal{Frac}}
\newcommand{\Irr}{\textnormal{Irr}}

% groups and sets
\newcommand{\GL}{\text{GL}}
\newcommand{\SO}{\text{SO}}
\newcommand{\Hess}[1]{\text{Hess}(#1)}

\newcommand{\Grp}{\textnormal{Grp}}
\newcommand{\Mag}{\textnormal{Mag}}
\newcommand{\Mon}{\textnormal{Mon}}
\newcommand{\Ens}{\textnormal{Ens}}
\newcommand{\Hom}{\textnormal{Hom}}
\newcommand{\Aut}{\textnormal{Aut}}

\newcommand{\Mat}{\textnormal{Mat}}
\newcommand{\Matnn}{\textnormal{Mat}(n \times n)}
\newcommand{\MatBB}[3]{\textnormal{Mat}^{#1}_{#2}\left(#3\right)}



% --- THEOREM AND PROOF TYPES ---
% \newtheorem{codename}{printedname}[countedwith]
\newtheorem{lemma}{Lemma}[section]
\newtheorem{theorem}[lemma]{Satz}
\newtheorem{proposition}[lemma]{Proposition}
\newtheorem{corollary}[lemma]{Korollar}

\theoremstyle{definition}
\newtheorem{definition}[lemma]{Definition}
\newtheorem{summary}[lemma]{Zusammenfassung}
\newtheorem{example}[lemma]{Beispiel}
\newtheorem{counterexample}[lemma]{Gegenbeispiel}
\newtheorem{beobachtung}[lemma]{Beobachtung}
\newtheorem{anmerkung}[lemma]{Anmerkung}
\newtheorem{question}[lemma]{Frage}
\newtheorem{application}[lemma]{Anwendung}
\newtheorem{reminder}[lemma]{Erinnerung}
\newtheorem{konsequenz}[lemma]{Konsequenz}

% --- CUSTOM ENVIRONMENTS ---

% sketchy proofs marked with untrustworthy QED symbol
\newenvironment{proofsketch}{\begin{proof}[Beweisskizze]\renewcommand*{\qedsymbol}{\("\square"\)}}{\end{proof}}

% align with exactly one number for labeling
\NewDocumentEnvironment{nalign}{}{\equation\aligned}{\endaligned\endequation}

\makeatletter
\let\amsmath@bigm\bigm

% fancy box behind proofs
\tcolorboxenvironment{proof}{
	colback=black!5!white,
	boxrule=0pt,
	sharp corners,
	breakable,
	enhanced,
}

\usepackage{algorithm}
\usepackage{algpseudocode}

\renewcommand*\contentsname{Inhalt}
\renewcommand*\proofname{Beweis}

\pagestyle{fancy} %allows headers

\lhead{Emma Bach}
\rhead{\today}


\begin{document}
	\begin{titlepage}
	\centering
	{\Large \textsc{Mitschrieb}\par}
	\vspace{0.5cm}
	{\huge\bfseries Numerik\par}
    \vspace{0.5cm}
	{\Large\itshape Emma Bach\par}
	\vfill
	
	Basierend auf:\par 
	\vspace{1cm}
	Vorlesungen Numerik I + II von\par
	Prof. Dr. Patrick  \textsc{Dondl}\par
	
	\vfill

% Bottom of the page
	{\large \today\par}
\end{titlepage}

	\tableofcontents
	\thispagestyle{fancy}
	\chapter{Aufgabenstellung}
	In der Numerik beschäftigt man sich mit der praktischen Berechnung von Lösungen mathematischer Probleme.
	\begin{example}
		Berechne $\displaystyle \int_0^1 e^{-x^2} dx$!
	\end{example}
	\begin{example}
		Berechne $\sin(20)$!
	\end{example}
	\begin{example}
		Berechne $\sqrt{753}$!
	\end{example}
	\begin{example}
		Berechne $\displaystyle \min_{x \in [0,1]} F(x)$, für eine geeignete Funktion $F$!
	\end{example}
	\begin{example}
		Berechne $x$, sodass $f(x) = 0$!
	\end{example}
	\begin{example}
		Berechne $x \in \bR^n$, sodass $Ax = b$!
	\end{example}
	\begin{definition}
		Eine Mathematische Aufgabe in der Numerik besteht im Finden einer Lösung von
		\begin{align*}
			F(x,d) = 0
		\end{align*}
		für gegebenes Datum $d$ und gegebene Funktion $F$.
	\end{definition}
	\noindent
	Typischerweise können in akzeptabler Zeit keine exakten Lösungen gefunden werden, sondern nur Approximationen. Insbesondere stehen statt den vollen Mengen $\bQ$, $\bR$, $\bC$ etc. auch nur endlich viele \tbf{Maschinenzahlen} zur Verfügung - arbiträre reelle Zahlen benötigen unendlich viel Speicher! Rechenoperationen sind dementsprechend Fehlerbehaftet, es gibt Rundungsfehler. Außerdem gibt es in reellen Anwendungen oft \tbf{Modellfehler} und \tbf{Datenfehler}.
	\newpar
	Eine Grundlegende Idee in der Numerik ist es deshalb, eine gute Balance zwischen Exaktheit und Aufwand der Berechnung zu finden.
	\begin{example}
		Die Berechnung der Determinante einer Matrix mittels Laplaceschem Entwicklungssatz benötigt $O(n!)$ Rechenoperationen. Die Determinante mit diesem Verfahren zu berechnen, dauert sehr viel länger, als das Universum alt ist.
		\newpar
		Besser: Matrix (approximativ) auf Dreiecksgestalt bringen und die Diagonalelemente multiplizieren.
	\end{example}
	\begin{definition}
		Eine Mathematische Aufgabe heißt \tbf{wohlgestellt}, wenn zu geeigneten Daten $d$ eindeutige Lösungen $x$ existieren, und diese stetig von $d$ abhängt. Andernfalls ergibt die Suche nach einer numerischen Lösung wenig Sinn. Für wohlgestellte Probleme existiert eine Lösungsfunktion $\phi$, sodass $x = \phi(d)$ das Problem löst, d.h. $f(\phi(d),d) = 0$.
	\end{definition}
	\begin{definition}
		Ein numerischer Algorithmus zur näherungsweisen Lösung einer wohlgestellten Aufgabe $\phi$ ist eine Abbildung $\tilde{\phi}$, die durch Hintereinanderausführung möglicherweise fehlerbehafteter elementarer Rechenoperationen definiert ist, also 
		\begin{align*}
			\tilde{\phi} = f_j \circ f_{j-1} \circ \hdots \circ f_1
		\end{align*}
	\end{definition}
	\begin{definition}
		Der \tbf{Aufwand} eines Verfahrens $\tilde \phi$ ist die Anzahl der benötigten elementaren Rechenschritte. Typischerweise interessiert uns nicht die exakte Anzahl an Schritten, sondern nur die Größenordnung.
	\end{definition}
	\begin{proposition}
		Das Gaußverfahren hat Aufwand $\cO(n^3)$.
	\end{proposition}
\chapter{Numerische Lineare Algebra}
\section{Matrixfaktorisierung}
\subsection{Dreiecksmatrizen}
\begin{definition}
	Eine Matrix $L \in \bR^{n \times n}$ heißt \tbf{untere Dreiecksmatrix}, falls $\forall i < j : l_{ij} = 0$.
\end{definition}
\begin{definition}
	Eine Matrix $U \in \bR^{n \times n}$ heißt \tbf{obere Dreiecksmatrix}, falls $U^\top$ eine untere Dreiecksmatrix ist.
\end{definition}
\begin{definition}
	Eine Dreiecksmatrix heißt \tbf{normalisiert}, falls alle ihre Diagonaleinträge $1$ sind.
\end{definition}
\begin{definition}
	Eine Matrix heißt \tbf{regulär}, wenn sie invertierbar ist.
\end{definition}
\begin{lemma}
	Die quadratischen oberen (bzw. unteren) Dreiecksmatrizen bilden unter Matrixmultiplikation eine Gruppe.
\end{lemma}
\noindent
Lineare Gleichungssysteme mit regulärer Dreiecksmatrix lassen sich leicht lösen. Sei $U \in \bR^{n \times n}$ eine reguläre obere Dreiecksmatrix und $b \in \bR^n$. Wir berechnen $x \in \bR^n$ folgendermaßen:
\begin{enumerate}
	\item for i = n : -1 : 1:
	\begin{enumerate}
		\item $\displaystyle x_i = \left(b_i - \sum_{j = i+1}^n u_{ij}x_j\right) \cdot \frac{1}{u_{ii}}$
	\end{enumerate}
	\item end.
\end{enumerate}
Der Aufwand dieses Verfahrens ist $\cO(n^2)$. Ein analoger Algorithmus existiert für untere Dreiecksmatrizen.
\subsection{$LU$-Zerlegung}
Falls für eine reguläre Matrix $A \in \bR^{n \times n}$ eine Zerlegung $A = LU$ in eine untere Dreiecksmatrix $U$ und eine obere Dreiecksmatrix $L$ gegeben ist, so lässt sich das lineare Gleichungssystem $Ax = b$ in zwei Schritten lösen:
\begin{enumerate}
	\item Löse $Ly = b$.
	\item Löse $Ux = y$.
\end{enumerate}
\begin{definition}
	Eine Faktorisierung $A = LU$ mit unterer Dreiecksmatrix $L \in \bR^{n \times n}$ und oberer Dreiecksmatrix $U \in \bR^{n \times n}$ heißt \tbf{$LU$-Zerlegung} von $A$. Die Zerlegung heißt \tbf{normalisiert}, falls $L$ normalisiert ist.
\end{definition}
\begin{theorem}
	Für jede reguläre Matrix $A \in \bR^{n \times n}$ sind äquivalent:
	\begin{enumerate}
		\item Es existiert eine eindeutige normalisierte $LU$-Zerlegung.
		\item Alle Untermatrizen $A_k = (a_{ij})_{(i,j) \in (1, \hdots, k)^2}$ sind regulär. 
	\end{enumerate}
\end{theorem}
\begin{proof}
	\phantom{}
	\begin{itemize}
		\item[$\rightarrow$] Ist $A$ regulär, so sind auch $L$ und $U$ regulär. Damit sind von $L$ und $U$ alle Diagonaleinträge nicht null. Somit sind auch die Untermatrizen $L_k$ und $U_k$ regulär, somit auch die Untermatrizen $A_k = L_k U_k$.
		\item[$\leftarrow$] Für $n = 1$ ist die Aussage klar. Sei nun angenommen, die Aussage gelte für Matrizen der Größe $(n - 1) \times (n - 1)$. Damit existieren Matrizen $L_{n-1}, U_{n-1}$, sodass $A_{n-1} = L_{n-1}U_{n-1}$ eine normalisierte $LU$-Zerlegung ist. Seien nun $\binom{b}{a_{nn}}$ die letzte Spalte und $(c^\top, a_{nn})$ die letzte Zeile von $A$. Die Aussage ist bewiesen, wenn geeignete $l,u \in \bR^{n-1}$ und $r \in \bR$ existieren, sodass
		\begin{align*}
			\begin{pmatrix}
				A_{n-1} & b\\
				c^\top & a_{nn}
			\end{pmatrix}
			&=
			\begin{pmatrix}
				L_{n-1} & 0\\
				l^\top & 1
			\end{pmatrix}
			\begin{pmatrix}
				U_{n-1} & u\\
				0 & r
			\end{pmatrix}\\
			&=
			\begin{pmatrix}
				L_{n-1}U_{n-1} & L_{n-1}u\\
				(U^\top_{n-1} l)^\top & l^\top u + r
			\end{pmatrix}\\
			&=
			\begin{pmatrix}
				A_{n-1} & L_{n-1}u\\
				(U^\top_{n-1} l)^\top & l^\top u + r
			\end{pmatrix}
		\end{align*}
		Wir brauchen also $L_{n-1}u = b$, $U_{n-1}^\top l = c$, und $a_{nn} = l^\top u + r$. Durch Regularität von $L_{n-1}$ und $U_{n-1}$ existieren eindeutige Lösungen $u, l$, der ersten beiden Gleichungen, somit ist auch $r$ festgelegt.
	\end{itemize}
\end{proof}
\begin{corollary}
	\phantom{}
	\begin{itemize}
		\item Jede positiv definite Matrix besitzt eine eindeutige $LU$-Zerlegung.
		\item Jede strikt diagonaldominante Matrix, also jede Matrix $A$ mit $\sum_{j \in 1, \hdots, n, i \neq j} \abs{a_ij} < \abs{a_{ii}}$ besitzt eine eindeutige $LU$-Zerlegung.
		\item Die Matrix $A = \begin{pmatrix}
			0 & 1\\
			1 & 0
		\end{pmatrix}$
		besitzt keine $LU$-Zerlegung.
		\item Die Nullmatrix besitzt zwar $LU$-Zerlegungen, diese sind aber nicht eindeutig.
	\end{itemize}
\end{corollary}
\begin{lemma}
	Falls $A = LU$ eine normalisierte $LU$-Zerlegung von $A$ ist, so gilt
	\begin{align*}
		a_{ik} = u_{ik} + \sum_{j = 1}^{i-1} l_{ij} u_{jk}
	\end{align*}
	und
	\begin{align*}
		a_{ki} = l_{ki} u_{ii} + \sum_{j = 1}^{i-1} l_{kj} u_{ji}
	\end{align*}
\end{lemma}
\begin{proof}
	Es gilt $l_{ij} = 0$ für $j > i$ und $l_{ii} = 1$. Es gilt außerdem $u_{ij} = 0$ für $j < i$. Die Formeln folgen direkt aus der Definition des Matrixprodukts.
\end{proof}
Diese Formeln lassen sich für $i \leq k$ nach $u_{ik}$ auflössen und für $k > i$ nach $l_{ki}$ auflösen. Wir erhalten folgenden Algorithmus:
\begin{algorithm}
\begin{algorithmic}[1]
	\For{$i = 1, i \leq n, i++$}
		\For{$k = i, k \leq n, k++$}
			\State $\displaystyle u_{ik} \gets a_{ik} - \sum_{j = 1}^{i-1}l_{ij}u_{jk}$
		\EndFor
		\For{$k = i+1, k \leq n, k++$}
		\State $\displaystyle l_{ki} \gets \frac{1}{u_{ii}} \cdot \left(a_{ki} - \sum_{j = 1}^{i-1}l_{kj}u_{ji}\right)$
		\EndFor
	\EndFor
\end{algorithmic}
\end{algorithm}
\begin{proposition}
	Der Rechenauftrag dieses Algorithmus beträgt $O(n^3)$.
\end{proposition}
\begin{proposition}
	Es ist nicht mehr Speicher nötig, als sowieso für $A$ benötigt wird. Die Einträge von $A$ können im Speicher einfach sukzessiv durch die jeweiligen Einträge von $L$ bzw. $U$ ersetzt werden.
\end{proposition}
\begin{definition}
	Ein numerisches Problem $\phi$ heißt \tbf{schlecht Konditioniert}, wenn kleine Unterschiede in der Eingabe zu großen Unterschieden in der korrekten Lösung führen, also wenn
	\begin{align*}
		\frac{\abs{\phi(\tilde{x}) - \phi(x)}}{\abs{\phi(x)}} >> \frac{\abs{\tilde{x} - x}}{\abs{x}}
	\end{align*}
	Ansonsten heißt die Aufgabe \tbf{gut konditioniert}.
\end{definition}
\begin{definition}
	Ein Verfahren $\tilde{\phi}$ heißt \tbf{numerisch instabil}, wenn eine Störung $\tilde{x}$ existiert, sodass der durch Rundungsfehler verursachte relative Fehler erheblich größer ist als der rein durch die Störung verursachte Fehler.
\end{definition}
\begin{example}
	Sei 
	\begin{align*}
		A = \begin{pmatrix}
			1 & 1\\ 
			1 & 1 + \epsilon
		\end{pmatrix}
	\end{align*}
	Für $\epsilon \in \bR^+$. Es gilt
	\begin{align*}
		A^{-1} = \begin{pmatrix}
			1 + \frac{1}{\epsilon} & -\frac{1}{\epsilon}\\ 
			- \frac{1}{\epsilon} & \frac{1}{\epsilon}
		\end{pmatrix}
	\end{align*}
	\begin{align*}
		A^{-1} \binom{1}{1} = \binom{1}{0}. \qquad \qquad A^{-1} \binom{1}{1 + \epsilon} = \binom{0}{1}.
	\end{align*}
\end{example}
\noindent Eine kleine Störung in den Daten $b$ des linearen Gleichungssystems $Ax = b$ führt zu Problemen der Größenordnung $1$!!! So können wir keine Numerik machen!!! Dieses Problem ist \tbf{schlecht konditioniert}.
\begin{example}
	Sei 
	\begin{align*}
		A = \begin{pmatrix}
			\epsilon & 1\\ 
			1 & 0
		\end{pmatrix},
	\end{align*}
	also
	\begin{align*}
		A^{-1} = \begin{pmatrix}
			0 & 1\\ 
			1 & -\epsilon
		\end{pmatrix}.
	\end{align*}
	So haben wir kein Problem bei der Berechnung von $A^{-1}b$ - die Aufgabe ist gut konditioniert. Sagen wir nun, wir versuchen, das Gleichungssystem effizient durch $LU$-Zerlegung zu lösen. Die $LU$-Zerlegung von $A$ ist jedoch gegeben durch
	\begin{align*}
		A = 
		\begin{pmatrix}
			1 & 0\\ 
			\frac{1}{\epsilon} & 1
		\end{pmatrix}
		\begin{pmatrix}
			\epsilon & 1\\ 
			0 & \frac{1}{\epsilon}
		\end{pmatrix},
	\end{align*}
	die Berechnung von $L^{-1}b$ und  $U^{-1}b$ führt nun wieder zu großen Rundungsfehlern. Aus unserer Idee entsteht also ein \tbf{instabiler Algorithmus.}
\end{example}
\begin{theorem}
	\theoremname{Cholesky-Zerlegung:} Sei $A \in \bR^{n \times n}$ symmetrisch und positiv definit. So existiert eine eindeutige untere Dreiecksmatrix $L$, sodass
	\begin{align*}
		A = LL^\top.
	\end{align*} 
	und $l_{ii} > 0$
\end{theorem}
\end{document}