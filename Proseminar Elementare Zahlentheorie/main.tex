\documentclass{report}

% custom margins
\usepackage[a4paper,margin=1.5in]{geometry}
\renewcommand{\baselinestretch}{1.2}

% emma's long list of custom macros and universally used packages
%AMS packages - Symbols, "Theorem" and "Proof" Environments
\usepackage{amsmath}
\usepackage{amssymb}
\usepackage{amsthm}
\usepackage{physics}
\renewcommand{\div}{\tn{div}~}

% Nicer Headers and Footers
\usepackage{fancyhdr}
\usepackage[yyyymmdd]{datetime}
\renewcommand{\dateseparator}{-}

% Table of Contents
\usepackage[titles]{tocloft}
\usepackage[titletoc]{appendix}

% Tikz and Graphics
\usepackage{tikz}
\usepackage{tikz-cd}
\usepackage{xcolor}
\usepackage[many]{tcolorbox}

% Nicer Underlining
\usepackage{contour}
\usepackage{ulem}

% Multiple Text Columns
\usepackage{multicol}

% FloatBarrier
\usepackage{placeins} 

% hyperref should be last apparently
\usepackage{hyperref}

%--- RENEWED COMMANDS ---

% Variant Greek Letters
\renewcommand\epsilon{\varepsilon}
\renewcommand\phi{\varphi}

% Nicer Table of Contents
\renewcommand\cftsecdotsep{\cftdot}
\renewcommand\cftsubsecdotsep{\cftdot}

%--- TEXT FORMATING ---

% nice underlining
\renewcommand{\ULdepth}{1.6pt}
\contourlength{0.8pt}
\newcommand{\ul}[1]{%
	\uline{\phantom{#1}}%
	\llap{\contour{white}{#1}}%
}

% shorthand
\newcommand{\tn}[1]{\textnormal{#1}}
\newcommand{\tbf}[1]{\textbf{#1}}
\newcommand{\tit}[1]{\textit{#1}}
\newcommand{\ttt}[1]{\texttt{#1}}

\newcommand{\mc}[1]{\mathcal{#1}}

\newcommand{\ol}[1]{\overline{{#1}}}

% Underlined, bold, non-cursive Theorem Name
\newcommand{\theoremname}[1]{\ul{\textnormal{\tbf{#1}}}}

% Starts a new paragraph without indentation
% and with an empty line between paragraphs
\newcommand*{\newpar}{\par\vspace{\baselineskip}\noindent}

% Make inf height match sup height
\renewcommand{\inf}{\mathop{\mathrm{inf}\vphantom{\mathrm{sup}}}}

% Make tildes more readable
\renewcommand{\tilde}{\widetilde}

%--- RELATION SYMBOLS AND OPERATORS ---
\newcommand{\surj}{\twoheadrightarrow}
\newcommand{\inj}{\hookrightarrow}
\newcommand{\iso}{\overset{\sim}{\rightarrow}}
\newcommand{\symdiff}{\vartriangle}
\newcommand{\trans}{\twoheadrightarrow}
\newcommand{\tensor}{\otimes}
\newcommand{\bigmid}{~\middle|~}

%--- STUFF IN FANCY BRACKETS ---
\newcommand{\scalar}[2]{\left\langle #1, #2 \right\rangle}
\newcommand{\angles}[1]{\left\langle #1 \right\rangle}
\newcommand{\lr}{\qty}


%--- LETTERS ---
\newcommand{\dmu}{\ d\mu}
\newcommand{\dist}{\textnormal{dist}}
\newcommand{\spt}{\textnormal{spt}}

% mathbb
\newcommand{\bC}{\mathbb{C}}
\newcommand{\bF}{\mathbb{F}}
\newcommand{\bN}{\mathbb{N}}
\newcommand{\bQ}{\mathbb{Q}}
\newcommand{\bR}{\mathbb{R}}
\newcommand{\bZ}{\mathbb{Z}}

% mathcal
\newcommand{\cA}{\mathcal{A}}
\newcommand{\cB}{\mathcal{B}}
\newcommand{\cC}{\mathcal{C}}
\newcommand{\cD}{\mathcal{D}}
\newcommand{\cE}{\mathcal{E}}
\newcommand{\cF}{\mathcal{F}}
\newcommand{\cH}{\mathcal{H}}
\newcommand{\cI}{\mathcal{I}}
\newcommand{\cL}{\mathcal{L}}
\newcommand{\cM}{\mathcal{M}}
\newcommand{\cN}{\mathcal{N}}
\newcommand{\cO}{\mathcal{O}}
\newcommand{\cP}{\mathcal{P}}
\newcommand{\cQ}{\mathcal{Q}}
\newcommand{\cR}{\mathcal{R}}
\newcommand{\cS}{\mathcal{S}}
\newcommand{\cT}{\mathcal{T}}
\newcommand{\cW}{\mathcal{W}}
\newcommand{\cX}{\mathcal{X}}
\newcommand{\cZ}{\mathcal{Z}}

% mathfrac
\newcommand{\fa}{\mathfrak{a}}

% real numbers in fancy costumes
\newcommand{\barR}{\ol{\mathbb{R}}}
\newcommand{\bRnn}{\mathbb{R}^{n \times n}}
\newcommand{\bRmn}{\mathbb{R}^{m \times n}}
\newcommand{\bRnm}{\mathbb{R}^{n \times m}}

% vectors
\renewcommand{\va}{\vec{a}}
\renewcommand{\vb}{\vec{b}}
\newcommand{\vc}{\vec{c}}
\newcommand{\ve}{\vec{e}}
\newcommand{\vF}{\vec{F}}
\newcommand{\vh}{\vec{h}}
\newcommand{\vp}{\vec{p}}
\newcommand{\vr}{\vec{r}}
\newcommand{\vs}{\vec{s}}
\newcommand{\vT}{\vec{T}}
\renewcommand{\vu}{\vec{u}}
\newcommand{\vv}{\vec{v}}
\newcommand{\vw}{\vec{w}}
\newcommand{\vW}{\vec{W}}
\newcommand{\vx}{\vec{x}}
\newcommand{\vy}{\vec{y}}
\newcommand{\vz}{\vec{z}}
\newcommand{\vzero}{\vec{0}}

\newcommand{\veta}{\vec{\eta}}

\renewcommand{\grad}{\vec{\nabla}}

% bold letters
\newcommand{\tbA}{\mathbf{A}}
\newcommand{\tbB}{\mathbf{B}}
\newcommand{\tbC}{\mathbf{C}}
\newcommand{\tbD}{\mathbf{D}}
\newcommand{\tbE}{\mathbf{E}}
\newcommand{\tbY}{\mathbf{Y}}
\newcommand{\tbZ}{\mathbf{Z}}

% sequences
\newcommand{\an}{(a_n)_{n \in \bN}}
\newcommand{\bn}{(b_n)_{n \in \bN}}
\newcommand{\sn}{(s_n)_{n \in \bN}}
\newcommand{\iinI}{_{i \in I}}
\newcommand{\iinN}{_{i \in \bN}}

% special functions
\newcommand{\at}{\textnormal{at}}
\newcommand{\ggT}{\textnormal{ggT}}
\newcommand{\kgV}{\textnormal{kgV}}
\newcommand{\id}{\textnormal{id}}
\newcommand{\Id}{\textnormal{Id}}
\newcommand{\im}{\textnormal{im}}
\newcommand{\inv}{\textnormal{inv}}
\newcommand{\ord}{\textnormal{ord}\ }
\newcommand{\rang}{\textnormal{rang}\ }
\renewcommand{\tr}{\textnormal{tr}\ }
\newcommand{\vol}{\textnormal{vol}}
\newcommand{\cond}{\textnormal{cond}}
\newcommand{\sgn}{\textnormal{sgn}}
\renewcommand{\char}{\textnormal{char}}
\newcommand{\Frac}{\textnormal{Frac}}
\newcommand{\Irr}{\textnormal{Irr}}

% groups and sets
\newcommand{\GL}{\text{GL}}
\newcommand{\SO}{\text{SO}}
\newcommand{\Hess}[1]{\text{Hess}(#1)}

\newcommand{\Grp}{\textnormal{Grp}}
\newcommand{\Mag}{\textnormal{Mag}}
\newcommand{\Mon}{\textnormal{Mon}}
\newcommand{\Ens}{\textnormal{Ens}}
\newcommand{\Hom}{\textnormal{Hom}}
\newcommand{\Aut}{\textnormal{Aut}}

\newcommand{\Mat}{\textnormal{Mat}}
\newcommand{\Matnn}{\textnormal{Mat}(n \times n)}
\newcommand{\MatBB}[3]{\textnormal{Mat}^{#1}_{#2}\left(#3\right)}



% --- THEOREM AND PROOF TYPES ---
% \newtheorem{codename}{printedname}[countedwith]
\newtheorem{lemma}{Lemma}[section]
\newtheorem{theorem}[lemma]{Satz}
\newtheorem{proposition}[lemma]{Proposition}
\newtheorem{corollary}[lemma]{Korollar}

\theoremstyle{definition}
\newtheorem{definition}[lemma]{Definition}
\newtheorem{summary}[lemma]{Zusammenfassung}
\newtheorem{example}[lemma]{Beispiel}
\newtheorem{counterexample}[lemma]{Gegenbeispiel}
\newtheorem{beobachtung}[lemma]{Beobachtung}
\newtheorem{anmerkung}[lemma]{Anmerkung}
\newtheorem{question}[lemma]{Frage}
\newtheorem{application}[lemma]{Anwendung}
\newtheorem{reminder}[lemma]{Erinnerung}
\newtheorem{konsequenz}[lemma]{Konsequenz}

% --- CUSTOM ENVIRONMENTS ---

% sketchy proofs marked with untrustworthy QED symbol
\newenvironment{proofsketch}{\begin{proof}[Beweisskizze]\renewcommand*{\qedsymbol}{\("\square"\)}}{\end{proof}}

% align with exactly one number for labeling
\NewDocumentEnvironment{nalign}{}{\equation\aligned}{\endaligned\endequation}

\makeatletter
\let\amsmath@bigm\bigm

% fancy bar next to proofs
\tcolorboxenvironment{proof}{
	colback=white,
	boxrule=0pt,
	leftrule=0.5mm,
	before skip=0.75cm,
	after skip=0.75cm,
	sharp corners,
	breakable,
	enhanced,
}

\usepackage{algorithm}
\usepackage{algpseudocode}

\renewcommand*\contentsname{Inhalt}
\renewcommand*\proofname{Beweis}

\pagestyle{fancy} %allows headers

\lhead{Emma Bach}
\rhead{\today}


\begin{document}
	\begin{titlepage}
	\centering
	{\Large \textsc{Mitschrieb}\par}
	\vspace{0.5cm}
	{\huge\bfseries Numerik\par}
    \vspace{0.5cm}
	{\Large\itshape Emma Bach\par}
	\vfill
	
	Basierend auf:\par 
	\vspace{1cm}
	Vorlesungen Numerik I + II von\par
	Prof. Dr. Patrick  \textsc{Dondl}\par
	
	\vfill

% Bottom of the page
	{\large \today\par}
\end{titlepage}

	\tableofcontents
	\thispagestyle{fancy}
	\chapter{Kettenbrüche II}
	\section{Wiederholung}
	\begin{definition}
		Ein regulärer Kettenbruch ist ein Bruch der Form
		\begin{align*}
			a_0 + \frac{1}{a_1 + \frac{1}{a_2 + \hdots}}
		\end{align*}
	\end{definition}
	\begin{proposition}
		Es gilt folgende Rekursionsformel:
		\begin{align*}
			p_{-2} := 0, p_{-1} := 1, p_i = ap_{i-1} + p_{i - 2}\\
			q_{-2} := 1, p_{-1} := 0, q_i = aq_{i-1} + q_{i - 2}
		\end{align*}
	\end{proposition}
	\begin{proposition}
		Es gilt:
		\begin{enumerate}
			\item Für $i \geq -1$ gilt $p_{i-1}q_i - p_iq_{i-1} = (-1)^i$
			\item Für $i \geq 0$ gilt 
			\begin{align*}
				\alpha - \frac{p_i}{q_i} = \frac{(-1)^i}{q_i(q_i\alpha_{i+1} + q_{i - 1})}
			\end{align*}
		\end{enumerate}
	\end{proposition}
	\section{Neues}
	\begin{definition}
		Eine rationale Zahl $\frac{a}{b}$ mit $a \in \bZ$, $b \in \bN$ heißt \tbf{beste Näherung} einer reellen Zahl $\alpha \in \bR$, falls für alle $c \in \bZ$ und $d \in \bN_1$ mit $\frac{a}{b} \neq \frac{c}{d}$ und $d  \leq b$
		\begin{align*}
			\abs{d\alpha - c} > \abs{b\alpha - a}
		\end{align*}
		gilt.
	\end{definition}
	
	Das Ziel des Vortrags ist, zu zeigen, dass jede beste Näherung von $\alpha \in \bR$ auch ein Näherungsbruch von $\alpha$ ist.
	
	\begin{lemma}\phantom{}
		\begin{enumerate}
			\item Sei $\alpha = \frac{p_k}{q_k} \in \bQ$. Dann hat man für alle $c \in \bZ$, $a \in \bN_1$ die Ungleichung
			\begin{align*}
				\abs{q_k \alpha - p_k} \leq \abs{d\alpha - c}
			\end{align*}
			mit Gleichheit für $\frac{c}{d} = \frac{p_k}{q_k}$.
			\item Gilt $q_k > 1$ für $\alpha \in \bQ$, so hat man für alle $c \in \bZ, d \in \bN$ mit $d < q_k$ die Ungleichung $\abs{q_{k-1}\alpha - p_{k-1}} \leq \abs{d \alpha c}$, mit Gleichheit genau dann, wenn $(c,d) = (p_{k-1},q_{k-1})$ oder $(c,d) = (p_k - p_{k-1}, q_k - q_{k-1})$ ist.
			\item Ist $\alpha \in \bQ, 0 \leq i \leq k-2$ oder $\alpha \in \bR \setminus \bQ$ und nicht gleichzeitig $i = 0$ und $a_1 = 1$, dann gilt für alle $c \in \bZ, d \in \bN_1$ mit $d < q_{i+1}$ die Ungleichung
			\begin{align*}
				\abs{q_i \alpha - p_i} \leq \abs{d \alpha - c}
			\end{align*}
			mit Gleichheit genau für $c = p_i, d = q_i$.
		\end{enumerate}
	\end{lemma}
	\begin{proof}
		\begin{enumerate}
			\item Da $\alpha \in \bQ$ gilt auch $\alpha = \frac{p_k}{q_k}$, also folgt $0 = \abs{q_k \alpha - p_j} \leq \abs{d\alpha - c}$.
			\item Betrachte das lineare Gleichungssystem
			\begin{align*}
				p_i x + p_{i+1} y &= c,\\
				q_i x + q_{i+1} y &= d
			\end{align*}
			In Matrixschreibweise:
			\begin{align*}
				\begin{pmatrix}
					p_i & p_{i+1}\\
					q_i & q_{i+1}
				\end{pmatrix}
				\binom{x}{y}
				=
				\binom{c}{d}
			\end{align*}
			Dieses Gleichungssystem ist Lösbar, wenn die Determinante nicht Null ist, also $p_{i}q_{i+1} - q_ip_{i+1} \neq 0$. Dass dies gilt, wurde bereits letzte Woche gezeigt. Insbesondere existieren die Lösungen gemäß der Kramerschen Regel sogar in $\bZ$. Es gilt außerdem $x \neq 0$, da sonst $q_{i+1} \mid d$ folgen würde, was unserer Annahme $d \leq q_{i + 1}$ widerspricht.
			\newpar
			Durch Umstellen des Gleichungssystems folgt 
			\begin{align*}
				x(q_{k-1} \alpha - p_{k-1}) = d\alpha - c.
			\end{align*}
			Da $x \in \bZ \setminus \{0\}$ folgt $\abs{q_{k-1} \alpha - p_{k-1}} \leq {d\alpha - c}$.
			\item In diesem Fall ist $c = \lambda p_i$ und $d = \lambda q_i$ $\lambda \in \bN_{\geq 2}$, da $q_i \alpha - p_i \neq 0$ und $\abs{q_i \alpha - p_i} < \lambda\abs{q_i\alpha - p_i} = \abs{d \alpha - c}$.
			\newpar
			Sei nun also $(c,d)$ von allen $(\lambda p_i, \lambda q_i)$ verschieden. Dann gilt für die Lösung des LGS $xy < 0$, denn $y = 0$ führt zu $\frac{p_i}{q_i} = \frac{c}{d}$ und $xy > 0$ ist im Widerspruch zu $0 < d < q_{i+1}$ und der Gleichheit in 1).
			\newpar
			Nach Wiederholung haben $q_i\alpha - p_i$ und $q_{i+1}\alpha - p_{i+1}$ unterschiedliche Vorzeichen. Da $0 \leq i \leq k-2$ sind auch beide Ungleich $0$.
			\newpar
			Insgesamt folgt, dass $x(q_i \alpha - p_{i})$ und $y(q_{i + 1}\alpha - p_{i+1})$ gleiches Vorzeichen haben. Dementsprechend gilt Gleichheit in der folgenden Dreiecksungleichung:
			\begin{align*}
				\abs{d \alpha - c}  &= \abs{x(q_i \alpha - p_i) + y(q_{i+1}\alpha - p_{i+1})}\\
								    &= \abs{x}\abs{x(q_i \alpha - p_i)} + \abs{y}\abs{(q_{i+1}\alpha - p_{i+1})}\\
								    &> \abs{q_i \alpha - p_i}
			\end{align*}
			
		\end{enumerate}
	\end{proof}
	\begin{theorem}
		Jede beste Näherung wird als Näherungsbruch angenommen.
	\end{theorem}
	\begin{proof}
		Sei $\frac{a}{b}$ eine beste Näherung, aber $\frac{a}{b} \neq \frac{q_i}{p_i}$ für alle $i$.
		\begin{description}
			\item[Fall 1:] Sei $\alpha \in \bQ$ und $q_k \leq b$. Wähle $c = p_n, d = q_n$. Dann gilt $\frac{c}{d} \neq \frac{a}{b}$, $d \leq b$ und nach Lemma 1 folgt
			\begin{align*}
				0 = \abs{q_k \alpha - p_k} = \abs{d\alpha - c} < \abs{b \alpha - c},
			\end{align*} 
		 	also war $\frac{a}{b}$ keine beste Näherung.
			\item[Fall 2:] Sei $\alpha \in \bQ$ mit $q_k > b$ oder $\alpha \in \bR \setminus \bQ$. Fixiere $i$, sodass $q_i \leq b \leq q_{i+1}$. Man erhält $i < k$ und 
			\begin{align*}
				1 < q_{i+1} = a_i q_i + q_{i-1},
			\end{align*}
			also $i \geq 1$ oder $a_i > 1$. Nach Lemma 2 und Lemma 3 und $b < q_{i+1}$ gilt
			\begin{align*}
				\abs{q_i \alpha - p_i} \leq \abs{b \alpha - a}
			\end{align*}
			Die Wahl von $c = p_i$ und $d = q_i$ liefert dann wieder $\frac{c}{d} \neq \frac{a}{b}$, $d \leq b$, und
			\begin{align*}
				 \abs{d \alpha - c} < \abs{b\alpha - c},
			\end{align*}
			also war $\frac{a}{b}$ wieder keine beste Näherung.
		\end{description}
	\end{proof}
	\begin{theorem}
		Sei $\alpha \in \bR$, $p \in \bZ$, $q \in \bN$ und
		\begin{align*}
			\abs{\alpha - \frac{p}{q}} < \frac{1}{2q^2}
		\end{align*}
		dann ist $\frac{p}{q}$ eine beste Näherung von $\alpha$, und somit insbesondere ein Näherungsbruch.
	\end{theorem}
	\begin{proof}
		Angenommen, $\frac{p}{q}$ wäre keine beste Näherung. Dann gibt es $c \in \bZ$, $d \in \bN$ mit $d \leq q$ und $\frac{p}{q} \neq \frac{c}{d}$, sodass
		\begin{align*}
			\abs{d\alpha - c} \leq \abs{q\alpha - c}
		\end{align*}
		Nach Vorraussetzung gilt
		\begin{align*}
			\abs{q\alpha - p} < \frac{1}{2q},
		\end{align*}
		also auch
		\begin{align*}
			\abs{d\alpha - c} < \frac{1}{2q}.
		\end{align*}
		Nun gilt
		\begin{align*}
			\frac{1}{qb} &\leq \frac{p}{q} - \frac{c}{d}\\
						 &\leq \abs{\alpha - \frac{p}{q}} +
						 	   \abs{\alpha - \frac{c}{d}}\\
						 &\leq \frac{1}{2q^2} + \frac{1}{2qd}
		\end{align*}
		Es folgt $\displaystyle \frac{1}{qd} < \frac{q + d}{2q^2d}$, also $2q < d + q$, also $q < d$, was ein Widerspruch zur Annahme ist.
	\end{proof}
	\begin{theorem}
		Sei $\alpha \in \bR$. Angenommen, der $0$-te Näherungsbrüchen von $\alpha$ hat nicht die Form $[0,2]$, $[a_0,1,a_2,\hdots,a_k]$, oder $[a_0,1,a_2, \hdots]$. Dann ist jeder Näherungsbruch von $\alpha$ eine beste Näherung.
	\end{theorem}
	\begin{proof}
		Wir müssen zeigen, dass für $\frac{p_i}{q_i}$ ($i \geq 1$ in den Ausnahmefällen) für alle $c \in \bZ$, $d \in \bN$ mit $\frac{c}{d} \neq \frac{p}{q}$ der Satz gilt.
		\begin{description}
			\item[Fall 1:] Sei $\alpha \in \bQ, i = k$. Dann gilt der Satz nach Lemma 1.
			\item[Fall 2:] Sei $\alpha \in \bQ, i = k-1$. Dann gilt nach Vorraussetzung $d \leq q_i = q_{k-1}$, also gilt nach Lemma 2 $\abs{q_{k-1}\alpha . p_{k-1}} \leq \abs{d \alpha - c}$.
			\newpar
			Da nicht $k = 1$, $\alpha_k = 2$ gilt, folgt $q_k > 2q_{k-1}$. Es kann also keine Gleichheit eintreten, da $(c,d) \neq (p_{k-1}, q_{k-1})$ und $(c,d) \neq (p_k - p_{k-1}, q_k - q_{k-1})$, da $d \leq q_{k-1} < q_k - q_{k-1}$. 
			\item[Fall 3:] Sei $\alpha \in \bQ$, $0 \leq i \leq k - 2$ oder $\alpha \in \bR \setminus Q$ nach Vorraussetzung ist $d \leq q_i \leq q_{i+1}$ und nach Ausnahmen ist nicht gleichzeitig $i = 0$ und $a_1 = 1$. Also folgt $\abs{q_i \alpha - p_i} < \abs{d\alpha - c}$, da $c \neq p_i$, $\alpha \neq q_i$.
		\end{description}
	\end{proof}
\end{document}